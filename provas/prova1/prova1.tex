

\begin{center}
  Universidade Federal de Mato Grosso do Sul \\
  Sistemas de Informação - Campus do Pantanal \\
  \vspace{1em}
  \textbf{Estruturas de Dados - Primeira prova} \\
  \vspace{1em}
  \today \\
  \vspace{1em}
  Aluno: \makebox[10cm]{\hrulefill} Nota: \makebox[2cm]{\hrulefill}


  \vspace{1em}
  \textbf{Observações}: 
  \begin{itemize}
    \item Para cada inserção ou remoção, faça um quadro mostrando o passo, e indique se está inserindo, removendo e que número
    \item Quando pertinente, exiba o fator de balanceamento.
    \item Indique que rotação está fazendo em cada operação.
  \end{itemize}
\end{center}
  \extraheadheight{-0.8in}
  % \end{coverpages}
  \vspace{3em}
  \vspace{0.1in}
  \hrule
 

\begin{questions}
  \question \textbf{Árvore Binária de busca}
  \begin{parts}
    \part[2] Inserir 13, 10, 8, 9, 12, 20, 6, 5, 17, 1

  \part[2] Dada a árvore anterior, agora você deverá remover: 8, 9, 12, 20, 6
  \end{parts}
  
  \question \textbf{Árvore AVL}
  \begin{parts}
    \part[2] Inserir: 3, 46, 41, 11, 5, 24, 20, 44, 48, 39
    \part[2] Dada a árvore anterior, agora você deverá remover: 11, 41, 39, 44, 20
  \end{parts}
 
 
  \question[2] \textbf{Árvore AVL} - Dada a árvore abaixo, remover: 47, 10, 31, 41, 46, 55, 39, 17, 12, 14, 

  \begin{figure}[!h]
    \centering
    \caption{Árvore Inicial}
    \begin{tikzpicture}[->,>=stealth',
    % level/.style={sibling distance = 8cm/#1, level distance = 1.5cm},
   level 1/.style={sibling distance=15em, level distance = 3em},
    level 2/.style={sibling distance=8em, level distance = 3em},
    level 3/.style={sibling distance=4em, level distance = 3em},
    level 4/.style={sibling distance=2em, level distance = 3em},
    level 5/.style={sibling distance=1em, level distance = 3em} ]
  \node [wv] (r){30}
  child { node [wv] {10}
          child { node [wv] {4}
            child { node [wv] {2}
              child {node [nil] {}}
              child {node [nil] {}}
            }
            child { node [wv] {8}
              child {node [nil] {}}
              child {node [nil] {}}
            }
          }
          child { node [wv] {17}
            child { node [wv] {12}
              child {node [nil] {}}
              child { node [wv] {14}
                child {node [nil] {}}
                child {node [nil] {}}
              }
            }
            child { node [wv] {27}
              child {node [nil] {}}
              child {node [nil] {}}
            }
          }
        }
  child { node [wv] {41}
          child { node [wv] {39}
            child { node [wv] {31}
              child {node [nil] {}}
              child {node [nil] {}}
            }
            child {node [nil] {}}
          }
          child { node [wv] {47}
            child { node [wv] {46}
              child {node [nil] {}}
              child {node [nil] {}}
            }
            child { node [wv] {50}
              child {node [nil] {}}
              child {node [nil] {}}
            }
          }
              };
  \end{tikzpicture}
  \end{figure}

  % \begin{figure}[!h]
  %   \centering
  %   \caption{Resultado de remover \textbf{47, 10, 31, 39, 17, 12, 14}}
  %   \begin{tikzpicture}[->,>=stealth',
  %   % level/.style={sibling distance = 8cm/#1, level distance = 1.5cm},
  %  level 1/.style={sibling distance=15em, level distance = 3em},
  %   level 2/.style={sibling distance=8em, level distance = 3em},
  %   level 3/.style={sibling distance=4em, level distance = 3em},
  %   level 4/.style={sibling distance=2em, level distance = 3em},
  %   level 5/.style={sibling distance=1em, level distance = 3em} ]
  % \node [wv] (r){30}
  %     child { node [wv] {4}
  %       child { node [wv] {2}
  %         child {node [nil] {}}
  %         child {node [nil] {}}
  %       }
  %       child { node [wv] {27}
  %         child { node [wv] {8}
  %           child {node [nil] {}}
  %           child {node [nil] {}}
  %         }
  %         child {node [nil] {}}
  %       }
  %     }
  %     child { node [wv] {46}
  %       child { node [wv] {41}
  %         child {node [nil] {}}
  %         child {node [nil] {}}
  %       }
  %       child { node [wv] {50}
  %         child {node [nil] {}}
  %         child {node [nil] {}}
  %       }
  %     };
  % \end{tikzpicture}
  % \end{figure}
\end{questions}
