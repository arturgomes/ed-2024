\title{Arvores AVL}
\date{\today}
\frame{\titlepage}


\section{Arvores AVL}
\subsection{Introdução}
\begin{frame}[fragile]{\secname : \subsecname}

  \begin{itemize}
  \item A árvore AVL (nomeada por seus inventores Adelson-Velskii e Landis) deve ser vista como um BST com a seguinte propriedade adicional: 
  \begin{itemize}
  \item Para cada nó, as alturas de suas subárvores esquerda e direita diferem em no máximo 1.
  \end{itemize}
  \item Para concretizar a árvore AVL, precisamos definir uma propriedade constante para todas os nós.
\end{itemize}
\begin{equation}
\forall n: Node \bullet~|~altura(n.left) - altura(n.right)~| \leq 1
\end{equation}

  
\end{frame}
%%%%%%%%%%%%%%%%
%%% frame %%%%%%
%%%%%%%%%%%%%%%%

\begin{frame}[fragile]{\secname : \subsecname}

  \begin{itemize}
  \item Contanto que a árvore se mantenha esta propriedade, se a árvore contiver $n$ 
  nós, então ela terá uma profundidade de no máximo $O (log~n)$. 
  \end{itemize}
\end{frame}
%%%%%%%%%%%%%%%%
%%% frame %%%%%%
%%%%%%%%%%%%%%%%
\begin{frame}[fragile]{\secname : \subsecname}
  \begin{itemize}

  \item Como resultado, a busca por qualquer nó custará $O (log~n)$, e 
  se as atualizações puderem ser feitas em tempo proporcional à profundidade 
  do nó inserido ou excluído, então as atualizações também 
  custarão $O (log~n)$, mesmo no pior caso.
  \item Lembrando que $O (log~n) < O (n)$ 
  \end{itemize}
\end{frame}
%%%%%%%%%%%%%%%%
%%% frame %%%%%%
%%%%%%%%%%%%%%%%
\subsection{O que muda da BST?}
\begin{frame}[fragile]{\secname : \subsecname}

  \begin{itemize}
  \item A chave para fazer a árvore AVL funcionar é alterar as rotinas de 
  inserção e exclusão de modo a manter a propriedade de balanceamento. 
  \item Obviamente, para ser prático, devemos ser capazes de implementar as 
  rotinas de atualização revisadas em tempo $\Theta (log~n)$.
  \item Isso significa dizer, que o crescimento da complexidade em 
  relação ao tempo tem que ser exatamente $(log~n)$
  \end{itemize}
\end{frame}

%%%%%%%%%%%%%%%%
%%% frame %%%%%%
%%%%%%%%%%%%%%%%
\begin{frame}[fragile]{\secname : \subsecname}

  \begin{figure}[!h]
    \centering
    \caption{Inserir 5}
  
  \begin{tikzpicture}[->,>=stealth',
    % level/.style={sibling distance = 8cm/#1, level distance = 1.5cm},
    level 1/.style={sibling distance=15em, level distance = 3em},
    level 2/.style={sibling distance=8em, level distance = 3em},
    level 3/.style={sibling distance=5em, level distance = 3em},
    level 4/.style={sibling distance=3em, level distance = 3em},
    level 5/.style={sibling distance=2em, level distance = 3em} ]
    \node [wv] (r){37}
    child { node [wv] {24}
      child { node [gv] {7}
        child { node [gv] {2}
          child {node [nil] {}}
          child { node [rv] {5}
            child {node [nil] {}}
            child {node [nil] {}}
          }
        }
        child {node [nil] {}}
      }
      child { node [wv] {32}
        child {node [nil] {}}
        child {node [nil] {}}
      }
    }
    child { node [wv] {42}
      child { node [wv] {40}
        child {node [nil] {}}
        child {node [nil] {}}
      }
      child { node [wv] {45}
        child {node [nil] {}}
        child { node [wv] {120}
          child {node [nil] {}}
          child {node [nil] {}}
        }
      }
    }
    ;
  \end{tikzpicture}
  \end{figure}
\end{frame}
%%%%%%%%%%%%%%%% 
%%% frame %%%%%%
%%%%%%%%%%%%%%%%
\section{Desbalanceamento}
\subsection{Na AVL}
\begin{frame}[fragile]{\secname : \subsecname}
  \begin{itemize}
  \item Considere o que acontece quando inserimos um nó com valor de chave 5, 
  conforme mostrado na Figura do slide anterior. 
  \item A árvore à esquerda atende aos requisitos de balanceamento da árvore AVL. 
  \item Após a inserção, dois nós não atendem mais aos requisitos. 
  \item Como a árvore original atendeu ao requisito de balanceamento, os 
  nós na nova árvore só podem ser desbalanceados por uma \textbf{diferença de altura} de no 
  \textbf{máximo $\pm~2$} nas subárvores. 
  \end{itemize}
\end{frame}
%%%%%%%%%%%%%%%%
%%% frame %%%%%%
%%%%%%%%%%%%%%%%
\begin{frame}[fragile]{\secname : \subsecname}

  \begin{figure}[!h]
    \centering
    \caption{Inserir 5}
  
  \begin{tikzpicture}[->,>=stealth',
    % level/.style={sibling distance = 8cm/#1, level distance = 1.5cm},
    level 1/.style={sibling distance=15em, level distance = 3em},
    level 2/.style={sibling distance=8em, level distance = 3em},
    level 3/.style={sibling distance=5em, level distance = 3em},
    level 4/.style={sibling distance=3em, level distance = 3em},
    level 5/.style={sibling distance=2em, level distance = 3em}]
    \node [wv] (r){37}
    child { node [wv] {24}
      child { node [gv,label=above:-2] {7}
        child { node [gv,label=above:+1] {2}
          child {node [nil] {}}
          child { node [rv,label=above:0] {5}
            child {node [nil] {}}
            child {node [nil] {}}
          }
        }
        child {node [nil] {}}
      }
      child { node [wv] {32}
        child {node [nil] {}}
        child {node [nil] {}}
      }
    }
    child { node [wv] {42}
      child { node [wv] {40}
        child {node [nil] {}}
        child {node [nil] {}}
      }
      child { node [wv] {45}
        child {node [nil] {}}
        child { node [wv] {120}
          child {node [nil] {}}
          child {node [nil] {}}
        }
      }
    }
    ;
  \end{tikzpicture}
  \end{figure}
\end{frame}

%%%%%%%%%%%%%%%%
%%% frame %%%%%%
%%%%%%%%%%%%%%%%
\begin{frame}[fragile]{\secname : \subsecname}
  \begin{itemize}
  \item O desbalanceamento é causado pela diferença na altura dos nós
  \item Na aula passada vimos que a altura é calculada tomando como referência uma linha imaginária abaixo da árvore
  \item Para o nó mais inferior desbalanceado, chame-o de $S$, onde podemos ter 4 casos:
    \begin{itemize}
      \item[1.] O nó \textbf{filho} está no \textbf{filho esquerdo} da \emph{sub-árvore esquerda} de $S$.
      \item[2.] O nó \textbf{filho} está no \textbf{filho direito} da \emph{sub-árvore direita} de $S$.
      \item[3.] O nó \textbf{filho} está no \textbf{filho direito} da \emph{sub-árvore esquerda} de $S$. 
      \item[4.] O nó \textbf{filho} está no \textbf{filho esquerdo} da \emph{sub-árvore direita} de $S$. 
    \end{itemize}
  \end{itemize}
\end{frame}
%%%%%%%%%%%%%%%%
%%% frame %%%%%%
%%%%%%%%%%%%%%%%




\section{Rotações Simples}

%%%%%%
\subsection{À direita (RSD)}
%%%%%%
\begin{frame}[fragile]{\secname : \subsecname}
  O nó \textbf{filho} está no \textbf{filho esquerdo} da \emph{sub-árvore esquerda} de $S$.
  \begin{columns}[T] % align columns
    \begin{column}{.48\textwidth}
    \rule{\linewidth}{4pt}
    \begin{figure}[!h]
      \centering
      
    
    \begin{tikzpicture}[->,>=stealth',
      % level/.style={sibling distance = 8cm/#1, level distance = 1.5cm},
      level 1/.style={sibling distance=7em},
      level 2/.style={sibling distance=4em},
      level 3/.style={sibling distance=2em},
      level 4/.style={sibling distance=2em}]
    \node [wv] (r){8}
    child { node [wv] {6}
    child { node [rv] {4}
    child {node [nil] {A}}
    child {node [nil] {B}}
    }
    child {node [nil] {C}}
    }
    child {node [nil] {D}}
    ;
    \end{tikzpicture}
    \end{figure}
   
    \end{column}%
    \hfill%
    \begin{column}{.48\textwidth}
    \rule{\linewidth}{4pt}
          
        
    \begin{figure}[!h]
      \centering
      

    \begin{tikzpicture}[->,>=stealth',
      % level/.style={sibling distance = 8cm/#1, level distance = 1.5cm},
      level 1/.style={sibling distance=7em},
      level 2/.style={sibling distance=4em},
      level 3/.style={sibling distance=2em},
      level 4/.style={sibling distance=2em}]
    \node [wv] (r){6}
    child { node [wv] {4}
      child {node [nil] {A}}
      child {node [nil] {B}}
    }
    child { node [wv] {8}
      child {node [nil] {C}}
      child {node [nil] {D}}
    };
    \end{tikzpicture}
    \end{figure}
    \end{column}%
    \end{columns}

\end{frame}
%%%%%%%%%%%%%%%%
%%% frame %%%%%%
%%%%%%%%%%%%%%%%

\subsection{À esquerda (RSE)}
%%%%%%
\begin{frame}[fragile]{\secname : \subsecname}
  O nó \textbf{filho} está no \textbf{filho direito} da \emph{sub-árvore direita} de $S$.
  \begin{columns}[T] % align columns
    \begin{column}{.48\textwidth}
    \rule{\linewidth}{4pt}
    

      \begin{figure}[!h]
        \centering
        
      \begin{tikzpicture}[->,>=stealth',
        % level/.style={sibling distance = 8cm/#1, level distance = 1.5cm},
        level 1/.style={sibling distance=7em},
        level 2/.style={sibling distance=4em},
        level 3/.style={sibling distance=2em},
        level 4/.style={sibling distance=2em}]
      \node [wv] (r){4}
      child {node [nil] {A}}
      child { node [wv] {6}
          child {node [nil] {B}}
          child { node [rv] {8}
          child {node [nil] {C}}
          child {node [nil] {D}}
        }
      };
      \end{tikzpicture}
      \end{figure}
   
    \end{column}%
    \hfill%
    \begin{column}{.48\textwidth}
    \rule{\linewidth}{4pt}
          
        
      \begin{figure}[!h]
        \centering
        

      \begin{tikzpicture}[->,>=stealth',
        % level/.style={sibling distance = 8cm/#1, level distance = 1.5cm},
        level 1/.style={sibling distance=7em},
        level 2/.style={sibling distance=4em},
        level 3/.style={sibling distance=2em},
        level 4/.style={sibling distance=2em}]
      \node [wv] (r){6}
      child { node [wv] {4}
      child {node [nil] {A}}
      child {node [nil] {B}}
      }
      child { node [wv] {8}
      child {node [nil] {C}}
      child {node [nil] {D}}
      };
      \end{tikzpicture}
      \end{figure}
    \end{column}%
    \end{columns}

\end{frame}
%%%%%%%%%%%%%%%%
%%% frame %%%%%%
%%%%%%%%%%%%%%%%


\section{Rotações Duplas}

%%%%%%
\subsection{RSE + RSD = RED}
%%%%%%
\begin{frame}[fragile]{\secname : \subsecname}
  O nó \textbf{filho} está no \textbf{filho direito} da \emph{sub-árvore esquerda} de $S$. 
  \begin{columns}[T] % align columns
    \begin{column}{.31\textwidth}
    \rule{\linewidth}{4pt}
    
    \begin{figure}[!h]
      \centering
      
    \begin{tikzpicture}[->,>=stealth',
      % level/.style={sibling distance = 8cm/#1, level distance = 1.5cm},
      level 1/.style={sibling distance=7em},
      level 2/.style={sibling distance=4em},
      level 3/.style={sibling distance=2em},
      level 4/.style={sibling distance=2em}]
    \node [wv] (r){8}
    child { node [wv] {4}
            child {node [nil] {A}}
            child { node [rv] {6}
              child {node [nil] {B}}
              child {node [nil] {C}}
            }
    }
    child {node [nil] {D}};
    \end{tikzpicture}
    \caption{Aplicar RSE}
    \end{figure}
    \end{column}%
    \hfill%
    \begin{column}{.31\textwidth}
    \rule{\linewidth}{4pt}
    
    \begin{figure}[!h]
      \centering
      
    \begin{tikzpicture}[->,>=stealth',
      % level/.style={sibling distance = 8cm/#1, level distance = 1.5cm},
      level 1/.style={sibling distance=7em},
      level 2/.style={sibling distance=4em},
      level 3/.style={sibling distance=2em},
      level 4/.style={sibling distance=2em}]
      \node [wv] (r){8}
        child { node [wv] {6}
          child { node [wv] {4}
            child {node [nil] {A}}
            child {node [nil] {B}}
          }
          child {node [nil] {C}}
        }
        child {node [nil] {D}};
    \end{tikzpicture}
      \caption{Aplicar RSD}
    \end{figure}
    \end{column}%

    \hfill%
    \begin{column}{.31\textwidth}
    \color{black}\rule{\linewidth}{4pt}
    
    \begin{figure}[!h]
      \centering
      
      \begin{tikzpicture}[->,>=stealth',
        % level/.style={sibling distance = 8cm/#1, level distance = 1.5cm},
        level 1/.style={sibling distance=7em},
        level 2/.style={sibling distance=4em},
        level 3/.style={sibling distance=2em},
        level 4/.style={sibling distance=2em}]
      \node [wv] (r){6}
      child { node [wv] {4}
        child {node [nil] {A}}
        child {node [nil] {B}}
      }
      child { node [wv] {8}
        child {node [nil] {C}}
        child {node [nil] {D}}
      };
      \end{tikzpicture}
      \caption{Final}
    \end{figure}
    \end{column}%
    \end{columns}

\end{frame}
%%%%%%%%%%%%%%%%
%%% frame %%%%%%
%%%%%%%%%%%%%%%%

\subsection{RSD + RSE = RDE}

%%%%%%
\begin{frame}[fragile]{\secname : \subsecname}
  O nó \textbf{filho} está no \textbf{filho esquerdo} da \emph{sub-árvore direita} de $S$. 
  \begin{columns}[T] % align columns
    \begin{column}{.31\textwidth}
    \rule{\linewidth}{4pt}
    
    \begin{figure}[!h]
      \centering
      
    \begin{tikzpicture}[->,>=stealth',
      % level/.style={sibling distance = 8cm/#1, level distance = 1.5cm},
      level 1/.style={sibling distance=7em},
      level 2/.style={sibling distance=4em},
      level 3/.style={sibling distance=2em},
      level 4/.style={sibling distance=2em}]
    \node [wv] (r){4}
    child {node [nil] {A}}
    child { node [wv] {8}
      child { node [rv] {6}
        child {node [nil] {B}}
        child {node [nil] {C}}
      }
      child {node [nil] {D}}
    };
    \end{tikzpicture}
    \caption{Aplicar RSD}
    \end{figure}
    \end{column}%
    \hfill%
    \begin{column}{.31\textwidth}
    \rule{\linewidth}{4pt}
    
    \begin{figure}[!h]
      \centering
      
      \begin{tikzpicture}[->,>=stealth',
        % level/.style={sibling distance = 8cm/#1, level distance = 1.5cm},
        level 1/.style={sibling distance=7em},
        level 2/.style={sibling distance=4em},
        level 3/.style={sibling distance=2em},
        level 4/.style={sibling distance=2em}]
        \node [wv] (r){4}
        child {node [nil] {A}}
        child { node [wv] {6}
        child {node [nil] {B}}
        child { node [wv] {8}
        child {node [nil] {C}}
        child {node [nil] {D}}
        }
        };
      \end{tikzpicture}
      \caption{Aplicar RSE}
    \end{figure}
    \end{column}%
    \hfill%
    \begin{column}{.31\textwidth}
    \color{black}\rule{\linewidth}{4pt}
    
    \begin{figure}[!h]
      \centering
      

    \begin{tikzpicture}[->,>=stealth',
      % level/.style={sibling distance = 8cm/#1, level distance = 1.5cm},
      level 1/.style={sibling distance=7em},
      level 2/.style={sibling distance=4em},
      level 3/.style={sibling distance=2em},
      level 4/.style={sibling distance=2em}]
    \node [wv] (r){6}
    child { node [wv] {4}
      child {node [nil] {A}}
      child {node [nil] {B}}
    }
    child { node [wv] {8}
      child {node [nil] {C}}
      child {node [nil] {D}}
    };
    \end{tikzpicture}
    \caption{Final}
    \end{figure}
    \end{column}%
    \end{columns}

\end{frame}
%%%%%%%%%%%%%%%%
%%% frame %%%%%%
%%%%%%%%%%%%%%%%

\subsection{Exemplo 1 - Remover 11, 24, 20, 48, 39, 46, 31}


\begin{frame}[fragile]{\secname : \subsecname}
  Remover: \textbf{11}, 24, 20, 48, 39, 46, 31
  \begin{figure}[!h]
  \centering
    \caption{Remover 11 : sucessor 14}

\begin{tikzpicture}[->,>=stealth',
  % level/.style={sibling distance = 8cm/#1, level distance = 1.5cm},
 level 1/.style={sibling distance=15em, level distance = 3em},
  level 2/.style={sibling distance=8em, level distance = 3em},
  level 3/.style={sibling distance=4em, level distance = 3em},
  level 4/.style={sibling distance=2em, level distance = 3em},
  level 5/.style={sibling distance=1em, level distance = 3em} ]
\node [wv] (r){20}
child { node [rv] {11}
  child { node [wv] {5}
    child { node [wv] {3}
      child {node [nil] {}}
      child {node [nil] {}}
    }
    child { node [wv] {10}
      child {node [nil] {}}
      child {node [nil] {}}
    }
  }
  child { node [wv] {14}
    child {node [nil] {}}
    child {node [nil] {}}
  }
}
child { node [wv] {41}
child { node [wv] {31}
  child { node [wv] {24}
      child { node [wv] {21}
        child {node [nil] {}}
        child {node [nil] {}}
      }
      child {node [nil] {}}
    }
    child { node [wv] {39}
      child {node [nil] {}}
      child {node [nil] {}}
    } 
  }
  child { node [wv] {44}
    child { node [wv] {42}
      child {node [nil] {}}
      child {node [nil] {}}
    }
    child { node [wv] {46}
      child {node [nil] {}}
      child { node [wv] {48}
        child {node [nil] {}}
        child {node [nil] {}}
      }
    }
  }
};
\end{tikzpicture}

\end{figure}
\end{frame}


\begin{frame}[fragile]{\secname : \subsecname}
  Remover: \textbf{11}, 24, 20, 48, 39, 46, 31
  \begin{figure}[!h]
  \centering
    \caption{Remover 11 : sucessor 14}

\begin{tikzpicture}[->,>=stealth',
  % level/.style={sibling distance = 8cm/#1, level distance = 1.5cm},
 level 1/.style={sibling distance=15em, level distance = 3em},
  level 2/.style={sibling distance=8em, level distance = 3em},
  level 3/.style={sibling distance=4em, level distance = 3em},
  level 4/.style={sibling distance=2em, level distance = 3em},
  level 5/.style={sibling distance=1em, level distance = 3em} ]
\node [wv] (r){20}
child { node [gv,label=above:-2] {14}
  child { node [gv,label=above:0] {5}
    child { node [gv,label=above:0] {3}
      child {node [nil] {}}
      child {node [nil] {}}
    }
    child { node [wv] {10}
      child {node [nil] {}}
      child {node [nil] {}}
    }
  }
  child {node [nil] {}}
}
child { node [wv] {41}
child { node [wv] {31}
  child { node [wv] {24}
      child { node [wv] {21}
        child {node [nil] {}}
        child {node [nil] {}}
      }
      child {node [nil] {}}
    }
    child { node [wv] {39}
      child {node [nil] {}}
      child {node [nil] {}}
    } 
  }
  child { node [wv] {44}
    child { node [wv] {42}
      child {node [nil] {}}
      child {node [nil] {}}
    }
    child { node [wv] {46}
      child {node [nil] {}}
      child { node [wv] {48}
        child {node [nil] {}}
        child {node [nil] {}}
      }
    }
  }
};
\end{tikzpicture}

\end{figure}
\end{frame}

\begin{frame}[fragile]{\secname : \subsecname}
  Remover: \textbf{11}, 24, 20, 48, 39, 46, 31
  \begin{figure}[!h]
  \centering
    \caption{RSD 14 AR 5}

\begin{tikzpicture}[->,>=stealth',
  % level/.style={sibling distance = 8cm/#1, level distance = 1.5cm},
 level 1/.style={sibling distance=15em, level distance = 3em},
  level 2/.style={sibling distance=8em, level distance = 3em},
  level 3/.style={sibling distance=4em, level distance = 3em},
  level 4/.style={sibling distance=2em, level distance = 3em},
  level 5/.style={sibling distance=1em, level distance = 3em} ]
\node [wv] (r){20}
child { node [wv] {5}
    child { node [wv] {3}
      child {node [nil] {}}
      child {node [nil] {}}
    }
  child { node [wv] {14}
    child { node [wv] {10}
      child {node [nil] {}}
      child {node [nil] {}}
    }
    child {node [nil] {}}
  }
}
child { node [wv] {41}
child { node [wv] {31}
  child { node [wv] {24}
      child { node [wv] {21}
        child {node [nil] {}}
        child {node [nil] {}}
      }
      child {node [nil] {}}
    }
    child { node [wv] {39}
      child {node [nil] {}}
      child {node [nil] {}}
    } 
  }
  child { node [wv] {44}
    child { node [wv] {42}
      child {node [nil] {}}
      child {node [nil] {}}
    }
    child { node [wv] {46}
      child {node [nil] {}}
      child { node [wv] {48}
        child {node [nil] {}}
        child {node [nil] {}}
      }
    }
  }
};
\end{tikzpicture}

\end{figure}
\end{frame}


\begin{frame}[fragile]{\secname : \subsecname}
  Remover: \textbf{11, 24}, 20, 48, 39, 46, 31
  \begin{figure}[!h]
  \centering
    \caption{Remover 24 - apenas um filho}

\begin{tikzpicture}[->,>=stealth',
  % level/.style={sibling distance = 8cm/#1, level distance = 1.5cm},
 level 1/.style={sibling distance=15em, level distance = 3em},
  level 2/.style={sibling distance=8em, level distance = 3em},
  level 3/.style={sibling distance=4em, level distance = 3em},
  level 4/.style={sibling distance=2em, level distance = 3em},
  level 5/.style={sibling distance=1em, level distance = 3em} ]
\node [wv] (r){20}
child { node [wv] {5}
    child { node [wv] {3}
      child {node [nil] {}}
      child {node [nil] {}}
    }
  child { node [wv] {14}
    child { node [wv] {10}
      child {node [nil] {}}
      child {node [nil] {}}
    }
    child {node [nil] {}}
  }
}
child { node [wv] {41}
child { node [wv] {31}
  child { node [rv] {24}
      child { node [wv] {21}
        child {node [nil] {}}
        child {node [nil] {}}
      }
      child {node [nil] {}}
    }
    child { node [wv] {39}
      child {node [nil] {}}
      child {node [nil] {}}
    } 
  }
  child { node [wv] {44}
    child { node [wv] {42}
      child {node [nil] {}}
      child {node [nil] {}}
    }
    child { node [wv] {46}
      child {node [nil] {}}
      child { node [wv] {48}
        child {node [nil] {}}
        child {node [nil] {}}
      }
    }
  }
};
\end{tikzpicture}

\end{figure}
\end{frame}

\begin{frame}[fragile]{\secname : \subsecname}
  Remover: \textbf{11, 24, 20}, 48, 39, 46, 31
  \begin{figure}[!h]
  \centering
    \caption{Remover 20}

\begin{tikzpicture}[->,>=stealth',
  % level/.style={sibling distance = 8cm/#1, level distance = 1.5cm},
 level 1/.style={sibling distance=15em, level distance = 3em},
  level 2/.style={sibling distance=8em, level distance = 3em},
  level 3/.style={sibling distance=4em, level distance = 3em},
  level 4/.style={sibling distance=2em, level distance = 3em},
  level 5/.style={sibling distance=1em, level distance = 3em} ]
\node [rv] (r){20}
child { node [wv] {5}
    child { node [wv] {3}
      child {node [nil] {}}
      child {node [nil] {}}
    }
  child { node [wv] {14}
    child { node [wv] {10}
      child {node [nil] {}}
      child {node [nil] {}}
    }
    child {node [nil] {}}
  }
}
child { node [wv] {41}
child { node [wv] {31}
  child { node [wv] {21}
      child {node [nil] {}}
      child {node [nil] {}}
    }
    child { node [wv] {39}
      child {node [nil] {}}
      child {node [nil] {}}
    } 
  }
  child { node [wv] {44}
    child { node [wv] {42}
      child {node [nil] {}}
      child {node [nil] {}}
    }
    child { node [wv] {46}
      child {node [nil] {}}
      child { node [wv] {48}
        child {node [nil] {}}
        child {node [nil] {}}
      }
    }
  }
};
\end{tikzpicture}

\end{figure}
\end{frame}

\begin{frame}[fragile]{\secname : \subsecname}
  Remover: \textbf{11, 24, 20}, 48, 39, 46, 31
  \begin{figure}[!h]
  \centering
    \caption{Remover 20: sucessor 21}

\begin{tikzpicture}[->,>=stealth',
  % level/.style={sibling distance = 8cm/#1, level distance = 1.5cm},
 level 1/.style={sibling distance=15em, level distance = 3em},
  level 2/.style={sibling distance=8em, level distance = 3em},
  level 3/.style={sibling distance=4em, level distance = 3em},
  level 4/.style={sibling distance=2em, level distance = 3em},
  level 5/.style={sibling distance=1em, level distance = 3em} ]
\node [wv] (r){21}
child { node [wv] {5}
    child { node [wv] {3}
      child {node [nil] {}}
      child {node [nil] {}}
    }
  child { node [wv] {14}
    child { node [wv] {10}
      child {node [nil] {}}
      child {node [nil] {}}
    }
    child {node [nil] {}}
  }
}
child { node [wv] {41}
child { node [wv] {31}
  child { node [wv] {21}
      child {node [nil] {}}
      child {node [nil] {}}
    }
    child { node [wv] {39}
      child {node [nil] {}}
      child {node [nil] {}}
    } 
  }
  child { node [wv] {44}
    child { node [wv] {42}
      child {node [nil] {}}
      child {node [nil] {}}
    }
    child { node [wv] {46}
      child {node [nil] {}}
      child { node [wv] {48}
        child {node [nil] {}}
        child {node [nil] {}}
      }
    }
  }
};
\end{tikzpicture}

\end{figure}
\end{frame}


\begin{frame}[fragile]{\secname : \subsecname}
  Remover: \textbf{11, 24, 20}, 48, 39, 46, 31
  \begin{figure}[!h]
  \centering
    \caption{Remover 20: remove sucessor 21 duplicado}

\begin{tikzpicture}[->,>=stealth',
  % level/.style={sibling distance = 8cm/#1, level distance = 1.5cm},
 level 1/.style={sibling distance=15em, level distance = 3em},
  level 2/.style={sibling distance=8em, level distance = 3em},
  level 3/.style={sibling distance=4em, level distance = 3em},
  level 4/.style={sibling distance=2em, level distance = 3em},
  level 5/.style={sibling distance=1em, level distance = 3em} ]
\node [wv] (r){21}
child { node [wv] {5}
    child { node [wv] {3}
      child {node [nil] {}}
      child {node [nil] {}}
    }
  child { node [wv] {14}
    child { node [wv] {10}
      child {node [nil] {}}
      child {node [nil] {}}
    }
    child {node [nil] {}}
  }
}
child { node [wv] {41}
child { node [wv] {31}
    child {node [nil] {}}
    child { node [wv] {39}
      child {node [nil] {}}
      child {node [nil] {}}
    } 
  }
  child { node [wv] {44}
    child { node [wv] {42}
      child {node [nil] {}}
      child {node [nil] {}}
    }
    child { node [wv] {46}
      child {node [nil] {}}
      child { node [wv] {48}
        child {node [nil] {}}
        child {node [nil] {}}
      }
    }
  }
};
\end{tikzpicture}

\end{figure}
\end{frame}


\begin{frame}[fragile]{\secname : \subsecname}
  Remover: \textbf{11, 24, 20}, 48, 39, 46, 31
  \begin{figure}[!h]
  \centering
    \caption{Remover48}

\begin{tikzpicture}[->,>=stealth',
  % level/.style={sibling distance = 8cm/#1, level distance = 1.5cm},
 level 1/.style={sibling distance=15em, level distance = 3em},
  level 2/.style={sibling distance=8em, level distance = 3em},
  level 3/.style={sibling distance=4em, level distance = 3em},
  level 4/.style={sibling distance=2em, level distance = 3em},
  level 5/.style={sibling distance=1em, level distance = 3em} ]
\node [wv] (r){21}
child { node [wv] {5}
    child { node [wv] {3}
      child {node [nil] {}}
      child {node [nil] {}}
    }
  child { node [wv] {14}
    child { node [wv] {10}
      child {node [nil] {}}
      child {node [nil] {}}
    }
    child {node [nil] {}}
  }
}
child { node [wv] {41}
child { node [wv] {31}
    child {node [nil] {}}
    child { node [wv] {39}
      child {node [nil] {}}
      child {node [nil] {}}
    } 
  }
  child { node [wv] {44}
    child { node [wv] {42}
      child {node [nil] {}}
      child {node [nil] {}}
    }
    child { node [wv] {46}
      child {node [nil] {}}
      child { node [rv] {48}
        child {node [nil] {}}
        child {node [nil] {}}
      }
    }
  }
};
\end{tikzpicture}

\end{figure}
\end{frame}

\begin{frame}[fragile]{\secname : \subsecname}
  Remover: \textbf{11, 24, 20, 48}, 39, 46, 31
  \begin{figure}[!h]
  \centering
    \caption{Remover 48}

\begin{tikzpicture}[->,>=stealth',
  % level/.style={sibling distance = 8cm/#1, level distance = 1.5cm},
 level 1/.style={sibling distance=15em, level distance = 3em},
  level 2/.style={sibling distance=8em, level distance = 3em},
  level 3/.style={sibling distance=4em, level distance = 3em},
  level 4/.style={sibling distance=2em, level distance = 3em},
  level 5/.style={sibling distance=1em, level distance = 3em} ]
\node [wv] (r){21}
child { node [wv] {5}
    child { node [wv] {3}
      child {node [nil] {}}
      child {node [nil] {}}
    }
  child { node [wv] {14}
    child { node [wv] {10}
      child {node [nil] {}}
      child {node [nil] {}}
    }
    child {node [nil] {}}
  }
}
child { node [wv] {41}
child { node [wv] {31}
    child {node [nil] {}}
    child { node [wv] {39}
      child {node [nil] {}}
      child {node [nil] {}}
    } 
  }
  child { node [wv] {44}
    child { node [wv] {42}
      child {node [nil] {}}
      child {node [nil] {}}
    }
    child { node [wv] {46}
      child {node [nil] {}}
      child {node [nil] {}}
    }
  }
};
\end{tikzpicture}

\end{figure}
\end{frame}


\begin{frame}[fragile]{\secname : \subsecname}
  Remover: \textbf{11, 24, 20, 48, 39}, 46, 31
  \begin{figure}[!h]
  \centering
    \caption{Remover 39}

\begin{tikzpicture}[->,>=stealth',
  % level/.style={sibling distance = 8cm/#1, level distance = 1.5cm},
 level 1/.style={sibling distance=15em, level distance = 3em},
  level 2/.style={sibling distance=8em, level distance = 3em},
  level 3/.style={sibling distance=4em, level distance = 3em},
  level 4/.style={sibling distance=2em, level distance = 3em},
  level 5/.style={sibling distance=1em, level distance = 3em} ]
\node [wv] (r){21}
child { node [wv] {5}
    child { node [wv] {3}
      child {node [nil] {}}
      child {node [nil] {}}
    }
  child { node [wv] {14}
    child { node [wv] {10}
      child {node [nil] {}}
      child {node [nil] {}}
    }
    child {node [nil] {}}
  }
}
child { node [wv] {41}
child { node [wv] {31}
    child {node [nil] {}}
    child { node [rv] {39}
      child {node [nil] {}}
      child {node [nil] {}}
    } 
  }
  child { node [wv] {44}
    child { node [wv] {42}
      child {node [nil] {}}
      child {node [nil] {}}
    }
    child { node [wv] {46}
      child {node [nil] {}}
      child {node [nil] {}}
    }
  }
};
\end{tikzpicture}

\end{figure}
\end{frame}



\begin{frame}[fragile]{\secname : \subsecname}
  Remover: \textbf{11, 24, 20, 48, 39}, 46, 31
  \begin{figure}[!h]
  \centering
    \caption{Remover 39}

\begin{tikzpicture}[->,>=stealth',
  % level/.style={sibling distance = 8cm/#1, level distance = 1.5cm},
 level 1/.style={sibling distance=15em, level distance = 3em},
  level 2/.style={sibling distance=8em, level distance = 3em},
  level 3/.style={sibling distance=4em, level distance = 3em},
  level 4/.style={sibling distance=2em, level distance = 3em},
  level 5/.style={sibling distance=1em, level distance = 3em} ]
\node [wv] (r){21}
child { node [wv] {5}
    child { node [wv] {3}
      child {node [nil] {}}
      child {node [nil] {}}
    }
  child { node [wv] {14}
    child { node [wv] {10}
      child {node [nil] {}}
      child {node [nil] {}}
    }
    child {node [nil] {}}
  }
}
child { node [wv] {41}
child { node [wv] {31}
    child {node [nil] {}}
    child {node [nil] {}}
  }
  child { node [wv] {44}
    child { node [wv] {42}
      child {node [nil] {}}
      child {node [nil] {}}
    }
    child { node [wv] {46}
      child {node [nil] {}}
      child {node [nil] {}}
    }
  }
};
\end{tikzpicture}

\end{figure}
\end{frame}




\begin{frame}[fragile]{\secname : \subsecname}
  Remover: \textbf{11, 24, 20, 48, 39, 46}, 31
  \begin{figure}[!h]
  \centering
    \caption{Remover 46}

\begin{tikzpicture}[->,>=stealth',
  % level/.style={sibling distance = 8cm/#1, level distance = 1.5cm},
 level 1/.style={sibling distance=15em, level distance = 3em},
  level 2/.style={sibling distance=8em, level distance = 3em},
  level 3/.style={sibling distance=4em, level distance = 3em},
  level 4/.style={sibling distance=2em, level distance = 3em},
  level 5/.style={sibling distance=1em, level distance = 3em} ]
\node [wv] (r){21}
child { node [wv] {5}
    child { node [wv] {3}
      child {node [nil] {}}
      child {node [nil] {}}
    }
  child { node [wv] {14}
    child { node [wv] {10}
      child {node [nil] {}}
      child {node [nil] {}}
    }
    child {node [nil] {}}
  }
}
child { node [wv] {41}
child { node [wv] {31}
    child {node [nil] {}}
    child {node [nil] {}}
  }
  child { node [wv] {44}
    child { node [wv] {42}
      child {node [nil] {}}
      child {node [nil] {}}
    }
    child { node [wv] {46}
      child {node [nil] {}}
      child {node [nil] {}}
    }
  }
};
\end{tikzpicture}

\end{figure}
\end{frame}




\begin{frame}[fragile]{\secname : \subsecname}
  Remover: \textbf{11, 24, 20, 48, 39, 46}, 31
  \begin{figure}[!h]
  \centering
    \caption{Remover 46}

\begin{tikzpicture}[->,>=stealth',
  % level/.style={sibling distance = 8cm/#1, level distance = 1.5cm},
 level 1/.style={sibling distance=15em, level distance = 3em},
  level 2/.style={sibling distance=8em, level distance = 3em},
  level 3/.style={sibling distance=4em, level distance = 3em},
  level 4/.style={sibling distance=2em, level distance = 3em},
  level 5/.style={sibling distance=1em, level distance = 3em} ]
\node [wv] (r){21}
child { node [wv] {5}
    child { node [wv] {3}
      child {node [nil] {}}
      child {node [nil] {}}
    }
  child { node [wv] {14}
    child { node [wv] {10}
      child {node [nil] {}}
      child {node [nil] {}}
    }
    child {node [nil] {}}
  }
}
child { node [wv] {41}
child { node [wv] {31}
    child {node [nil] {}}
    child {node [nil] {}}
  }
  child { node [wv] {44}
    child { node [wv] {42}
      child {node [nil] {}}
      child {node [nil] {}}
    }
    child {node [nil] {}}
  }
};
\end{tikzpicture}

\end{figure}
\end{frame}



\begin{frame}[fragile]{\secname : \subsecname}
  Remover: \textbf{11, 24, 20, 48, 39, 46, 31}
  \begin{figure}[!h]
  \centering
    \caption{Remover 31}

\begin{tikzpicture}[->,>=stealth',
  % level/.style={sibling distance = 8cm/#1, level distance = 1.5cm},
 level 1/.style={sibling distance=15em, level distance = 3em},
  level 2/.style={sibling distance=8em, level distance = 3em},
  level 3/.style={sibling distance=4em, level distance = 3em},
  level 4/.style={sibling distance=2em, level distance = 3em},
  level 5/.style={sibling distance=1em, level distance = 3em} ]
\node [wv] (r){21}
child { node [wv] {5}
    child { node [wv] {3}
      child {node [nil] {}}
      child {node [nil] {}}
    }
  child { node [wv] {14}
    child { node [wv] {10}
      child {node [nil] {}}
      child {node [nil] {}}
    }
    child {node [nil] {}}
  }
}
child { node [wv] {41}
  child { node [rv] {31}
    child {node [nil] {}}
    child {node [nil] {}}
  }
  child { node [wv] {44}
    child { node [wv] {42}
      child {node [nil] {}}
      child {node [nil] {}}
    }
    child {node [nil] {}}
  }
};
\end{tikzpicture}

\end{figure}
\end{frame}



\begin{frame}[fragile]{\secname : \subsecname}
  Remover: \textbf{11, 24, 20, 48, 39, 46, 31}
  \begin{figure}[!h]
  \centering
    \caption{Remover 31}

\begin{tikzpicture}[->,>=stealth',
  % level/.style={sibling distance = 8cm/#1, level distance = 1.5cm},
 level 1/.style={sibling distance=15em, level distance = 3em},
  level 2/.style={sibling distance=8em, level distance = 3em},
  level 3/.style={sibling distance=4em, level distance = 3em},
  level 4/.style={sibling distance=2em, level distance = 3em},
  level 5/.style={sibling distance=1em, level distance = 3em} ]
\node [wv] (r){21}
child { node [wv] {5}
    child { node [wv] {3}
      child {node [nil] {}}
      child {node [nil] {}}
    }
  child { node [wv] {14}
    child { node [wv] {10}
      child {node [nil] {}}
      child {node [nil] {}}
    }
    child {node [nil] {}}
  }
}
child { node [gv,label=above:+2] {41}
  child {node [nil] {}}
  child { node [gv,label=above:-1] {44}
    child { node [gv] {42}
      child {node [nil] {}}
      child {node [nil] {}}
    }
    child {node [nil] {}}
  }
};
\end{tikzpicture}

\end{figure}
\end{frame}



\begin{frame}[fragile]{\secname : \subsecname}
  Remover: \textbf{11, 24, 20, 48, 39, 46, 31}
  \begin{figure}[!h]
  \centering
    \caption{RSD 44 AR 42 + RSE 41 AR 42}

\begin{tikzpicture}[->,>=stealth',
  % level/.style={sibling distance = 8cm/#1, level distance = 1.5cm},
 level 1/.style={sibling distance=15em, level distance = 3em},
  level 2/.style={sibling distance=8em, level distance = 3em},
  level 3/.style={sibling distance=4em, level distance = 3em},
  level 4/.style={sibling distance=2em, level distance = 3em},
  level 5/.style={sibling distance=1em, level distance = 3em} ]
\node [wv] (r){21}
child { node [wv] {5}
    child { node [wv] {3}
      child {node [nil] {}}
      child {node [nil] {}}
    }
  child { node [wv] {14}
    child { node [wv] {10}
      child {node [nil] {}}
      child {node [nil] {}}
    }
    child {node [nil] {}}
  }
}
child { node [wv] {42}
    child { node [wv] {41}
      child {node [nil] {}}
      child {node [nil] {}}
    }
  child { node [wv] {44}
    child {node [nil] {}}
    child {node [nil] {}}
  }
};
\end{tikzpicture}

\end{figure}
\end{frame}

\subsection{Exemplo 2}
\begin{frame}[fragile] {\secname : \subsecname}
  Inserir: \textbf{4}, 39, 30, 27, 41, 46, 47, 10, 31, 8, 17, 12, 50, 2, 14
\begin{figure}[!h]
  \centering
  \caption{ Inserir 4 }

\begin{tikzpicture}[->,>=stealth',
  % level/.style={sibling distance = 8cm/#1, level distance = 1.5cm},
 level 1/.style={sibling distance=15em, level distance = 3em},
  level 2/.style={sibling distance=8em, level distance = 3em},
  level 3/.style={sibling distance=4em, level distance = 3em},
  level 4/.style={sibling distance=2em, level distance = 3em},
  level 5/.style={sibling distance=1em, level distance = 3em} ]
\node [wv] (r){4}
child {node [nil] {}}
child {node [nil] {}};
\end{tikzpicture}
\end{figure}
\end{frame}

%%%


\begin{frame}[fragile] {\secname : \subsecname}
  Inserir: \textbf{4, 39}, 30, 27, 41, 46, 47, 10, 31, 8, 17, 12, 50, 2, 14
\begin{figure}[!h]
  \centering
  \caption{ Inserir 39}

\begin{tikzpicture}[->,>=stealth',
  % level/.style={sibling distance = 8cm/#1, level distance = 1.5cm},
 level 1/.style={sibling distance=15em, level distance = 3em},
  level 2/.style={sibling distance=8em, level distance = 3em},
  level 3/.style={sibling distance=4em, level distance = 3em},
  level 4/.style={sibling distance=2em, level distance = 3em},
  level 5/.style={sibling distance=1em, level distance = 3em} ]
\node [wv] (r){4}

child {node [nil] {}}
child { node [wv] {39}
              child {node [nil] {}}
              child {node [nil] {}}
            };
\end{tikzpicture}

\end{figure}
\end{frame}

%%%


\begin{frame}[fragile] {\secname : \subsecname}
  Inserir: \textbf{4, 39, 30}, 27, 41, 46, 47, 10, 31, 8, 17, 12, 50, 2, 14
\begin{figure}[!h]
  \centering
  \caption{Inserir 30}

\begin{tikzpicture}[->,>=stealth',
  % level/.style={sibling distance = 8cm/#1, level distance = 1.5cm},
 level 1/.style={sibling distance=15em, level distance = 3em},
  level 2/.style={sibling distance=8em, level distance = 3em},
  level 3/.style={sibling distance=4em, level distance = 3em},
  level 4/.style={sibling distance=2em, level distance = 3em},
  level 5/.style={sibling distance=1em, level distance = 3em} ]
\node [gv,label=above:+2] (r){4}
child {node [nil] {}}
child {node [gv,label=above:-1] {39}
              child { node [gv,label=above:0] {30}
                child {node [nil] {}}
                child {node [nil] {}}
              }
              child {node [nil] {}}
            };
\end{tikzpicture}
\end{figure}
\end{frame}

%%%


\begin{frame}[fragile] {\secname : \subsecname}
  Inserir: \textbf{4, 39, 30}, 27, 41, 46, 47, 10, 31, 8, 17, 12, 50, 2, 14
\begin{figure}[!h]
  \centering
  \caption{ RSD 39 AR 30 }

\begin{tikzpicture}[->,>=stealth',
  % level/.style={sibling distance = 8cm/#1, level distance = 1.5cm},
 level 1/.style={sibling distance=15em, level distance = 3em},
  level 2/.style={sibling distance=8em, level distance = 3em},
  level 3/.style={sibling distance=4em, level distance = 3em},
  level 4/.style={sibling distance=2em, level distance = 3em},
  level 5/.style={sibling distance=1em, level distance = 3em} ]
\node [gv,label=above:+2] (r){4}
child {node [nil] {}}
child {node [gv,label=above:+1] {30}
              child {node [nil] {}}
              child { node [gv,label=above:0] {39}
                child {node [nil] {}}
                child {node [nil] {}}
              }
            };
\end{tikzpicture}
\end{figure}
\end{frame}

%%%


\begin{frame}[fragile] {\secname : \subsecname}
  Inserir: \textbf{4, 39, 30}, 27, 41, 46, 47, 10, 31, 8, 17, 12, 50, 2, 14
\begin{figure}[!h]
  \centering
  \caption{RSE 4 AR 30}

\begin{tikzpicture}[->,>=stealth',
  % level/.style={sibling distance = 8cm/#1, level distance = 1.5cm},
 level 1/.style={sibling distance=15em, level distance = 3em},
  level 2/.style={sibling distance=8em, level distance = 3em},
  level 3/.style={sibling distance=4em, level distance = 3em},
  level 4/.style={sibling distance=2em, level distance = 3em},
  level 5/.style={sibling distance=1em, level distance = 3em} ]
\node [wv] (r){30}
child { node [wv] {4}
                child {node [nil] {}}
                child {node [nil] {}}
              }
child { node [wv] {39}
              child {node [nil] {}}
              child {node [nil] {}}
            };
\end{tikzpicture}

\end{figure}
\end{frame}

%%%


\begin{frame}[fragile] {\secname : \subsecname}
  Inserir: \textbf{4, 39, 30, 27}, 41, 46, 47, 10, 31, 8, 17, 12, 50, 2, 14
\begin{figure}[!h]
  \centering
  \caption{Inserir 27}

\begin{tikzpicture}[->,>=stealth',
  % level/.style={sibling distance = 8cm/#1, level distance = 1.5cm},
 level 1/.style={sibling distance=15em, level distance = 3em},
  level 2/.style={sibling distance=8em, level distance = 3em},
  level 3/.style={sibling distance=4em, level distance = 3em},
  level 4/.style={sibling distance=2em, level distance = 3em},
  level 5/.style={sibling distance=1em, level distance = 3em} ]
\node [wv] (r){30}
child { node [wv] {4}
                child {node [nil] {}}
                child { node [wv] {27}
                  child {node [nil] {}}
                  child {node [nil] {}}
                }
              }
child { node [wv] {39}
              child {node [nil] {}}
              child {node [nil] {}}
            };
\end{tikzpicture}

\end{figure}
\end{frame}

%%%


\begin{frame}[fragile] {\secname : \subsecname}
  Inserir: \textbf{4, 39, 30, 27, 41}, 46, 47, 10, 31, 8, 17, 12, 50, 2, 14
\begin{figure}[!h]
  \centering
  \caption{Inserir 41}

\begin{tikzpicture}[->,>=stealth',
  % level/.style={sibling distance = 8cm/#1, level distance = 1.5cm},
 level 1/.style={sibling distance=15em, level distance = 3em},
  level 2/.style={sibling distance=8em, level distance = 3em},
  level 3/.style={sibling distance=4em, level distance = 3em},
  level 4/.style={sibling distance=2em, level distance = 3em},
  level 5/.style={sibling distance=1em, level distance = 3em} ]
\node [wv] (r){30}
child { node [wv] {4}
                child {node [nil] {}}
                child { node [wv] {27}
                  child {node [nil] {}}
                  child {node [nil] {}}
                }
              }
child { node [wv] {39}
              child {node [nil] {}}
              child { node [wv] {41}
              child {node [nil] {}}
              child {node [nil] {}}
            }
            };
\end{tikzpicture}

\end{figure}
\end{frame}

%%%


\begin{frame}[fragile] {\secname : \subsecname}
  Inserir: \textbf{4, 39, 30, 27, 41, 46}, 47, 10, 31, 8, 17, 12, 50, 2, 14
\begin{figure}[!h]
  \centering
  \caption{Inserir 46}

\begin{tikzpicture}[->,>=stealth',
  % level/.style={sibling distance = 8cm/#1, level distance = 1.5cm},
 level 1/.style={sibling distance=15em, level distance = 3em},
  level 2/.style={sibling distance=8em, level distance = 3em},
  level 3/.style={sibling distance=4em, level distance = 3em},
  level 4/.style={sibling distance=2em, level distance = 3em},
  level 5/.style={sibling distance=1em, level distance = 3em} ]
\node [wv] (r){30}
child { node [wv] {4}
                child {node [nil] {}}
                child { node [wv] {27}
                  child {node [nil] {}}
                  child {node [nil] {}}
                }
              }
child { node [gv,label=above:+2] {39}
              child {node [nil] {}}
              child { node [gv,label=above:+1] {41}
                child {node [nil] {}}
                child { node [gv,label=above:0] {46}
                  child {node [nil] {}}
                  child {node [nil] {}}
                }
              }
            };
\end{tikzpicture}

\end{figure}
\end{frame}

%%%


\begin{frame}[fragile] {\secname : \subsecname}
  Inserir: \textbf{4, 39, 30, 27, 41, 46}, 47, 10, 31, 8, 17, 12, 50, 2, 14
\begin{figure}[!h]
  \centering
  \caption{RSE 39 AR 41}

\begin{tikzpicture}[->,>=stealth',
  % level/.style={sibling distance = 8cm/#1, level distance = 1.5cm},
 level 1/.style={sibling distance=15em, level distance = 3em},
  level 2/.style={sibling distance=8em, level distance = 3em},
  level 3/.style={sibling distance=4em, level distance = 3em},
  level 4/.style={sibling distance=2em, level distance = 3em},
  level 5/.style={sibling distance=1em, level distance = 3em} ]
\node [wv] (r){30}
child { node [wv] {4}
                child {node [nil] {}}
                child { node [wv] {27}
                  child {node [nil] {}}
                  child {node [nil] {}}
                }
              }
child { node [wv] {41}
        child { node [wv] {39}
        child {node [nil] {}}
        child {node [nil] {}}
        }
        child { node [wv] {46}
        child {node [nil] {}}
        child {node [nil] {}}
        }
            };
\end{tikzpicture}

\end{figure}
\end{frame}

%%%


\begin{frame}[fragile] {\secname : \subsecname}
  Inserir: \textbf{4, 39, 30, 27, 41, 46, 47}, 10, 31, 8, 17, 12, 50, 2, 14
\begin{figure}[!h]
  \centering
  \caption{Inserir 47}

\begin{tikzpicture}[->,>=stealth',
  % level/.style={sibling distance = 8cm/#1, level distance = 1.5cm},
 level 1/.style={sibling distance=15em, level distance = 3em},
  level 2/.style={sibling distance=8em, level distance = 3em},
  level 3/.style={sibling distance=4em, level distance = 3em},
  level 4/.style={sibling distance=2em, level distance = 3em},
  level 5/.style={sibling distance=1em, level distance = 3em} ]
\node [wv] (r){30}
child { node [wv] {4}
                child {node [nil] {}}
                child { node [wv] {27}
                  child {node [nil] {}}
                  child {node [nil] {}}
                }
              }
child { node [wv] {41}
        child { node [wv] {39}
        child {node [nil] {}}
        child {node [nil] {}}
        }
        child { node [wv] {46}
        child {node [nil] {}}
          child { node [wv] {47}
            child {node [nil] {}}
            child {node [nil] {}}
          }
        }
            };
\end{tikzpicture}

\end{figure}
\end{frame}

%%%


\begin{frame}[fragile] {\secname : \subsecname}
  Inserir: \textbf{4, 39, 30, 27, 41, 46, 47, 10}, 31, 8, 17, 12, 50, 2, 14
\begin{figure}[!h]
  \centering
  \caption{Inserir 10}

\begin{tikzpicture}[->,>=stealth',
  % level/.style={sibling distance = 8cm/#1, level distance = 1.5cm},
 level 1/.style={sibling distance=15em, level distance = 3em},
  level 2/.style={sibling distance=8em, level distance = 3em},
  level 3/.style={sibling distance=4em, level distance = 3em},
  level 4/.style={sibling distance=2em, level distance = 3em},
  level 5/.style={sibling distance=1em, level distance = 3em} ]
\node [wv] (r){30}
child { node [gv,label=above:+2] {4}
                child {node [nil] {}}
                child { node [gv,label=above:-1] {27}
                  child { node [gv,label=above:0] {10}
                    child {node [nil] {}}
                    child {node [nil] {}}
                  }
                  child {node [nil] {}}
                }
              }
child { node [wv] {41}
        child { node [wv] {39}
        child {node [nil] {}}
        child {node [nil] {}}
        }
        child { node [wv] {46}
        child {node [nil] {}}
          child { node [wv] {47}
            child {node [nil] {}}
            child {node [nil] {}}
          }
        }
            };
\end{tikzpicture}

\end{figure}
\end{frame}

%%%


\begin{frame}[fragile] {\secname : \subsecname}
  Inserir: \textbf{4, 39, 30, 27, 41, 46, 47, 10}, 31, 8, 17, 12, 50, 2, 14
\begin{figure}[!h]
  \centering
  \caption{RSD 27 AR 10 + RSE 4 AR 10}

\begin{tikzpicture}[->,>=stealth',
  % level/.style={sibling distance = 8cm/#1, level distance = 1.5cm},
 level 1/.style={sibling distance=15em, level distance = 3em},
  level 2/.style={sibling distance=8em, level distance = 3em},
  level 3/.style={sibling distance=4em, level distance = 3em},
  level 4/.style={sibling distance=2em, level distance = 3em},
  level 5/.style={sibling distance=1em, level distance = 3em} ]
\node [wv] (r){30}
child { node [wv] {10}
        child { node [wv] {4}
          child {node [nil] {}}
          child {node [nil] {}}
        }
        child { node [wv] {27}
                child {node [nil] {}}
                child {node [nil] {}}
        }
      }
child { node [wv] {41}
        child { node [wv] {39}
          child {node [nil] {}}
         child {node [nil] {}}
        }
        child { node [wv] {46}
          child {node [nil] {}}
          child { node [wv] {47}
            child {node [nil] {}}
            child {node [nil] {}}
          }
        }
            };
\end{tikzpicture}

\end{figure}
\end{frame}

%%%


\begin{frame}[fragile] {\secname : \subsecname}
  Inserir: \textbf{4, 39, 30, 27, 41, 46, 47, 10, 31}, 8, 17, 12, 50, 2, 14
\begin{figure}[!h]
  \centering
  \caption{Inserir 31}

\begin{tikzpicture}[->,>=stealth',
  % level/.style={sibling distance = 8cm/#1, level distance = 1.5cm},
 level 1/.style={sibling distance=15em, level distance = 3em},
  level 2/.style={sibling distance=8em, level distance = 3em},
  level 3/.style={sibling distance=4em, level distance = 3em},
  level 4/.style={sibling distance=2em, level distance = 3em},
  level 5/.style={sibling distance=1em, level distance = 3em} ]
\node [wv] (r){30}
child { node [wv] {10}
        child { node [wv] {4}
          child {node [nil] {}}
          child {node [nil] {}}
        }
        child { node [wv] {27}
                child {node [nil] {}}
                child {node [nil] {}}
        }
      }
child { node [wv] {41}
        child { node [wv] {39}
          child { node [wv] {31}
            child {node [nil] {}}
            child {node [nil] {}}
          }
          child {node [nil] {}}
        }
        child { node [wv] {46}
          child {node [nil] {}}
          child { node [wv] {47}
            child {node [nil] {}}
            child {node [nil] {}}
          }
        }
            };
\end{tikzpicture}

\end{figure}
\end{frame}

%%%


\begin{frame}[fragile] {\secname : \subsecname}
  Inserir: \textbf{4, 39, 30, 27, 41, 46, 47, 10, 31, 8}, 17, 12, 50, 2, 14
\begin{figure}[!h]
  \centering
  \caption{Inserir 8}

\begin{tikzpicture}[->,>=stealth',
  % level/.style={sibling distance = 8cm/#1, level distance = 1.5cm},
 level 1/.style={sibling distance=15em, level distance = 3em},
  level 2/.style={sibling distance=8em, level distance = 3em},
  level 3/.style={sibling distance=4em, level distance = 3em},
  level 4/.style={sibling distance=2em, level distance = 3em},
  level 5/.style={sibling distance=1em, level distance = 3em} ]
\node [wv] (r){30}
child { node [wv] {10}
        child { node [wv] {4}
          child {node [nil] {}}
          child { node [wv] {8}
            child {node [nil] {}}
            child {node [nil] {}}
          }
        }
        child { node [wv] {27}
                child {node [nil] {}}
                child {node [nil] {}}
        }
      }
child { node [wv] {41}
        child { node [wv] {39}
          child { node [wv] {31}
            child {node [nil] {}}
            child {node [nil] {}}
          }
          child {node [nil] {}}
        }
        child { node [wv] {46}
          child {node [nil] {}}
          child { node [wv] {47}
            child {node [nil] {}}
            child {node [nil] {}}
          }
        }
            };
\end{tikzpicture}

\end{figure}
\end{frame}

%%%


\begin{frame}[fragile] {\secname : \subsecname}
  Inserir: \textbf{4, 39, 30, 27, 41, 46, 47, 10, 31, 8, 17}, 12, 50, 2, 14
\begin{figure}[!h]
  \centering
  \caption{Inserir 17}

\begin{tikzpicture}[->,>=stealth',
  % level/.style={sibling distance = 8cm/#1, level distance = 1.5cm},
 level 1/.style={sibling distance=15em, level distance = 3em},
  level 2/.style={sibling distance=8em, level distance = 3em},
  level 3/.style={sibling distance=4em, level distance = 3em},
  level 4/.style={sibling distance=2em, level distance = 3em},
  level 5/.style={sibling distance=1em, level distance = 3em} ]
\node [wv] (r){30}
child { node [wv] {10}
        child { node [wv] {4}
          child {node [nil] {}}
          child { node [wv] {8}
            child {node [nil] {}}
            child {node [nil] {}}
          }
        }
        child { node [wv] {27}
          child { node [wv] {17}
            child {node [nil] {}}
            child {node [nil] {}}
           }
                child {node [nil] {}}
        }
      }
child { node [wv] {41}
        child { node [wv] {39}
          child { node [wv] {31}
            child {node [nil] {}}
            child {node [nil] {}}
          }
          child {node [nil] {}}
        }
        child { node [wv] {46}
          child {node [nil] {}}
          child { node [wv] {47}
            child {node [nil] {}}
            child {node [nil] {}}
          }
        }
            };
\end{tikzpicture}

\end{figure}
\end{frame}

%%%


\begin{frame}[fragile] {\secname : \subsecname}
  Inserir: \textbf{4, 39, 30, 27, 41, 46, 47, 10, 31, 8, 17, 12}, 50, 2, 14
\begin{figure}[!h]
  \centering
  \caption{Inserir 12}

\begin{tikzpicture}[->,>=stealth',
  % level/.style={sibling distance = 8cm/#1, level distance = 1.5cm},
 level 1/.style={sibling distance=15em, level distance = 3em},
  level 2/.style={sibling distance=8em, level distance = 3em},
  level 3/.style={sibling distance=4em, level distance = 3em},
  level 4/.style={sibling distance=2em, level distance = 3em},
  level 5/.style={sibling distance=1em, level distance = 3em} ]
\node [wv] (r){30}
child { node [wv] {10}
        child { node [wv] {4}
          child {node [nil] {}}
          child { node [wv] {8}
            child {node [nil] {}}
            child {node [nil] {}}
          }
        }
        child { node [gv,label=above:-2] {27}
          child { node [gv,label=above:-1] {17}
            child { node [gv,label=above:0] {12}
              child {node [nil] {}}
              child {node [nil] {}}
            }
            child {node [nil] {}}
           }
                child {node [nil] {}}
        }
      }
child { node [wv] {41}
        child { node [wv] {39}
          child { node [wv] {31}
            child {node [nil] {}}
            child {node [nil] {}}
          }
          child {node [nil] {}}
        }
        child { node [wv] {46}
          child {node [nil] {}}
          child { node [wv] {47}
            child {node [nil] {}}
            child {node [nil] {}}
          }
        }
            };
\end{tikzpicture}

\end{figure}
\end{frame}

%%%


\begin{frame}[fragile] {\secname : \subsecname}
  Inserir: \textbf{4, 39, 30, 27, 41, 46, 47, 10, 31, 8, 17, 12}, 50, 2, 14
\begin{figure}[!h]
  \centering
  \caption{RSD 27 AR 17}

\begin{tikzpicture}[->,>=stealth',
  % level/.style={sibling distance = 8cm/#1, level distance = 1.5cm},
 level 1/.style={sibling distance=15em, level distance = 3em},
  level 2/.style={sibling distance=8em, level distance = 3em},
  level 3/.style={sibling distance=4em, level distance = 3em},
  level 4/.style={sibling distance=2em, level distance = 3em},
  level 5/.style={sibling distance=1em, level distance = 3em} ]
\node [wv] (r){30}
child { node [wv] {10}
        child { node [wv] {4}
          child {node [nil] {}}
          child { node [wv] {8}
            child {node [nil] {}}
            child {node [nil] {}}
          }
        }
        child { node [wv] {17}
          child { node [wv] {12}
            child {node [nil] {}}
            child {node [nil] {}}
          }
          child { node [wv] {27}
            child {node [nil] {}}
            child {node [nil] {}}
          }
        }
      }
child { node [wv] {41}
        child { node [wv] {39}
          child { node [wv] {31}
            child {node [nil] {}}
            child {node [nil] {}}
          }
          child {node [nil] {}}
        }
        child { node [wv] {46}
          child {node [nil] {}}
          child { node [wv] {47}
            child {node [nil] {}}
            child {node [nil] {}}
          }
        }
            };
\end{tikzpicture}

\end{figure}
\end{frame}

%%%


\begin{frame}[fragile] {\secname : \subsecname}
  Inserir: \textbf{4, 39, 30, 27, 41, 46, 47, 10, 31, 8, 17, 12, 50}, 2, 14
\begin{figure}[!h]
  \centering
  \caption{Inserir 50}

\begin{tikzpicture}[->,>=stealth',
  % level/.style={sibling distance = 8cm/#1, level distance = 1.5cm},
 level 1/.style={sibling distance=15em, level distance = 3em},
  level 2/.style={sibling distance=8em, level distance = 3em},
  level 3/.style={sibling distance=4em, level distance = 3em},
  level 4/.style={sibling distance=2em, level distance = 3em},
  level 5/.style={sibling distance=1em, level distance = 3em} ]
\node [wv] (r){30}
child { node [wv] {10}
        child { node [wv] {4}
          child {node [nil] {}}
          child { node [wv] {8}
            child {node [nil] {}}
            child {node [nil] {}}
          }
        }
        child { node [wv] {17}
          child { node [wv] {12}
            child {node [nil] {}}
            child {node [nil] {}}
          }
          child { node [wv] {27}
            child {node [nil] {}}
            child {node [nil] {}}
          }
        }
      }
child { node [wv] {41}
        child { node [wv] {39}
          child { node [wv] {31}
            child {node [nil] {}}
            child {node [nil] {}}
          }
          child {node [nil] {}}
        }
        child { node [gv,label=above:+2] {46}
          child {node [nil] {}}
          child { node [gv,label=above:+1] {47}
            child {node [nil] {}}
            child { node [gv,label=above:0] {50}
              child {node [nil] {}}
              child {node [nil] {}}
            }
          }
        }
            };
\end{tikzpicture}

\end{figure}
\end{frame}

%%%


\begin{frame}[fragile] {\secname : \subsecname}
  Inserir: \textbf{4, 39, 30, 27, 41, 46, 47, 10, 31, 8, 17, 12, 50}, 2, 14
\begin{figure}[!h]
  \centering
  \caption{RSE 46 AR 47}

\begin{tikzpicture}[->,>=stealth',
  % level/.style={sibling distance = 8cm/#1, level distance = 1.5cm},
 level 1/.style={sibling distance=15em, level distance = 3em},
  level 2/.style={sibling distance=8em, level distance = 3em},
  level 3/.style={sibling distance=4em, level distance = 3em},
  level 4/.style={sibling distance=2em, level distance = 3em},
  level 5/.style={sibling distance=1em, level distance = 3em} ]
\node [wv] (r){30}
child { node [wv] {10}
        child { node [wv] {4}
          child {node [nil] {}}
          child { node [wv] {8}
            child {node [nil] {}}
            child {node [nil] {}}
          }
        }
        child { node [wv] {17}
          child { node [wv] {12}
            child {node [nil] {}}
            child {node [nil] {}}
          }
          child { node [wv] {27}
            child {node [nil] {}}
            child {node [nil] {}}
          }
        }
      }
child { node [wv] {41}
        child { node [wv] {39}
          child { node [wv] {31}
            child {node [nil] {}}
            child {node [nil] {}}
          }
          child {node [nil] {}}
        }
        child { node [wv] {47}
          child { node [wv] {46}
            child {node [nil] {}}
            child {node [nil] {}}
          }
          child { node [wv] {50}
            child {node [nil] {}}
            child {node [nil] {}}
          }
        }
            };
\end{tikzpicture}

\end{figure}
\end{frame}

%%%


\begin{frame}[fragile] {\secname : \subsecname}
  Inserir: \textbf{4, 39, 30, 27, 41, 46, 47, 10, 31, 8, 17, 12, 50, 2}, 14
\begin{figure}[!h]
  \centering
  \caption{Inserir 2}

\begin{tikzpicture}[->,>=stealth',
  % level/.style={sibling distance = 8cm/#1, level distance = 1.5cm},
 level 1/.style={sibling distance=15em, level distance = 3em},
  level 2/.style={sibling distance=8em, level distance = 3em},
  level 3/.style={sibling distance=4em, level distance = 3em},
  level 4/.style={sibling distance=2em, level distance = 3em},
  level 5/.style={sibling distance=1em, level distance = 3em} ]
\node [wv] (r){30}
child { node [wv] {10}
        child { node [wv] {4}
          child { node [wv] {2}
            child {node [nil] {}}
            child {node [nil] {}}
          }
          child { node [wv] {8}
            child {node [nil] {}}
            child {node [nil] {}}
          }
        }
        child { node [wv] {17}
          child { node [wv] {12}
            child {node [nil] {}}
            child {node [nil] {}}
          }
          child { node [wv] {27}
            child {node [nil] {}}
            child {node [nil] {}}
          }
        }
      }
child { node [wv] {41}
        child { node [wv] {39}
          child { node [wv] {31}
            child {node [nil] {}}
            child {node [nil] {}}
          }
          child {node [nil] {}}
        }
        child { node [wv] {47}
          child { node [wv] {46}
            child {node [nil] {}}
            child {node [nil] {}}
          }
          child { node [wv] {50}
            child {node [nil] {}}
            child {node [nil] {}}
          }
        }
            };
\end{tikzpicture}

\end{figure}
\end{frame}

%%%


\begin{frame}[fragile] {\secname : \subsecname}
  Inserir: \textbf{4, 39, 30, 27, 41, 46, 47, 10, 31, 8, 17, 12, 50, 2, 14}
\begin{figure}[!h]
  \centering
  \caption{Inserir 14}

\begin{tikzpicture}[->,>=stealth',
  % level/.style={sibling distance = 8cm/#1, level distance = 1.5cm},
 level 1/.style={sibling distance=15em, level distance = 3em},
  level 2/.style={sibling distance=8em, level distance = 3em},
  level 3/.style={sibling distance=4em, level distance = 3em},
  level 4/.style={sibling distance=2em, level distance = 3em},
  level 5/.style={sibling distance=1em, level distance = 3em} ]
\node [wv] (r){30}
child { node [wv] {10}
        child { node [wv] {4}
          child { node [wv] {2}
            child {node [nil] {}}
            child {node [nil] {}}
          }
          child { node [wv] {8}
            child {node [nil] {}}
            child {node [nil] {}}
          }
        }
        child { node [wv] {17}
          child { node [wv] {12}
            child {node [nil] {}}
            child { node [wv] {14}
              child {node [nil] {}}
              child {node [nil] {}}
            }
          }
          child { node [wv] {27}
            child {node [nil] {}}
            child {node [nil] {}}
          }
        }
      }
child { node [wv] {41}
        child { node [wv] {39}
          child { node [wv] {31}
            child {node [nil] {}}
            child {node [nil] {}}
          }
          child {node [nil] {}}
        }
        child { node [wv] {47}
          child { node [wv] {46}
            child {node [nil] {}}
            child {node [nil] {}}
          }
          child { node [wv] {50}
            child {node [nil] {}}
            child {node [nil] {}}
          }
        }
            };
\end{tikzpicture}
\end{figure}
\end{frame}

%%%



\subsection{Exemplo 3 - 24, 50, 44, 15, 43, 35, 40, 14, 48, 45, 36, 21, 37, 32, 31}


\begin{frame}[noframenumbering,plain,fragile] {\secname : \subsecname}
  Inserir: \textbf{24, 50, 44, 15, 43, 35, 40, 14, 48, 45, 36, 21, 37, 32, 31}
\begin{figure}[!h]
  \centering
  \caption{Inserir 24}
  \begin{tikzpicture}[->,>=stealth',
    % level/.style={sibling distance = 8cm/#1, level distance = 1.5cm},
    level 1/.style={sibling distance=15em, level distance = 3em},
  level 2/.style={sibling distance=8em, level distance = 3em},
  level 3/.style={sibling distance=4em, level distance = 3em},
  level 4/.style={sibling distance=2em, level distance = 3em},
  level 5/.style={sibling distance=1em, level distance = 3em} 
  \node [wv] (r){24}
    child {node [nil] {}}
    child {node [nil] {}};
  \end{tikzpicture}
\end{figure}
\end{frame}


\begin{frame}[noframenumbering,plain,fragile] {\secname : \subsecname}
\begin{figure}[!h]
  \centering
  \caption{Inserir 50}
\begin{tikzpicture}[->,>=stealth',
  % level/.style={sibling distance = 8cm/#1, level distance = 1.5cm},
  level 1/.style={sibling distance=15em, level distance = 3em},
  level 2/.style={sibling distance=8em, level distance = 3em},
  level 3/.style={sibling distance=4em, level distance = 3em},
  level 4/.style={sibling distance=2em, level distance = 3em},
  level 5/.style={sibling distance=1em, level distance = 3em} ]
\node [wv] (r){24}
child {node [nil] {}}
child {node [wv] {50}
  child {node [nil] {}}
  child {node [nil] {}}
};
\end{tikzpicture}
\end{figure}
\end{frame}


\begin{frame}[noframenumbering,plain,fragile] {\secname : \subsecname}
\begin{figure}[!h]
  \centering
  \caption{Inserir 44}

\begin{tikzpicture}[->,>=stealth',
  % level/.style={sibling distance = 8cm/#1, level distance = 1.5cm},
  level 1/.style={sibling distance=15em, level distance = 3em},
  level 2/.style={sibling distance=8em, level distance = 3em},
  level 3/.style={sibling distance=4em, level distance = 3em},
  level 4/.style={sibling distance=2em, level distance = 3em},
  level 5/.style={sibling distance=1em, level distance = 3em} ]
\node [gv,label=above:+2] (r){24}
child {node [nil] {}}
child {node [gv,label=above:-1] {50}
              child { node [gv,label=above:0] {44}
                child {node [nil] {}}
                child {node [nil] {}}
              }
              child {node [nil] {}}
            };
\end{tikzpicture}
\end{figure}
\end{frame}


\begin{frame}[noframenumbering,plain,fragile] {\secname : \subsecname}
\begin{figure}[!h]
 
  \centering
  \caption{RSD 50 AR 44 + RSE 24 AR 44}
\begin{tikzpicture}[->,>=stealth',
  % level/.style={sibling distance = 8cm/#1, level distance = 1.5cm},
  level 1/.style={sibling distance=15em, level distance = 3em},
  level 2/.style={sibling distance=8em, level distance = 3em},
  level 3/.style={sibling distance=4em, level distance = 3em},
  level 4/.style={sibling distance=2em, level distance = 3em},
  level 5/.style={sibling distance=1em, level distance = 3em} ]
\node [wv] (r){44}
child { node [wv] {24}
                child {node [nil] {}}
                child {node [nil] {}}
              }
child { node [wv] {50}
              child {node [nil] {}}
              child {node [nil] {}}
            };
\end{tikzpicture}
\end{figure}
\end{frame}


\begin{frame}[noframenumbering,plain,fragile] {\secname : \subsecname}
\begin{figure}[!h]
  \centering
  \caption{Inserir 15}

\begin{tikzpicture}[->,>=stealth',
  % level/.style={sibling distance = 8cm/#1, level distance = 1.5cm},
  level 1/.style={sibling distance=15em, level distance = 3em},
  level 2/.style={sibling distance=8em, level distance = 3em},
  level 3/.style={sibling distance=4em, level distance = 3em},
  level 4/.style={sibling distance=2em, level distance = 3em},
  level 5/.style={sibling distance=1em, level distance = 3em} ]
\node [wv] (r){44}
child { node [wv] {24}
  child { node [wv] {15}
    child {node [nil] {}}
    child {node [nil] {}}
  }
  child {node [nil] {}}
}
child { node [wv] {50}
  child {node [nil] {}}
  child {node [nil] {}}
};
\end{tikzpicture}
\end{figure}
\end{frame}


\begin{frame}[noframenumbering,plain,fragile] {\secname : \subsecname}
\begin{figure}[!h]
  \centering
  \caption{Inserir 43}

\begin{tikzpicture}[->,>=stealth',
  % level/.style={sibling distance = 8cm/#1, level distance = 1.5cm},
  level 1/.style={sibling distance=15em, level distance = 3em},
  level 2/.style={sibling distance=8em, level distance = 3em},
  level 3/.style={sibling distance=4em, level distance = 3em},
  level 4/.style={sibling distance=2em, level distance = 3em},
  level 5/.style={sibling distance=1em, level distance = 3em} ]
\node [wv] (r){44}
child { node [wv] {24}
  child { node [wv] {15}
    child {node [nil] {}}
    child {node [nil] {}}
  }
  child { node [wv] {43}
    child {node [nil] {}}
    child {node [nil] {}}
  }
}
child { node [wv] {50}
  child {node [nil] {}}
  child {node [nil] {}}
};
\end{tikzpicture}
\end{figure}
\end{frame}


\begin{frame}[noframenumbering,plain,fragile] {\secname : \subsecname}
\begin{figure}[!h]
  \centering
  \caption{Inserir 35}

\begin{tikzpicture}[->,>=stealth',
  % level/.style={sibling distance = 8cm/#1, level distance = 1.5cm},
  level 1/.style={sibling distance=15em, level distance = 3em},
  level 2/.style={sibling distance=8em, level distance = 3em},
  level 3/.style={sibling distance=4em, level distance = 3em},
  level 4/.style={sibling distance=2em, level distance = 3em},
  level 5/.style={sibling distance=1em, level distance = 3em} ]
\node [gv,label=above:-2] (r){44}
child { node [gv,label=above:+1] {24}
  child { node [wv] {15}
    child {node [nil] {}}
    child {node [nil] {}}
  }
  child { node [gv,label=above:0] {43}
    child { node [wv] {35}
      child {node [nil] {}}
      child {node [nil] {}}
    }
    child {node [nil] {}}
  }
}
child { node [wv] {50}
  child {node [nil] {}}
  child {node [nil] {}}
};
\end{tikzpicture}
\end{figure}
\end{frame}


\begin{frame}[noframenumbering,plain,fragile] {\secname : \subsecname}
\begin{figure}[!h]
  \centering
  \caption{RSE 24 AR 43 + RSD 44 AR 43}

\begin{tikzpicture}[->,>=stealth',
  % level/.style={sibling distance = 8cm/#1, level distance = 1.5cm},
  level 1/.style={sibling distance=15em, level distance = 3em},
  level 2/.style={sibling distance=8em, level distance = 3em},
  level 3/.style={sibling distance=4em, level distance = 3em},
  level 4/.style={sibling distance=2em, level distance = 3em},
  level 5/.style={sibling distance=1em, level distance = 3em} ]
\node [wv] (r){43}
child { node [wv] {24}
  child { node [wv] {15}
    child {node [nil] {}}
    child {node [nil] {}}
  }
  child { node [wv] {35}
      child {node [nil] {}}
      child {node [nil] {}}
    }
}
child { node [wv] {44}
  child {node [nil] {}}
  child { node [wv] {50}
    child {node [nil] {}}
    child {node [nil] {}}
  }
};
\end{tikzpicture}

\end{figure}
\end{frame}


\begin{frame}[noframenumbering,plain,fragile] {\secname : \subsecname}
\begin{figure}[!h]
  \centering
  \caption{Inserir 40}

\begin{tikzpicture}[->,>=stealth',
  % level/.style={sibling distance = 8cm/#1, level distance = 1.5cm},
  level 1/.style={sibling distance=15em, level distance = 3em},
  level 2/.style={sibling distance=8em, level distance = 3em},
  level 3/.style={sibling distance=4em, level distance = 3em},
  level 4/.style={sibling distance=2em, level distance = 3em},
  level 5/.style={sibling distance=1em, level distance = 3em} ]
\node [wv] (r){43}
child { node [wv] {24}
  child { node [wv] {15}
    child {node [nil] {}}
    child {node [nil] {}}
  }
  child { node [wv] {35}
    child {node [nil] {}}
    child { node [wv] {40}
      child {node [nil] {}}
      child {node [nil] {}}
    }
  }
}
child { node [wv] {44}
  child {node [nil] {}}
  child { node [wv] {50}
    child {node [nil] {}}
    child {node [nil] {}}
  }
};
\end{tikzpicture}
\end{figure}
\end{frame}


\begin{frame}[noframenumbering,plain,fragile] {\secname : \subsecname}
\begin{figure}[!h]
  \centering
  \caption{Inserir 14}

\begin{tikzpicture}[->,>=stealth',
  % level/.style={sibling distance = 8cm/#1, level distance = 1.5cm},
  level 1/.style={sibling distance=15em, level distance = 3em},
  level 2/.style={sibling distance=8em, level distance = 3em},
  level 3/.style={sibling distance=4em, level distance = 3em},
  level 4/.style={sibling distance=2em, level distance = 3em},
  level 5/.style={sibling distance=1em, level distance = 3em} ]
\node [wv] (r){43}
child { node [wv] {24}
  child { node [wv] {15}
    child { node [wv] {14}
      child {node [nil] {}}
      child {node [nil] {}}
    }
    child {node [nil] {}}
  }
  child { node [wv] {35}
    child {node [nil] {}}
    child { node [wv] {40}
      child {node [nil] {}}
      child {node [nil] {}}
    }
  }
}
child { node [wv] {44}
  child {node [nil] {}}
  child { node [wv] {50}
    child {node [nil] {}}
    child {node [nil] {}}
  }
};
\end{tikzpicture}

\end{figure}
\end{frame}


\begin{frame}[noframenumbering,plain,fragile] {\secname : \subsecname}
\begin{figure}[!h]
  \centering
  \caption{Inserir 48}

\begin{tikzpicture}[->,>=stealth',
  % level/.style={sibling distance = 8cm/#1, level distance = 1.5cm},
  level 1/.style={sibling distance=15em, level distance = 3em},
  level 2/.style={sibling distance=8em, level distance = 3em},
  level 3/.style={sibling distance=4em, level distance = 3em},
  level 4/.style={sibling distance=2em, level distance = 3em},
  level 5/.style={sibling distance=1em, level distance = 3em} ]
\node [wv] (r){43}
child { node [wv] {24}
  child { node [wv] {15}
    child { node [wv] {14}
      child {node [nil] {}}
      child {node [nil] {}}
    }
    child {node [nil] {}}
  }
  child { node [wv] {35}
    child {node [nil] {}}
    child { node [wv] {40}
      child {node [nil] {}}
      child {node [nil] {}}
    }
  }
}
child { node [gv,label=above:+2] {44}
  child {node [nil] {}}
  child { node [gv,label=above:-1] {50}
    child { node [gv,label=above:0] {48}
      child {node [nil] {}}
      child {node [nil] {}}
    }
    child {node [nil] {}}
  }
};
\end{tikzpicture}
\end{figure}
\end{frame}


\begin{frame}[noframenumbering,plain,fragile] {\secname : \subsecname}
\begin{figure}[!h]
  \centering

  \caption{RSD 50 AR 48 + RSE 40 AR 48}

\begin{tikzpicture}[->,>=stealth',
  % level/.style={sibling distance = 8cm/#1, level distance = 1.5cm},
  level 1/.style={sibling distance=15em, level distance = 3em},
  level 2/.style={sibling distance=8em, level distance = 3em},
  level 3/.style={sibling distance=4em, level distance = 3em},
  level 4/.style={sibling distance=2em, level distance = 3em},
  level 5/.style={sibling distance=1em, level distance = 3em} ]
\node [wv] (r){43}
child { node [wv] {24}
  child { node [wv] {15}
    child { node [wv] {14}
      child {node [nil] {}}
      child {node [nil] {}}
    }
    child {node [nil] {}}
  }
  child { node [wv] {35}
    child {node [nil] {}}
    child { node [wv] {40}
      child {node [nil] {}}
      child {node [nil] {}}
    }
  }
}
child { node [wv] {48}
  child { node [wv] {44}
    child {node [nil] {}}
    child {node [nil] {}}
  }
  child { node [wv] {50}
    child {node [nil] {}}
    child {node [nil] {}}
  }
};
\end{tikzpicture}

\end{figure}
\end{frame}


\begin{frame}[noframenumbering,plain,fragile] {\secname : \subsecname}
\begin{figure}[!h]
  \centering
  \caption{Inserir 45}

\begin{tikzpicture}[->,>=stealth',
  % level/.style={sibling distance = 8cm/#1, level distance = 1.5cm},
  level 1/.style={sibling distance=15em, level distance = 3em},
  level 2/.style={sibling distance=8em, level distance = 3em},
  level 3/.style={sibling distance=4em, level distance = 3em},
  level 4/.style={sibling distance=2em, level distance = 3em},
  level 5/.style={sibling distance=1em, level distance = 3em} ]
\node [wv] (r){43}
child { node [wv] {24}
  child { node [wv] {15}
    child { node [wv] {14}
      child {node [nil] {}}
      child {node [nil] {}}
    }
    child {node [nil] {}}
  }
  child { node [wv] {35}
    child {node [nil] {}}
    child { node [wv] {40}
      child {node [nil] {}}
      child {node [nil] {}}
    }
  }
}
child { node [wv] {48}
  child { node [wv] {44}
    child {node [nil] {}}
    child { node [wv] {45}
      child {node [nil] {}}
      child {node [nil] {}}
    }
  }
  child { node [wv] {50}
    child {node [nil] {}}
    child {node [nil] {}}
  }
};
\end{tikzpicture}
\end{figure}
\end{frame}


\begin{frame}[noframenumbering,plain,fragile] {\secname : \subsecname}
\begin{figure}[!h]
  \centering
  \caption{Inserir 36}

\begin{tikzpicture}[->,>=stealth',
  % level/.style={sibling distance = 8cm/#1, level distance = 1.5cm},
  level 1/.style={sibling distance=15em, level distance = 3em},
  level 2/.style={sibling distance=8em, level distance = 3em},
  level 3/.style={sibling distance=4em, level distance = 3em},
  level 4/.style={sibling distance=2em, level distance = 3em},
  level 5/.style={sibling distance=1em, level distance = 3em} ]
\node [wv] (r){43}
child { node [wv] {24}
  child { node [wv] {15}
    child { node [wv] {14}
      child {node [nil] {}}
      child {node [nil] {}}
    }
    child {node [nil] {}}
  }
  child { node [gv,label=above:+2] {35}
    child {node [nil] {}}
    child { node [gv,label=above:-1] {40}
      child { node [gv,label=above:0] {36}
        child {node [nil] {}}
        child {node [nil] {}}
      }
      child {node [nil] {}}
    }
  }
}
child { node [wv] {48}
  child { node [wv] {44}
    child {node [nil] {}}
    child { node [wv] {45}
      child {node [nil] {}}
      child {node [nil] {}}
    }
  }
  child { node [wv] {50}
    child {node [nil] {}}
    child {node [nil] {}}
  }
};
\end{tikzpicture}
\end{figure}
\end{frame}


\begin{frame}[noframenumbering,plain,fragile] {\secname : \subsecname}
\begin{figure}[!h]
  \centering
  \caption{RSD 40 AR 36 + RSE 35 AR 36}

\begin{tikzpicture}[->,>=stealth',
  % level/.style={sibling distance = 8cm/#1, level distance = 1.5cm},
  level 1/.style={sibling distance=15em, level distance = 3em},
  level 2/.style={sibling distance=8em, level distance = 3em},
  level 3/.style={sibling distance=4em, level distance = 3em},
  level 4/.style={sibling distance=2em, level distance = 3em},
  level 5/.style={sibling distance=1em, level distance = 3em} ]
\node [wv] (r){43}
child { node [wv] {24}
  child { node [wv] {15}
    child { node [wv] {14}
      child {node [nil] {}}
      child {node [nil] {}}
    }
    child {node [nil] {}}
  }
  child { node [wv] {36}
    child { node [wv] {35}
      child {node [nil] {}}
      child {node [nil] {}}
    }
    child { node [wv] {40}
      child {node [nil] {}}
      child {node [nil] {}}
    }
  }
}
child { node [wv] {48}
  child { node [wv] {44}
    child {node [nil] {}}
    child { node [wv] {45}
      child {node [nil] {}}
      child {node [nil] {}}
    }
  }
  child { node [wv] {50}
    child {node [nil] {}}
    child {node [nil] {}}
  }
};
\end{tikzpicture}
\end{figure}
\end{frame}


\begin{frame}[noframenumbering,plain,fragile] {\secname : \subsecname}
\begin{figure}[!h]
  \centering
  \caption{Inserir 21}
  \begin{tikzpicture}[->,>=stealth',
  % level/.style={sibling distance = 8cm/#1, level distance = 1.5cm},
  level 1/.style={sibling distance=15em, level distance = 3em},
  level 2/.style={sibling distance=8em, level distance = 3em},
  level 3/.style={sibling distance=4em, level distance = 3em},
  level 4/.style={sibling distance=2em, level distance = 3em},
  level 5/.style={sibling distance=1em, level distance = 3em} ]
\node [wv] (r){43}
child { node [wv] {24}
  child { node [wv] {15}
    child { node [wv] {14}
      child {node [nil] {}}
      child {node [nil] {}}
    }
    child { node [wv] {21}
      child {node [nil] {}}
      child {node [nil] {}}
    }
  }
  child { node [wv] {36}
    child { node [wv] {35}
      child {node [nil] {}}
      child {node [nil] {}}
    }
    child { node [wv] {40}
      child {node [nil] {}}
      child {node [nil] {}}
    }
  }
}
child { node [wv] {48}
  child { node [wv] {44}
    child {node [nil] {}}
    child { node [wv] {45}
      child {node [nil] {}}
      child {node [nil] {}}
    }
  }
  child { node [wv] {50}
    child {node [nil] {}}
    child {node [nil] {}}
  }
};
\end{tikzpicture}

\end{figure}
\end{frame}


\begin{frame}[noframenumbering,plain,fragile] {\secname : \subsecname}
\begin{figure}[!h]
  \centering
  \caption{Inserir 37}
  \begin{tikzpicture}[->,>=stealth',
  % level/.style={sibling distance = 8cm/#1, level distance = 1.5cm},
  level 1/.style={sibling distance=15em, level distance = 3em},
  level 2/.style={sibling distance=8em, level distance = 3em},
  level 3/.style={sibling distance=4em, level distance = 3em},
  level 4/.style={sibling distance=2em, level distance = 3em},
  level 5/.style={sibling distance=1em, level distance = 3em} ]
\node [wv] (r){43}
child { node [wv] {24}
  child { node [wv] {15}
    child { node [wv] {14}
      child {node [nil] {}}
      child {node [nil] {}}
    }
    child { node [wv] {21}
      child {node [nil] {}}
      child {node [nil] {}}
    }
  }
  child { node [wv] {36}
    child { node [wv] {35}
      child {node [nil] {}}
      child {node [nil] {}}
    }
    child { node [wv] {40}
      child { node [wv] {37}
        child {node [nil] {}}
        child {node [nil] {}}
      }
      child {node [nil] {}}
    }
  }
}
child { node [wv] {48}
  child { node [wv] {44}
    child {node [nil] {}}
    child { node [wv] {45}
      child {node [nil] {}}
      child {node [nil] {}}
    }
  }
  child { node [wv] {50}
    child {node [nil] {}}
    child {node [nil] {}}
  }
};
\end{tikzpicture}

\end{figure}
\end{frame}


\begin{frame}[noframenumbering,plain,fragile] {\secname : \subsecname}
\begin{figure}[!h]
  \centering
  \caption{Inserir 32}
  \begin{tikzpicture}[->,>=stealth',
  % level/.style={sibling distance = 8cm/#1, level distance = 1.5cm},
  level 1/.style={sibling distance=15em, level distance = 3em},
  level 2/.style={sibling distance=8em, level distance = 3em},
  level 3/.style={sibling distance=4em, level distance = 3em},
  level 4/.style={sibling distance=2em, level distance = 3em},
  level 5/.style={sibling distance=1em, level distance = 3em} ]
\node [wv] (r){43}
child { node [wv] {24}
  child { node [wv] {15}
    child { node [wv] {14}
      child {node [nil] {}}
      child {node [nil] {}}
    }
    child { node [wv] {21}
      child {node [nil] {}}
      child {node [nil] {}}
    }
  }
  child { node [wv] {36}
    child { node [wv] {35}
      child { node [wv] {32}
        child {node [nil] {}}
        child {node [nil] {}}
      }
      child {node [nil] {}}
    }
    child { node [wv] {40}
      child { node [wv] {37}
        child {node [nil] {}}
        child {node [nil] {}}
      }
      child {node [nil] {}}
    }
  }
}
child { node [wv] {48}
  child { node [wv] {44}
    child {node [nil] {}}
    child { node [wv] {45}
      child {node [nil] {}}
      child {node [nil] {}}
    }
  }
  child { node [wv] {50}
    child {node [nil] {}}
    child {node [nil] {}}
  }
};
\end{tikzpicture}

\end{figure}
\end{frame}


\begin{frame}[noframenumbering,plain,fragile] {\secname : \subsecname}
\begin{figure}[!h]
  \centering
  \caption{Inserir 31}
  \begin{tikzpicture}[->,>=stealth',
  % level/.style={sibling distance = 8cm/#1, level distance = 1.5cm},
  level 1/.style={sibling distance=15em, level distance = 3em},
  level 2/.style={sibling distance=8em, level distance = 3em},
  level 3/.style={sibling distance=4em, level distance = 3em},
  level 4/.style={sibling distance=2em, level distance = 3em},
  level 5/.style={sibling distance=1em, level distance = 3em} ]
\node [wv] (r){43}
child { node [wv] {24}
  child { node [wv] {15}
    child { node [wv] {14}
      child {node [nil] {}}
      child {node [nil] {}}
    }
    child { node [wv] {21}
      child {node [nil] {}}
      child {node [nil] {}}
    }
  }
  child { node [wv] {36}
    child { node [gv,label=above:-2] {35}
      child { node [gv,label=above:-1] {32}
        child { node [gv,label=above:0] {31}
          child {node [nil] {}}
          child {node [nil] {}}
        }
        child {node [nil] {}}
      }
      child {node [nil] {}}
    }
    child { node [wv] {40}
      child { node [wv] {37}
        child {node [nil] {}}
        child {node [nil] {}}
      }
      child {node [nil] {}}
    }
  }
}
child { node [wv] {48}
  child { node [wv] {44}
    child {node [nil] {}}
    child { node [wv] {45}
      child {node [nil] {}}
      child {node [nil] {}}
    }
  }
  child { node [wv] {50}
    child {node [nil] {}}
    child {node [nil] {}}
  }
};
\end{tikzpicture}

\end{figure}
\end{frame}


\begin{frame}[noframenumbering,plain,fragile] {\secname : \subsecname}
\begin{figure}[!h]
  \centering
  \caption{RSE 36 AR 32}
  \begin{tikzpicture}[->,>=stealth',
  % level/.style={sibling distance = 8cm/#1, level distance = 1.5cm},
  level 1/.style={sibling distance=15em, level distance = 3em},
  level 2/.style={sibling distance=8em, level distance = 3em},
  level 3/.style={sibling distance=4em, level distance = 3em},
  level 4/.style={sibling distance=2em, level distance = 3em},
  level 5/.style={sibling distance=1em, level distance = 3em} ]
\node [wv] (r){43}
child { node [wv] {24}
  child { node [wv] {15}
    child { node [wv] {14}
      child {node [nil] {}}
      child {node [nil] {}}
    }
    child { node [wv] {21}
      child {node [nil] {}}
      child {node [nil] {}}
    }
  }
  child { node [wv] {36}
    child { node [wv] {32}
      child { node [wv] {31}
        child {node [nil] {}}
        child {node [nil] {}}
      }
      child { node [wv] {35}
        child {node [nil] {}}
        child {node [nil] {}}
      }
    }
    child { node [wv] {40}
      child { node [wv] {37}
        child {node [nil] {}}
        child {node [nil] {}}
      }
      child {node [nil] {}}
    }
  }
}
child { node [wv] {48}
  child { node [wv] {44}
    child {node [nil] {}}
    child { node [wv] {45}
      child {node [nil] {}}
      child {node [nil] {}}
    }
  }
  child { node [wv] {50}
    child {node [nil] {}}
    child {node [nil] {}}
  }
};
\end{tikzpicture}

\end{figure}
\end{frame}


\subsection{Exemplo 4 - Remover 38, 31, 40, 44, 10, 42, 33, 26}
% gv,label=above:+2


\begin{frame}[noframenumbering,plain,fragile] {\secname : \subsecname}
  Fazer para casa
  Remover: \textbf{38, 31, 40, 44, 10, 42, 33, 26}

\begin{figure}[!h]
  \centering
  \caption{Inserir 33}

\begin{tikzpicture}[->,>=stealth',
  % level/.style={sibling distance = 8cm/#1, level distance = 1.5cm},
  level 1/.style={sibling distance=15em, level distance = 3em},
  level 2/.style={sibling distance=8em, level distance = 3em},
  level 3/.style={sibling distance=4em, level distance = 3em},
  level 4/.style={sibling distance=2em, level distance = 3em},
  level 5/.style={sibling distance=1em, level distance = 3em} ]
  \node [wv] (r){31}
  child { node [wv] {19}
    child { node [wv] {9}
      child { node [wv] {7}
        child { node [wv] {1}
          child {node [nil] {}}
          child {node [nil] {}}
        }
        child {node [nil] {}}
      }
      child { node [wv] {10}
        child {node [nil] {}}
        child {node [nil] {}}
      }
    }
    child { node [wv] {30}
        child { node [wv] {26}
            child {node [nil] {}}
            child {node [nil] {}}
        }
        child {node [nil] {}}
    }
  }
  child { node [wv] {40}
    child { node [wv] {35}
      child { node [wv] {34}
        child { node [wv] {33}
          child {node [nil] {}}
          child {node [nil] {}}
        }
        child {node [nil] {}}
      }
      child { node [wv] {38}
          child {node [nil] {}}
          child {node [nil] {}}
        }
    }
    child { node [wv] {42}
      child {node [nil] {}}
      child { node [wv] {44}
        child {node [nil] {}}
        child {node [nil] {}}
      }
    }
  }
  ;
\end{tikzpicture}
\end{figure}
\end{frame}





\subsection{Exemplo 5 - 6, 29, 5, 9, 16, 28, 49, 36, 22, 39, 27, 37, 1, 4, 48}

\begin{frame}[noframenumbering,plain,fragile]{\secname : \subsecname}
  Inserir: \textbf{6, 29, 5, 9, 16, 28, 49, 36, 22, 39, 27, 37, 1, 4, 48}
\begin{figure}[!h]
  \centering
  \caption{Inserir 6}
  \begin{tikzpicture}[->,>=stealth',
    % level/.style={sibling distance = 8cm/#1, level distance = 1.5cm},
    level 1/.style={sibling distance=15em, level distance = 3em},
  level 2/.style={sibling distance=8em, level distance = 3em},
  level 3/.style={sibling distance=4em, level distance = 3em},
  level 4/.style={sibling distance=2em, level distance = 3em},
  level 5/.style={sibling distance=1em, level distance = 3em}]
    \node [wv] (r){6}
    child {node [nil] {}}
    child {node [nil] {}};
  \end{tikzpicture}
\end{figure}
\end{frame}


\begin{frame}[noframenumbering,plain,fragile] {\secname : \subsecname}

\begin{figure}[!h]
  \centering
  \caption{Inserir 29}
\begin{tikzpicture}[->,>=stealth',
  % level/.style={sibling distance = 8cm/#1, level distance = 1.5cm},
  level 1/.style={sibling distance=15em, level distance = 3em},
  level 2/.style={sibling distance=8em, level distance = 3em},
  level 3/.style={sibling distance=4em, level distance = 3em},
  level 4/.style={sibling distance=2em, level distance = 3em},
  level 5/.style={sibling distance=1em, level distance = 3em}]
\node [wv] (r){6}

child {node [nil] {}}
child { node [wv] {29}
              child {node [nil] {}}
              child {node [nil] {}}
            };
\end{tikzpicture}
\end{figure}
\end{frame}


\begin{frame}[noframenumbering,plain,fragile] {\secname : \subsecname}

\begin{figure}[!h]
  \centering
  \caption{Inserir 5}

\begin{tikzpicture}[->,>=stealth',
  % level/.style={sibling distance = 8cm/#1, level distance = 1.5cm},
  level 1/.style={sibling distance=15em, level distance = 3em},
  level 2/.style={sibling distance=8em, level distance = 3em},
  level 3/.style={sibling distance=4em, level distance = 3em},
  level 4/.style={sibling distance=2em, level distance = 3em},
  level 5/.style={sibling distance=1em, level distance = 3em}]
\node [wv] (r){6}
child { node [wv] {5}
              child {node [nil] {}}
              child {node [nil] {}}
            }
child { node [wv] {29}
              child {node [nil] {}}
              child {node [nil] {}}
            };
\end{tikzpicture}
\end{figure}
\end{frame}


\begin{frame}[noframenumbering,plain,fragile] {\secname : \subsecname}

\begin{figure}[!h]
  \centering
  \caption{Inserir 9}

\begin{tikzpicture}[->,>=stealth',
  % level/.style={sibling distance = 8cm/#1, level distance = 1.5cm},
  level 1/.style={sibling distance=15em, level distance = 3em},
  level 2/.style={sibling distance=8em, level distance = 3em},
  level 3/.style={sibling distance=4em, level distance = 3em},
  level 4/.style={sibling distance=2em, level distance = 3em},
  level 5/.style={sibling distance=1em, level distance = 3em}]
\node [wv] (r){6}
child { node [wv] {5}
              child {node [nil] {}}
              child {node [nil] {}}
            }
child { node [wv] {29}
            child { node [wv] {9}
              child {node [nil] {}}
             child {node [nil] {}}
            }
              child {node [nil] {}}
            };
\end{tikzpicture}
\end{figure}
\end{frame}


\begin{frame}[noframenumbering,plain,fragile] {\secname : \subsecname}

\begin{figure}[!h]
  \centering
  \caption{Inserir 16}

\begin{tikzpicture}[->,>=stealth',
  % level/.style={sibling distance = 8cm/#1, level distance = 1.5cm},
  level 1/.style={sibling distance=15em, level distance = 3em},
  level 2/.style={sibling distance=8em, level distance = 3em},
  level 3/.style={sibling distance=4em, level distance = 3em},
  level 4/.style={sibling distance=2em, level distance = 3em},
  level 5/.style={sibling distance=1em, level distance = 3em}]
\node [wv] (r){6}
child { node [wv] {5}
              child {node [nil] {}}
              child {node [nil] {}}
            }
child { node [gv,label=above:-2] {29}
            child { node [gv,label=above:+1] {9}
              child {node [nil] {}}
              child { node [gv,label=above:0] {16}
                child {node [nil] {}}
                child {node [nil] {}}
              }
            }
              child {node [nil] {}}
            };
\end{tikzpicture}
\end{figure}
\end{frame}


\begin{frame}[noframenumbering,plain,fragile] {\secname : \subsecname}

\begin{figure}[!h]
  \centering
  \caption{RSE 9 AR 16 + RSD 29 AR 16}

\begin{tikzpicture}[->,>=stealth',
  % level/.style={sibling distance = 8cm/#1, level distance = 1.5cm},
  level 1/.style={sibling distance=15em, level distance = 3em},
  level 2/.style={sibling distance=8em, level distance = 3em},
  level 3/.style={sibling distance=4em, level distance = 3em},
  level 4/.style={sibling distance=2em, level distance = 3em},
  level 5/.style={sibling distance=1em, level distance = 3em}]
\node [wv] (r){6}
child { node [wv] {5}
              child {node [nil] {}}
              child {node [nil] {}}
            }
child { node [wv] {16}
  child { node [wv] {9}
    child {node [nil] {}}
    child {node [nil] {}}
  }
  child { node [wv] {29}
    child {node [nil] {}}
    child {node [nil] {}}
  }
};
\end{tikzpicture}
\end{figure}
\end{frame}


\begin{frame}[noframenumbering,plain,fragile] {\secname : \subsecname}

\begin{figure}[!h]
  \centering
  \caption{Inserir 28}

\begin{tikzpicture}[->,>=stealth',
  % level/.style={sibling distance = 8cm/#1, level distance = 1.5cm},
  level 1/.style={sibling distance=15em, level distance = 3em},
  level 2/.style={sibling distance=8em, level distance = 3em},
  level 3/.style={sibling distance=4em, level distance = 3em},
  level 4/.style={sibling distance=2em, level distance = 3em},
  level 5/.style={sibling distance=1em, level distance = 3em}]
\node [gv,label=above:+2] (r){6}
child { node [wv] {5}
              child {node [nil] {}}
              child {node [nil] {}}
            }
child { node [gv,label=above:+1] {16}
  child { node [wv] {9}
    child {node [nil] {}}
    child {node [nil] {}}
  }
  child { node [gv,label=above:0] {29}
    child { node [wv] {28}
      child {node [nil] {}}
      child {node [nil] {}}
    }
    child {node [nil] {}}
  }
};
\end{tikzpicture}
\end{figure}
\end{frame}


\begin{frame}[noframenumbering,plain,fragile] {\secname : \subsecname}

\begin{figure}[!h]
  \centering
  \caption{RSE 6 AR 16}

\begin{tikzpicture}[->,>=stealth',
  % level/.style={sibling distance = 8cm/#1, level distance = 1.5cm},
  level 1/.style={sibling distance=15em, level distance = 3em},
  level 2/.style={sibling distance=8em, level distance = 3em},
  level 3/.style={sibling distance=4em, level distance = 3em},
  level 4/.style={sibling distance=2em, level distance = 3em},
  level 5/.style={sibling distance=1em, level distance = 3em}]
\node [wv] (r){16}
child { node [wv] {6}
  child { node [wv] {5}
    child {node [nil] {}}
    child {node [nil] {}}
  }
  child { node [wv] {9}
    child {node [nil] {}}
    child {node [nil] {}}
  }
}
child { node [wv] {29}
    child { node [wv] {28}
      child {node [nil] {}}
      child {node [nil] {}}
    }
    child {node [nil] {}}
};
\end{tikzpicture}
\end{figure}
\end{frame}


\begin{frame}[noframenumbering,plain,fragile] {\secname : \subsecname}

\begin{figure}[!h]
  \centering
  \caption{Inserir 49}

\begin{tikzpicture}[->,>=stealth',
  % level/.style={sibling distance = 8cm/#1, level distance = 1.5cm},
  level 1/.style={sibling distance=15em, level distance = 3em},
  level 2/.style={sibling distance=8em, level distance = 3em},
  level 3/.style={sibling distance=4em, level distance = 3em},
  level 4/.style={sibling distance=2em, level distance = 3em},
  level 5/.style={sibling distance=1em, level distance = 3em}]
\node [wv] (r){16}
child { node [wv] {6}
  child { node [wv] {5}
    child {node [nil] {}}
    child {node [nil] {}}
  }
  child { node [wv] {9}
    child {node [nil] {}}
    child {node [nil] {}}
  }
}
child { node [wv] {29}
    child { node [wv] {28}
      child {node [nil] {}}
      child {node [nil] {}}
    }
    child { node [wv] {49}
      child {node [nil] {}}
      child {node [nil] {}}
    }
};
\end{tikzpicture}
\end{figure}
\end{frame}


\begin{frame}[noframenumbering,plain,fragile] {\secname : \subsecname}

\begin{figure}[!h]
  \centering
  \caption{Inserir 36}

\begin{tikzpicture}[->,>=stealth',
  % level/.style={sibling distance = 8cm/#1, level distance = 1.5cm},
  level 1/.style={sibling distance=15em, level distance = 3em},
  level 2/.style={sibling distance=8em, level distance = 3em},
  level 3/.style={sibling distance=4em, level distance = 3em},
  level 4/.style={sibling distance=2em, level distance = 3em},
  level 5/.style={sibling distance=1em, level distance = 3em}]
\node [wv] (r){16}
child { node [wv] {6}
  child { node [wv] {5}
    child {node [nil] {}}
    child {node [nil] {}}
  }
  child { node [wv] {9}
    child {node [nil] {}}
    child {node [nil] {}}
  }
}
child { node [wv] {29}
    child { node [wv] {28}
      child {node [nil] {}}
      child {node [nil] {}}
    }
    child { node [wv] {49}
      child { node [wv] {36}
        child {node [nil] {}}
        child {node [nil] {}}
      }
      child {node [nil] {}}
    }
};
\end{tikzpicture}
\end{figure}
\end{frame}


\begin{frame}[noframenumbering,plain,fragile] {\secname : \subsecname}

\begin{figure}[!h]
  \centering
  \caption{Inserir 22}

\begin{tikzpicture}[->,>=stealth',
  % level/.style={sibling distance = 8cm/#1, level distance = 1.5cm},
  level 1/.style={sibling distance=15em, level distance = 3em},
  level 2/.style={sibling distance=8em, level distance = 3em},
  level 3/.style={sibling distance=4em, level distance = 3em},
  level 4/.style={sibling distance=2em, level distance = 3em},
  level 5/.style={sibling distance=1em, level distance = 3em}]
\node [wv] (r){16}
child { node [wv] {6}
  child { node [wv] {5}
    child {node [nil] {}}
    child {node [nil] {}}
  }
  child { node [wv] {9}
    child {node [nil] {}}
    child {node [nil] {}}
  }
}
child { node [wv] {29}
    child { node [wv] {28}
      child { node [wv] {22}
        child {node [nil] {}}
        child {node [nil] {}}
      }
      child {node [nil] {}}
    }
    child { node [wv] {49}
      child { node [wv] {36}
        child {node [nil] {}}
        child {node [nil] {}}
      }
      child {node [nil] {}}
    }
};
\end{tikzpicture}
\end{figure}
\end{frame}


\begin{frame}[noframenumbering,plain,fragile] {\secname : \subsecname}

\begin{figure}[!h]
  \centering
  \caption{Inserir 39}

\begin{tikzpicture}[->,>=stealth',
  % level/.style={sibling distance = 8cm/#1, level distance = 1.5cm},
  level 1/.style={sibling distance=15em, level distance = 3em},
  level 2/.style={sibling distance=8em, level distance = 3em},
  level 3/.style={sibling distance=4em, level distance = 3em},
  level 4/.style={sibling distance=2em, level distance = 3em},
  level 5/.style={sibling distance=1em, level distance = 3em}]
\node [wv] (r){16}
child { node [wv] {6}
  child { node [wv] {5}
    child {node [nil] {}}
    child {node [nil] {}}
  }
  child { node [wv] {9}
    child {node [nil] {}}
    child {node [nil] {}}
  }
}
child { node [wv] {29}
    child { node [wv] {28}
      child { node [wv] {22}
        child {node [nil] {}}
        child {node [nil] {}}
      }
      child {node [nil] {}}
    }
    child { node [gv,label=above:-2] {49}
      child { node [gv,label=above:+1] {36}
        child {node [nil] {}}
        child { node [gv,label=above:0] {39}
          child {node [nil] {}}
          child {node [nil] {}}
        }
      }
      child {node [nil] {}}
    }
};
\end{tikzpicture}
\end{figure}
\end{frame}


\begin{frame}[noframenumbering,plain,fragile] {\secname : \subsecname}

\begin{figure}[!h]
  \centering
  \caption{RSE 36 AR 39 + RSD 49 AR 39}

\begin{tikzpicture}[->,>=stealth',
  % level/.style={sibling distance = 8cm/#1, level distance = 1.5cm},
  level 1/.style={sibling distance=15em, level distance = 3em},
  level 2/.style={sibling distance=8em, level distance = 3em},
  level 3/.style={sibling distance=4em, level distance = 3em},
  level 4/.style={sibling distance=2em, level distance = 3em},
  level 5/.style={sibling distance=1em, level distance = 3em}]
\node [wv] (r){16}
child { node [wv] {6}
  child { node [wv] {5}
    child {node [nil] {}}
    child {node [nil] {}}
  }
  child { node [wv] {9}
    child {node [nil] {}}
    child {node [nil] {}}
  }
}
child { node [wv] {29}
  child { node [wv] {28}
    child { node [wv] {22}
      child {node [nil] {}}
      child {node [nil] {}}
    }
    child {node [nil] {}}
  }
  child { node [wv] {39}
    child { node [wv] {36}
          child {node [nil] {}}
          child {node [nil] {}}
          }
    child { node [wv] {49}
          child {node [nil] {}}
          child {node [nil] {}}
      }
    }
};
\end{tikzpicture}
\end{figure}
\end{frame}


\begin{frame}[noframenumbering,plain,fragile] {\secname : \subsecname}

\begin{figure}[!h]
  \centering
  \caption{Inserir 27}

\begin{tikzpicture}[->,>=stealth',
  % level/.style={sibling distance = 8cm/#1, level distance = 1.5cm},
  level 1/.style={sibling distance=15em, level distance = 3em},
  level 2/.style={sibling distance=8em, level distance = 3em},
  level 3/.style={sibling distance=4em, level distance = 3em},
  level 4/.style={sibling distance=2em, level distance = 3em},
  level 5/.style={sibling distance=1em, level distance = 3em}]
\node [wv] (r){16}
child { node [wv] {6}
  child { node [wv] {5}
    child {node [nil] {}}
    child {node [nil] {}}
  }
  child { node [wv] {9}
    child {node [nil] {}}
    child {node [nil] {}}
  }
}
child { node [wv] {29}
  child { node [gv,label=above:-2] {28}
    child { node [gv,label=above:+1] {22}
      child {node [nil] {}}
      child { node [gv,label=above:0] {27}
        child {node [nil] {}}
        child {node [nil] {}}
      }
    }
    child {node [nil] {}}
  }
  child { node [wv] {39}
    child { node [wv] {36}
          child {node [nil] {}}
          child {node [nil] {}}
          }
    child { node [wv] {49}
          child {node [nil] {}}
          child {node [nil] {}}
      }
    }
};
\end{tikzpicture}
\end{figure}
\end{frame}


\begin{frame}[noframenumbering,plain,fragile] {\secname : \subsecname}

\begin{figure}[!h]
  \centering
  \caption{RSE 22 AR 27 + RSD 28 AR 27}

\begin{tikzpicture}[->,>=stealth',
  % level/.style={sibling distance = 8cm/#1, level distance = 1.5cm},
  level 1/.style={sibling distance=15em, level distance = 3em},
  level 2/.style={sibling distance=8em, level distance = 3em},
  level 3/.style={sibling distance=4em, level distance = 3em},
  level 4/.style={sibling distance=2em, level distance = 3em},
  level 5/.style={sibling distance=1em, level distance = 3em}]
\node [wv] (r){16}
child { node [wv] {6}
  child { node [wv] {5}
    child {node [nil] {}}
    child {node [nil] {}}
  }
  child { node [wv] {9}
    child {node [nil] {}}
    child {node [nil] {}}
  }
}
child { node [wv] {29}
  child { node [wv] {27}
    child { node [wv] {22}
      child {node [nil] {}}
      child {node [nil] {}}
    }
    child { node [wv] {28}
      child {node [nil] {}}
      child {node [nil] {}}
    }
    }
  child { node [wv] {39}
    child { node [wv] {36}
          child {node [nil] {}}
          child {node [nil] {}}
          }
    child { node [wv] {49}
          child {node [nil] {}}
          child {node [nil] {}}
      }
    }
};
\end{tikzpicture}
\end{figure}
\end{frame}


\begin{frame}[noframenumbering,plain,fragile] {\secname : \subsecname}

\begin{figure}[!h]
  \centering
  \caption{Inserir 37}

\begin{tikzpicture}[->,>=stealth',
  % level/.style={sibling distance = 8cm/#1, level distance = 1.5cm},
  level 1/.style={sibling distance=15em, level distance = 3em},
  level 2/.style={sibling distance=8em, level distance = 3em},
  level 3/.style={sibling distance=4em, level distance = 3em},
  level 4/.style={sibling distance=2em, level distance = 3em},
  level 5/.style={sibling distance=1em, level distance = 3em}]
\node [gv,label=above:+2] (r){16}
child { node [wv] {6}
  child { node [wv] {5}
    child {node [nil] {}}
    child {node [nil] {}}
  }
  child { node [wv] {9}
    child {node [nil] {}}
    child {node [nil] {}}
  }
}
child { node [gv,label=above:+1] {29}
  child { node [wv] {27}
      child { node [wv] {22}
        child {node [nil] {}}
        child {node [nil] {}}
      }
      child { node [wv] {28}
        child {node [nil] {}}
        child {node [nil] {}}
      }
    }
  child { node [gv,label=above:-1] {39}
    child { node [wv,label=above:+1] {36}
          child {node [nil] {}}
          child { node [wv] {37}
            child {node [nil] {}}
            child {node [nil] {}}
          }
        }
    child { node [wv] {49}
          child {node [nil] {}}
          child {node [nil] {}}
      }
    }
};
\end{tikzpicture}
\end{figure}
\end{frame}


\begin{frame}[noframenumbering,plain,fragile] {\secname : \subsecname}

\begin{figure}[!h]
  \centering
  \caption{RSE 16 AR 29}

\begin{tikzpicture}[->,>=stealth',
  % level/.style={sibling distance = 8cm/#1, level distance = 1.5cm},
  level 1/.style={sibling distance=15em, level distance = 3em},
  level 2/.style={sibling distance=8em, level distance = 3em},
  level 3/.style={sibling distance=4em, level distance = 3em},
  level 4/.style={sibling distance=2em, level distance = 3em},
  level 5/.style={sibling distance=1em, level distance = 3em}]
\node [wv] (r){29}
child { node [wv] {16}
  child { node [wv] {6}
    child { node [wv] {5}
      child {node [nil] {}}
      child {node [nil] {}}
    }
    child { node [wv] {9}
      child {node [nil] {}}
      child {node [nil] {}}
    }
  }
  child { node [wv] {27}
      child { node [wv] {22}
        child {node [nil] {}}
        child {node [nil] {}}
      }
      child { node [wv] {28}
        child {node [nil] {}}
        child {node [nil] {}}
      }
    }
}
child { node [wv] {39}
    child { node [wv] {36}
          child {node [nil] {}}
          child { node [wv] {37}
            child {node [nil] {}}
            child {node [nil] {}}
          }
        }
    child { node [wv] {49}
          child {node [nil] {}}
          child {node [nil] {}}
      }
};
\end{tikzpicture}
\end{figure}
\end{frame}


\begin{frame}[noframenumbering,plain,fragile] {\secname : \subsecname}

\begin{figure}[!h]
  \centering
  \caption{Inserir 1}

  \begin{tikzpicture}[->,>=stealth',
    % level/.style={sibling distance = 8cm/#1, level distance = 1.5cm},
    level 1/.style={sibling distance=19em},
    level 2/.style={sibling distance=9em},
    level 3/.style={sibling distance=5em},
    level 4/.style={sibling distance=2.5em}]
  \node [wv] (r){29}
  child { node [wv] {16}
    child { node [wv] {6}
      child { node [wv] {5}
        child { node [wv] {1}
          child {node [nil] {}}
          child {node [nil] {}}
        }
        child {node [nil] {}}
      }
      child { node [wv] {9}
        child {node [nil] {}}
        child {node [nil] {}}
      }
    }
    child { node [wv] {27}
        child { node [wv] {22}
          child {node [nil] {}}
          child {node [nil] {}}
        }
        child { node [wv] {28}
          child {node [nil] {}}
          child {node [nil] {}}
        }
      }
  }
  child { node [wv] {39}
      child { node [wv] {36}
            child {node [nil] {}}
            child { node [wv] {37}
              child {node [nil] {}}
              child {node [nil] {}}
            }
          }
      child { node [wv] {49}
            child {node [nil] {}}
            child {node [nil] {}}
        }
  };
  \end{tikzpicture}
\end{figure}
\end{frame}


\begin{frame}[noframenumbering,plain,fragile] {\secname : \subsecname}

\begin{figure}[!h]
  \centering
  \caption{Inserir 4}

  \begin{tikzpicture}[->,>=stealth',
    % level/.style={sibling distance = 8cm/#1, level distance = 1.5cm},
    level 1/.style={sibling distance=19em},
    level 2/.style={sibling distance=9em},
    level 3/.style={sibling distance=5em},
    level 4/.style={sibling distance=2.5em}]
  \node [wv] (r){29}
  child { node [wv] {16}
    child { node [wv] {6}
      child { node [gv,label=above:-2] {5}
        child { node [gv,label=above:+1] {1}
          child {node [nil] {}}
          child { node [gv,label=above:0] {4}
            child {node [nil] {}}
            child {node [nil] {}}
          }
        }
        child {node [nil] {}}
      }
      child { node [wv] {9}
        child {node [nil] {}}
        child {node [nil] {}}
      }
    }
    child { node [wv] {27}
        child { node [wv] {22}
          child {node [nil] {}}
          child {node [nil] {}}
        }
        child { node [wv] {28}
          child {node [nil] {}}
          child {node [nil] {}}
        }
      }
  }
  child { node [wv] {39}
      child { node [wv] {36}
            child {node [nil] {}}
            child { node [wv] {37}
              child {node [nil] {}}
              child {node [nil] {}}
            }
          }
      child { node [wv] {49}
            child {node [nil] {}}
            child {node [nil] {}}
        }
  };
  \end{tikzpicture}
\end{figure}
\end{frame}


\begin{frame}[noframenumbering,plain,fragile] {\secname : \subsecname}

\begin{figure}[!h]
  \centering
  \caption{RSE 1 AR 4 + RSD 5 AR 4}

  \begin{tikzpicture}[->,>=stealth',
    % level/.style={sibling distance = 8cm/#1, level distance = 1.5cm},
    level 1/.style={sibling distance=19em},
    level 2/.style={sibling distance=9em},
    level 3/.style={sibling distance=5em},
    level 4/.style={sibling distance=2.5em}]
  \node [wv] (r){29}
  child { node [wv] {16}
    child { node [wv] {6}
      child { node [wv] {4}
        child { node [wv] {1}
          child {node [nil] {}}
          child {node [nil] {}}
        }
      child { node [wv] {5}
          child {node [nil] {}}
          child {node [nil] {}}
        }
      }
      child { node [wv] {9}
        child {node [nil] {}}
        child {node [nil] {}}
      }
    }
    child { node [wv] {27}
        child { node [wv] {22}
          child {node [nil] {}}
          child {node [nil] {}}
        }
        child { node [wv] {28}
          child {node [nil] {}}
          child {node [nil] {}}
        }
      }
  }
  child { node [wv] {39}
      child { node [wv] {36}
            child {node [nil] {}}
            child { node [wv] {37}
              child {node [nil] {}}
              child {node [nil] {}}
            }
          }
      child { node [wv] {49}
            child {node [nil] {}}
            child {node [nil] {}}
        }
  };
  \end{tikzpicture}
\end{figure}
\end{frame}


\begin{frame}[noframenumbering,plain,fragile] {\secname : \subsecname}

\begin{figure}[!h]
  \centering
  \caption{Inserir 48}

  \begin{tikzpicture}[->,>=stealth',
    % level/.style={sibling distance = 8cm/#1, level distance = 1.5cm},
    level 1/.style={sibling distance=19em},
    level 2/.style={sibling distance=9em},
    level 3/.style={sibling distance=5em},
    level 4/.style={sibling distance=2.5em}]
  \node [wv] (r){29}
  child { node [wv] {16}
    child { node [wv] {6}
      child { node [wv] {4}
        child { node [wv] {1}
          child {node [nil] {}}
          child {node [nil] {}}
        }
      child { node [wv] {5}
          child {node [nil] {}}
          child {node [nil] {}}
        }
      }
      child { node [wv] {9}
        child {node [nil] {}}
        child {node [nil] {}}
      }
    }
    child { node [wv] {27}
        child { node [wv] {22}
          child {node [nil] {}}
          child {node [nil] {}}
        }
        child { node [wv] {28}
          child {node [nil] {}}
          child {node [nil] {}}
        }
      }
  }
  child { node [wv] {39}
      child { node [wv] {36}
            child {node [nil] {}}
            child { node [wv] {37}
              child {node [nil] {}}
              child {node [nil] {}}
            }
          }
      child { node [wv] {49}
            child { node [wv] {48}
                child {node [nil] {}}
                child {node [nil] {}}
            }
            child {node [nil] {}}
        }
  };
  \end{tikzpicture}
\end{figure}
\end{frame}


% \begin{frame}[noframenumbering,plain,fragile] {\secname : \subsecname}


% \begin{figure}[!h]
 
%   \centering
%   \caption{RSD 50 AR 44 + RSE 24 AR 44}
% \begin{tikzpicture}[->,>=stealth',
%   % level/.style={sibling distance = 8cm/#1, level distance = 1.5cm},
%   level 1/.style={sibling distance=19em},
%   level 2/.style={sibling distance=9em},
%   level 3/.style={sibling distance=5em},
%   level 4/.style={sibling distance=2.5em}]
% \node [wv] (r){44}
% child { node [wv] {24}
%                 child {node [nil] {}}
%                 child {node [nil] {}}
%               }
% child { node [wv] {50}
%               child {node [nil] {}}
%               child {node [nil] {}}
%             };
% \end{tikzpicture}
% \end{figure}
% \end{frame}


% \begin{frame}[noframenumbering,plain,fragile] {\secname : \subsecname}

% \begin{figure}[!h]
%   \centering
%   \caption{Inserir 15}

% \begin{tikzpicture}[->,>=stealth',
%   % level/.style={sibling distance = 8cm/#1, level distance = 1.5cm},
%   level 1/.style={sibling distance=19em},
%   level 2/.style={sibling distance=9em},
%   level 3/.style={sibling distance=5em},
%   level 4/.style={sibling distance=2.5em}]
% \node [wv] (r){44}
% child { node [wv] {24}
%   child { node [wv] {15}
%     child {node [nil] {}}
%     child {node [nil] {}}
%   }
%   child {node [nil] {}}
% }
% child { node [wv] {50}
%   child {node [nil] {}}
%   child {node [nil] {}}
% };
% \end{tikzpicture}
% \end{figure}
% \end{frame}


% \begin{frame}[noframenumbering,plain,fragile] {\secname : \subsecname}

% \begin{figure}[!h]
%   \centering
%   \caption{Inserir 43}

% \begin{tikzpicture}[->,>=stealth',
%   % level/.style={sibling distance = 8cm/#1, level distance = 1.5cm},
%   level 1/.style={sibling distance=19em},
%   level 2/.style={sibling distance=9em},
%   level 3/.style={sibling distance=5em},
%   level 4/.style={sibling distance=2.5em},
%   level 5/.style={sibling distance=2em}]
% \node [wv] (r){44}
% child { node [wv] {24}
%   child { node [wv] {15}
%     child {node [nil] {}}
%     child {node [nil] {}}
%   }
%   child { node [wv] {43}
%     child {node [nil] {}}
%     child {node [nil] {}}
%   }
% }
% child { node [wv] {50}
%   child {node [nil] {}}
%   child {node [nil] {}}
% };
% \end{tikzpicture}
% \end{figure}
% \end{frame}


% \begin{frame}[noframenumbering,plain,fragile] {\secname : \subsecname}

% \begin{figure}[!h]
%   \centering
%   \caption{Inserir 35}

% \begin{tikzpicture}[->,>=stealth',
%   % level/.style={sibling distance = 8cm/#1, level distance = 1.5cm},
%   level 1/.style={sibling distance=19em},
%   level 2/.style={sibling distance=9em},
%   level 3/.style={sibling distance=5em},
%   level 4/.style={sibling distance=2.5em},
%   level 5/.style={sibling distance=2em}]
% \node [wv,label=above: -2] (r){44}
% child { node [wv,label=above: +1] {24}
%   child { node [wv] {15}
%     child {node [nil] {}}
%     child {node [nil] {}}
%   }
%   child { node [wv,label=above: 0] {43}
%     child { node [wv] {35}
%       child {node [nil] {}}
%       child {node [nil] {}}
%     }
%     child {node [nil] {}}
%   }
% }
% child { node [wv] {50}
%   child {node [nil] {}}
%   child {node [nil] {}}
% };
% \end{tikzpicture}
% \end{figure}
% \end{frame}


% \begin{frame}[noframenumbering,plain,fragile] {\secname : \subsecname}

% \begin{figure}[!h]
%   \centering
%   \caption{RSE 24 AR 43 + RSD 44 AR 43}

% \begin{tikzpicture}[->,>=stealth',
%   % level/.style={sibling distance = 8cm/#1, level distance = 1.5cm},
%   level 1/.style={sibling distance=19em},
%   level 2/.style={sibling distance=9em},
%   level 3/.style={sibling distance=5em},
%   level 4/.style={sibling distance=2.5em},
%   level 5/.style={sibling distance=2em}]
% \node [wv] (r){43}
% child { node [wv] {24}
%   child { node [wv] {15}
%     child {node [nil] {}}
%     child {node [nil] {}}
%   }
%   child { node [wv] {35}
%       child {node [nil] {}}
%       child {node [nil] {}}
%     }
% }
% child { node [wv] {44}
%   child {node [nil] {}}
%   child { node [wv] {50}
%     child {node [nil] {}}
%     child {node [nil] {}}
%   }
% };
% \end{tikzpicture}

% \end{figure}
% \end{frame}


% \begin{frame}[noframenumbering,plain,fragile] {\secname : \subsecname}

% \begin{figure}[!h]
%   \centering
%   \caption{Inserir 40}

% \begin{tikzpicture}[->,>=stealth',
%   % level/.style={sibling distance = 8cm/#1, level distance = 1.5cm},
%   level 1/.style={sibling distance=19em},
%   level 2/.style={sibling distance=9em},
%   level 3/.style={sibling distance=5em},
%   level 4/.style={sibling distance=2.5em},
%   level 5/.style={sibling distance=2em}]
% \node [wv] (r){43}
% child { node [wv] {24}
%   child { node [wv] {15}
%     child {node [nil] {}}
%     child {node [nil] {}}
%   }
%   child { node [wv] {35}
%     child {node [nil] {}}
%     child { node [wv] {40}
%       child {node [nil] {}}
%       child {node [nil] {}}
%     }
%   }
% }
% child { node [wv] {44}
%   child {node [nil] {}}
%   child { node [wv] {50}
%     child {node [nil] {}}
%     child {node [nil] {}}
%   }
% };
% \end{tikzpicture}
% \end{figure}
% \end{frame}


% \begin{frame}[noframenumbering,plain,fragile] {\secname : \subsecname}

% \begin{figure}[!h]
%   \centering
%   \caption{Inserir 14}

% \begin{tikzpicture}[->,>=stealth',
%   % level/.style={sibling distance = 8cm/#1, level distance = 1.5cm},
%   level 1/.style={sibling distance=19em},
%   level 2/.style={sibling distance=9em},
%   level 3/.style={sibling distance=5em},
%   level 4/.style={sibling distance=2.5em},
%   level 5/.style={sibling distance=2em}]
% \node [wv] (r){43}
% child { node [wv] {24}
%   child { node [wv] {15}
%     child { node [wv] {14}
%       child {node [nil] {}}
%       child {node [nil] {}}
%     }
%     child {node [nil] {}}
%   }
%   child { node [wv] {35}
%     child {node [nil] {}}
%     child { node [wv] {40}
%       child {node [nil] {}}
%       child {node [nil] {}}
%     }
%   }
% }
% child { node [wv] {44}
%   child {node [nil] {}}
%   child { node [wv] {50}
%     child {node [nil] {}}
%     child {node [nil] {}}
%   }
% };
% \end{tikzpicture}

% \end{figure}
% \end{frame}


% \begin{frame}[noframenumbering,plain,fragile] {\secname : \subsecname}

% \begin{figure}[!h]
%   \centering
%   \caption{Inserir 48}

% \begin{tikzpicture}[->,>=stealth',
%   % level/.style={sibling distance = 8cm/#1, level distance = 1.5cm},
%   level 1/.style={sibling distance=19em},
%   level 2/.style={sibling distance=9em},
%   level 3/.style={sibling distance=5em},
%   level 4/.style={sibling distance=2.5em},
%   level 5/.style={sibling distance=2em}]
% \node [wv] (r){43}
% child { node [wv] {24}
%   child { node [wv] {15}
%     child { node [wv] {14}
%       child {node [nil] {}}
%       child {node [nil] {}}
%     }
%     child {node [nil] {}}
%   }
%   child { node [wv] {35}
%     child {node [nil] {}}
%     child { node [wv] {40}
%       child {node [nil] {}}
%       child {node [nil] {}}
%     }
%   }
% }
% child { node [wv,label=above: +2] {44}
%   child {node [nil] {}}
%   child { node [wv,label=above: -1] {50}
%     child { node [wv,label=above: 0] {48}
%       child {node [nil] {}}
%       child {node [nil] {}}
%     }
%     child {node [nil] {}}
%   }
% };
% \end{tikzpicture}
% \end{figure}
% \end{frame}


% \begin{frame}[noframenumbering,plain,fragile] {\secname : \subsecname}

% \begin{figure}[!h]
%   \centering

%   \caption{RSD 50 AR 48 + RSE 40 AR 48}

% \begin{tikzpicture}[->,>=stealth',
%   % level/.style={sibling distance = 8cm/#1, level distance = 1.5cm},
%   level 1/.style={sibling distance=19em},
%   level 2/.style={sibling distance=9em},
%   level 3/.style={sibling distance=5em},
%   level 4/.style={sibling distance=2.5em},
%   level 5/.style={sibling distance=2em}]
% \node [wv] (r){43}
% child { node [wv] {24}
%   child { node [wv] {15}
%     child { node [wv] {14}
%       child {node [nil] {}}
%       child {node [nil] {}}
%     }
%     child {node [nil] {}}
%   }
%   child { node [wv] {35}
%     child {node [nil] {}}
%     child { node [wv] {40}
%       child {node [nil] {}}
%       child {node [nil] {}}
%     }
%   }
% }
% child { node [wv] {48}
%   child { node [wv] {44}
%     child {node [nil] {}}
%     child {node [nil] {}}
%   }
%   child { node [wv] {50}
%     child {node [nil] {}}
%     child {node [nil] {}}
%   }
% };
% \end{tikzpicture}

% \end{figure}
% \end{frame}


% \begin{frame}[noframenumbering,plain,fragile] {\secname : \subsecname}

% \begin{figure}[!h]
%   \centering
%   \caption{Inserir 45}

% \begin{tikzpicture}[->,>=stealth',
%   % level/.style={sibling distance = 8cm/#1, level distance = 1.5cm},
%   level 1/.style={sibling distance=19em},
%   level 2/.style={sibling distance=9em},
%   level 3/.style={sibling distance=5em},
%   level 4/.style={sibling distance=2.5em},
%   level 5/.style={sibling distance=2em}]
% \node [wv] (r){43}
% child { node [wv] {24}
%   child { node [wv] {15}
%     child { node [wv] {14}
%       child {node [nil] {}}
%       child {node [nil] {}}
%     }
%     child {node [nil] {}}
%   }
%   child { node [wv] {35}
%     child {node [nil] {}}
%     child { node [wv] {40}
%       child {node [nil] {}}
%       child {node [nil] {}}
%     }
%   }
% }
% child { node [wv] {48}
%   child { node [wv] {44}
%     child {node [nil] {}}
%     child { node [wv] {45}
%       child {node [nil] {}}
%       child {node [nil] {}}
%     }
%   }
%   child { node [wv] {50}
%     child {node [nil] {}}
%     child {node [nil] {}}
%   }
% };
% \end{tikzpicture}
% \end{figure}
% \end{frame}


% \begin{frame}[noframenumbering,plain,fragile] {\secname : \subsecname}

% \begin{figure}[!h]
%   \centering
%   \caption{Inserir 36}

% \begin{tikzpicture}[->,>=stealth',
%   % level/.style={sibling distance = 8cm/#1, level distance = 1.5cm},
%   level 1/.style={sibling distance=19em},
%   level 2/.style={sibling distance=9em},
%   level 3/.style={sibling distance=5em},
%   level 4/.style={sibling distance=2.5em},
%   level 5/.style={sibling distance=2em}]
% \node [wv] (r){43}
% child { node [wv] {24}
%   child { node [wv] {15}
%     child { node [wv] {14}
%       child {node [nil] {}}
%       child {node [nil] {}}
%     }
%     child {node [nil] {}}
%   }
%   child { node [wv,label=above: +2] {35}
%     child {node [nil] {}}
%     child { node [wv,label=above: -1] {40}
%       child { node [wv,label=above: 0] {36}
%         child {node [nil] {}}
%         child {node [nil] {}}
%       }
%       child {node [nil] {}}
%     }
%   }
% }
% child { node [wv] {48}
%   child { node [wv] {44}
%     child {node [nil] {}}
%     child { node [wv] {45}
%       child {node [nil] {}}
%       child {node [nil] {}}
%     }
%   }
%   child { node [wv] {50}
%     child {node [nil] {}}
%     child {node [nil] {}}
%   }
% };
% \end{tikzpicture}

% \end{figure}
% \end{frame}


% \begin{frame}[noframenumbering,plain,fragile] {\secname : \subsecname}

% \begin{figure}[!h]
%   \centering
% \caption{RSD 40 AR 36 + RSE 35 AR 36}

% \begin{tikzpicture}[->,>=stealth',
%   % level/.style={sibling distance = 8cm/#1, level distance = 1.5cm},
%   level 1/.style={sibling distance=19em},
%   level 2/.style={sibling distance=9em},
%   level 3/.style={sibling distance=5em},
%   level 4/.style={sibling distance=2.5em},
%   level 5/.style={sibling distance=2em}]
% \node [wv] (r){43}
% child { node [wv] {24}
%   child { node [wv] {15}
%     child { node [wv] {14}
%       child {node [nil] {}}
%       child {node [nil] {}}
%     }
%     child {node [nil] {}}
%   }
%   child { node [wv] {36}
%     child { node [wv] {35}
%       child {node [nil] {}}
%       child {node [nil] {}}
%     }
%     child { node [wv] {40}
%       child {node [nil] {}}
%       child {node [nil] {}}
%     }
%   }
% }
% child { node [wv] {48}
%   child { node [wv] {44}
%     child {node [nil] {}}
%     child { node [wv] {45}
%       child {node [nil] {}}
%       child {node [nil] {}}
%     }
%   }
%   child { node [wv] {50}
%     child {node [nil] {}}
%     child {node [nil] {}}
%   }
% };
% \end{tikzpicture}
% \end{figure}
% \end{frame}


% \begin{frame}[noframenumbering,plain,fragile] {\secname : \subsecname}

% \begin{figure}[!h]
%   \centering
%   \caption{Inserir 21}
%   \begin{tikzpicture}[->,>=stealth',
%   % level/.style={sibling distance = 8cm/#1, level distance = 1.5cm},
%   level 1/.style={sibling distance=19em},
%   level 2/.style={sibling distance=9em},
%   level 3/.style={sibling distance=5em},
%   level 4/.style={sibling distance=2.5em},
%   level 5/.style={sibling distance=2em}]
% \node [wv] (r){43}
% child { node [wv] {24}
%   child { node [wv] {15}
%     child { node [wv] {14}
%       child {node [nil] {}}
%       child {node [nil] {}}
%     }
%     child { node [wv] {21}
%       child {node [nil] {}}
%       child {node [nil] {}}
%     }
%   }
%   child { node [wv] {36}
%     child { node [wv] {35}
%       child {node [nil] {}}
%       child {node [nil] {}}
%     }
%     child { node [wv] {40}
%       child {node [nil] {}}
%       child {node [nil] {}}
%     }
%   }
% }
% child { node [wv] {48}
%   child { node [wv] {44}
%     child {node [nil] {}}
%     child { node [wv] {45}
%       child {node [nil] {}}
%       child {node [nil] {}}
%     }
%   }
%   child { node [wv] {50}
%     child {node [nil] {}}
%     child {node [nil] {}}
%   }
% };
% \end{tikzpicture}

% \end{figure}
% \end{frame}


% \begin{frame}[noframenumbering,plain,fragile] {\secname : \subsecname}

% \begin{figure}[!h]
%   \centering
%   \caption{Inserir 37}
%   \begin{tikzpicture}[->,>=stealth',
%   % level/.style={sibling distance = 8cm/#1, level distance = 1.5cm},
%   level 1/.style={sibling distance=19em},
%   level 2/.style={sibling distance=9em},
%   level 3/.style={sibling distance=5em},
%   level 4/.style={sibling distance=2.5em},
%   level 5/.style={sibling distance=2em}]
% \node [wv] (r){43}
% child { node [wv] {24}
%   child { node [wv] {15}
%     child { node [wv] {14}
%       child {node [nil] {}}
%       child {node [nil] {}}
%     }
%     child { node [wv] {21}
%       child {node [nil] {}}
%       child {node [nil] {}}
%     }
%   }
%   child { node [wv] {36}
%     child { node [wv] {35}
%       child {node [nil] {}}
%       child {node [nil] {}}
%     }
%     child { node [wv] {40}
%       child { node [wv] {37}
%         child {node [nil] {}}
%         child {node [nil] {}}
%       }
%       child {node [nil] {}}
%     }
%   }
% }
% child { node [wv] {48}
%   child { node [wv] {44}
%     child {node [nil] {}}
%     child { node [wv] {45}
%       child {node [nil] {}}
%       child {node [nil] {}}
%     }
%   }
%   child { node [wv] {50}
%     child {node [nil] {}}
%     child {node [nil] {}}
%   }
% };
% \end{tikzpicture}

% \end{figure}
% \end{frame}


% \begin{frame}[noframenumbering,plain,fragile] {\secname : \subsecname}

% \begin{figure}[!h]
%   \centering
%   \caption{Inserir 32}
%   \begin{tikzpicture}[->,>=stealth',
%   % level/.style={sibling distance = 8cm/#1, level distance = 1.5cm},
%   level 1/.style={sibling distance=19em},
%   level 2/.style={sibling distance=9em},
%   level 3/.style={sibling distance=5em},
%   level 4/.style={sibling distance=2.5em},
%   level 5/.style={sibling distance=2em}]
% \node [wv] (r){43}
% child { node [wv] {24}
%   child { node [wv] {15}
%     child { node [wv] {14}
%       child {node [nil] {}}
%       child {node [nil] {}}
%     }
%     child { node [wv] {21}
%       child {node [nil] {}}
%       child {node [nil] {}}
%     }
%   }
%   child { node [wv] {36}
%     child { node [wv] {35}
%       child { node [wv] {32}
%         child {node [nil] {}}
%         child {node [nil] {}}
%       }
%       child {node [nil] {}}
%     }
%     child { node [wv] {40}
%       child { node [wv] {37}
%         child {node [nil] {}}
%         child {node [nil] {}}
%       }
%       child {node [nil] {}}
%     }
%   }
% }
% child { node [wv] {48}
%   child { node [wv] {44}
%     child {node [nil] {}}
%     child { node [wv] {45}
%       child {node [nil] {}}
%       child {node [nil] {}}
%     }
%   }
%   child { node [wv] {50}
%     child {node [nil] {}}
%     child {node [nil] {}}
%   }
% };
% \end{tikzpicture}

% \end{figure}
% \end{frame}


% \begin{frame}[noframenumbering,plain,fragile] {\secname : \subsecname}

% \begin{figure}[!h]
%   \centering
%   \caption{Inserir 31}
%   \begin{tikzpicture}[->,>=stealth',
%   % level/.style={sibling distance = 8cm/#1, level distance = 1.5cm},
%   level 1/.style={sibling distance=19em},
%   level 2/.style={sibling distance=9em},
%   level 3/.style={sibling distance=5em},
%   level 4/.style={sibling distance=2.5em},
%   level 5/.style={sibling distance=2em}]
% \node [wv] (r){43}
% child { node [wv] {24}
%   child { node [wv] {15}
%     child { node [wv] {14}
%       child {node [nil] {}}
%       child {node [nil] {}}
%     }
%     child { node [wv] {21}
%       child {node [nil] {}}
%       child {node [nil] {}}
%     }
%   }
%   child { node [wv] {36}
%     child { node [wv,label=above: -2] {35}
%       child { node [wv,label=above: -1] {32}
%         child { node [wv,label=above: 0] {31}
%           child {node [nil] {}}
%           child {node [nil] {}}
%         }
%         child {node [nil] {}}
%       }
%       child {node [nil] {}}
%     }
%     child { node [wv] {40}
%       child { node [wv] {37}
%         child {node [nil] {}}
%         child {node [nil] {}}
%       }
%       child {node [nil] {}}
%     }
%   }
% }
% child { node [wv] {48}
%   child { node [wv] {44}
%     child {node [nil] {}}
%     child { node [wv] {45}
%       child {node [nil] {}}
%       child {node [nil] {}}
%     }
%   }
%   child { node [wv] {50}
%     child {node [nil] {}}
%     child {node [nil] {}}
%   }
% };
% \end{tikzpicture}

% \end{figure}
% \end{frame}


% \begin{frame}[noframenumbering,plain,fragile] {\secname : \subsecname}

% \begin{figure}[!h]
%   \centering
%   \caption{RSE 36 AR 32}
%   \begin{tikzpicture}[->,>=stealth',
%   % level/.style={sibling distance = 8cm/#1, level distance = 1.5cm},
%   level 1/.style={sibling distance=19em},
%   level 2/.style={sibling distance=9em},
%   level 3/.style={sibling distance=5em},
%   level 4/.style={sibling distance=2.5em},
%   level 5/.style={sibling distance=2em}]
% \node [wv] (r){43}
% child { node [wv] {24}
%   child { node [wv] {15}
%     child { node [wv] {14}
%       child {node [nil] {}}
%       child {node [nil] {}}
%     }
%     child { node [wv] {21}
%       child {node [nil] {}}
%       child {node [nil] {}}
%     }
%   }
%   child { node [wv] {36}
%     child { node [wv] {32}
%       child { node [wv] {31}
%         child {node [nil] {}}
%         child {node [nil] {}}
%       }
%       child { node [wv] {35}
%         child {node [nil] {}}
%         child {node [nil] {}}
%       }
%     }
%     child { node [wv] {40}
%       child { node [wv] {37}
%         child {node [nil] {}}
%         child {node [nil] {}}
%       }
%       child {node [nil] {}}
%     }
%   }
% }
% child { node [wv] {48}
%   child { node [wv] {44}
%     child {node [nil] {}}
%     child { node [wv] {45}
%       child {node [nil] {}}
%       child {node [nil] {}}
%     }
%   }
%   child { node [wv] {50}
%     child {node [nil] {}}
%     child {node [nil] {}}
%   }
% };
% \end{tikzpicture}

% \end{figure}
% \end{frame}



