\title{Rubro Negra}
\date{\today}
\frame{\titlepage}
% \chapter{Árvores Rubro-Negras}
\begin{frame}[fragile] 
  \frametitle{Árvore Rubro-Negra}
  \begin{theorem}
    Uma árvore rubro-negra é uma árvore binária de busca balanceada na 
    qual cada nó interno tem dois filhos. Cada nó interno tem uma cor, 
    tal que:
    
    \begin{itemize}
    \item Árvore balanceada
    \item Trata cada nó como preta ou vermelhos
    \item Cada vez que adicionamos um nó, conferimos a árvore com as regras da ARN.
    \item Dependendo da violação das regras, faremos uma inversão de cores ou rotação para corrigir os erros na árvore
    \end{itemize}
    \end{theorem}
    
    Precisamos adaptar as operações de inserção e deleção de forma que as 
    propriedades rubro negras sejam mantidas.
\end{frame}
    

%%%%%%%%%%%%
%% FRAME %%%
%%%%%%%%%%%%

\begin{frame}[fragile]{Árvore Rubro-Negra - Regras}
  \begin{theorem}
  
  \begin{itemize}
  \item[0.] Cada nó é preto ou vermelho 
  \item[1.] O nó raiz é sempre preto.
  \item[2.] Todos os nós folha (Null) são pretos.
  \item[3.] Novas inserções são sempre vermelhas.  
  \item[4.] Todos os caminhos percorridos na árvore contém o mesmo número de nós pretos (\textbf{\emph{altura-negra}}).
  \item[5.] Nenhum caminho pode ter dois nós vermelhos consecultivos. 
  \item[6.] Se um nó é vermelho, então ambos os filhos são pretos. 
  \item[7.] Os ponteiros dos nós folha (representados como quadrados pretos) continuam sendo "nulos" e não carregam nenhum valor.
  \end{itemize}
  \end{theorem}
  
\end{frame}


\begin{frame}[fragile]

    \begin{figure}[!h]
      \centering
      \caption{Típica árvore rubro-negra} 
    % \begin{preview}
    \begin{tikzpicture}[->,>=stealth',
      % level/.style={sibling distance = 8cm/#1, level distance = 1.5cm},
      level 1/.style={sibling distance=0.4\textwidth, level distance = 0.8cm},
      level 2/.style={sibling distance=0.2\textwidth, level distance = 0.8cm},
      level 3/.style={sibling distance=0.1\textwidth, level distance = 0.8cm},
      level 4/.style={sibling distance=0.05\textwidth, level distance = 0.8cm}]
    \node [bbv] (r){6}
    child {node [brv] {3}
      child {node [bbv] {1}
        child {node [nil] {}}
        child {node [nil] {}}
      }
      child {node [bbv] {5}
        child {node [nil] {}}
        child {node [nil] {}}
      }
    }
    child {node [brv] {8}
      child {node [bbv] {7}
        child {node [nil] {}}
        child {node [nil] {}}
      }
      child {node [bbv] {9}
        child {node [nil] {}}
        child {node [brv] {10}
          child {node [nil] {}}
          child {node [nil] {}}
        }
      }
    }
    ;
  \end{tikzpicture}
  \end{figure}
\end{frame}

%%%%%%%%%%%%
%% FRAME %%%
%%%%%%%%%%%%
\begin{frame}[fragile]
  \frametitle{Regras pra corrigir a árvore}

  \begin{itemize}
    \item \textbf{Tio Preto} - Rotaciona (TPR)
    \item \textbf{Tio Vermelho} - inverte cor (TVI)
  \end{itemize}

\begin{figure}
\begin{minipage}[t]{0.48\linewidth}
\centering
\caption{Resultado da TPR}
\begin{tikzpicture}[->,>=stealth',
  level 1/.style={sibling distance=6em, level distance = 3em},
  level 2/.style={sibling distance=2em, level distance = 3em},
  level 3/.style={sibling distance=1em, level distance = 3em}
]
\node [bbv] (r){c}
    child [color=black] {node [brv] {a}}
    child [color=black] {node [brv] {c}}
;
\end{tikzpicture}
\end{minipage}\hfill
\begin{minipage}[t]{0.48\linewidth}
\caption{Resultado da TVI}
  \begin{tikzpicture}[->,>=stealth',
  level 1/.style={sibling distance=6em, level distance = 3em},
  level 2/.style={sibling distance=2em, level distance = 3em},
  level 3/.style={sibling distance=1em, level distance = 3em}
]
\node [brv] (r){c}
    child [color=black] {node [bbv] {a}}
    child [color=black] {node [bbv] {c}}
;
\end{tikzpicture}
\end{minipage}
\end{figure}
\end{frame}


%%%%%%%%%%%%`'
%% FRAME %%%
%%%%%%%%%%%%

\begin{frame}[fragile]{Caso 1 - O tio de \textbf{a}, (\textbf{d}), é vermelho}
TVC - tio-vermelho-colorir 
Então como a gente faz pra corrigir a quebra de invariante (coloração rubro-negra?) 

É o que vamos ver agora!
\begin{figure}
\begin{minipage}[t]{0.48\linewidth}
\centering
\caption{Antes}
\begin{tikzpicture}[->,>=stealth',
  level 1/.style={sibling distance=6em, level distance = 3em},
  level 2/.style={sibling distance=2em, level distance = 3em},
  level 3/.style={sibling distance=1em, level distance = 3em}
]
\node [bbv] (r){c}
child [color=black] { node [brv] {b}
        child [color=black] {node [bwv] {z}}
        child [color=black] { node [brv] {a}
          child [color=black] {node [bwv] {x}}
          child [color=black] {node [bwv] {y}}
        }
      }
child [color=black] { node [brv] {d}
        child [color=black] {node [bwv] {p}}
        child [color=black] {node [bwv] {q}}
};
\end{tikzpicture}
\end{minipage}\hfill
\begin{minipage}[t]{0.48\linewidth}
\caption{Depois}
  \begin{tikzpicture}[->,>=stealth',
  level 1/.style={sibling distance=6em, level distance = 3em},
  level 2/.style={sibling distance=2em, level distance = 3em},
  level 3/.style={sibling distance=1em, level distance = 3em}
]
\node [brv] (r){c}
child [color=black] { node [bbv] {b}
        child [color=black] {node [bwv] {z}}
        child [color=black] { node [brv] {a}
          child [color=black] {node [bwv] {x}}
          child [color=black] {node [bwv] {y}}
        }
      }
child [color=black] { node [bbv] {d}
        child [color=black] {node [bwv] {p}}
        child [color=black] {node [bwv] {q}}
};
\end{tikzpicture}
\end{minipage}
\end{figure}
\end{frame}

%%%%%%%%%%%%
%% FRAME %%%
%%%%%%%%%%%%

\begin{frame}[fragile]{Caso 2 - O tio de \textbf{a}, (\textbf{d}), é preto e \textbf{a} é filho direito}
TPR - tio-preto-rotaciona
% \pagebreak
\begin{itemize}
  \item Rotação à esquerda do pai de \textbf{a}, (\textbf{b}), ao redor de \textbf{a}
  \item Continua para o \textbf{Caso 3}
\end{itemize}

\begin{columns}[T] % align columns
  \begin{column}{.48\textwidth}
  \rule{\linewidth}{4pt}
  \begin{figure}[!h]
  \caption{Antes}
\begin{tikzpicture}[->,>=stealth',
  % level/.style={sibling distance = 20em, level distance = 3em},
  level 1/.style={sibling distance=6em, level distance = 3em},
  level 2/.style={sibling distance=3em, level distance = 3em},
  level 3/.style={sibling distance=2em, level distance = 3em}
]
\node [bbv] (r){c}
child [color=black] { node [brv] {b}
        child [color=black] {node [bwv] {x}}
        child [color=black] { node [brv] {a}
          child [color=black] {node [bwv] {y}}
          child [color=black] {node [bwv] {z}}
        }
      }
child [color=black] { node [bwv] {w}};
\end{tikzpicture}
\end{figure}
   
\end{column}%
\hfill%
\begin{column}{.48\textwidth}
\rule{\linewidth}{4pt}
      
    
\begin{figure}[!h]
  \centering
  % \centering
  \caption{Depois}
\begin{tikzpicture}[->,>=stealth',
   % level/.style={sibling distance = 20em, level distance = 3em},
   level 1/.style={sibling distance=6em, level distance = 3em},
  level 2/.style={sibling distance=3em, level distance = 3em},
  level 3/.style={sibling distance=2em, level distance = 3em},
  ] 
\node [bbv] (r){c}
child [color=black] { node [brv] {a}
    child [color=black] { node [brv] {b}
      child [color=black] {node [bwv] {x}}
      child [color=black] {node [bwv] {y}}
    }
    child [color=black] {node [bwv] {z}}
}
child [color=black] { node [bwv] {w}};
\end{tikzpicture}
\end{figure}
\end{column}%
\end{columns}
\end{frame}


%%%%%%%%%%%%
%% FRAME %%%
%%%%%%%%%%%%

\begin{frame}[fragile]{Caso 3 - O tio de \textbf{a}, (\textbf{d}), é preto e \textbf{a} é filho esquerdo }

\begin{itemize}
  \item Rotação à direita de \textbf{c}, ao redor de \textbf{b}
  \item Troque as cores entre o pai de \textbf{b}, (\textbf{a}) e seu novo irmão (\textbf{c}).
\end{itemize}
  

\begin{columns}[T] % align columns
  \begin{column}{.31\textwidth}
  \rule{\linewidth}{4pt}
  \begin{figure}[!h]
  \caption{Antes}
\begin{tikzpicture}[->,>=stealth',
  % level/.style={sibling distance = 20em, level distance = 3em},
  level 1/.style={sibling distance=6em, level distance = 3em},
  level 2/.style={sibling distance=3em, level distance = 3em},
  level 3/.style={sibling distance=2em, level distance = 3em}
]
\node [bbv] (r){c}
child [color=black] { node [brv] {a}
    child [color=black] { node [brv] {b}
      child [color=black] {node [bwv] {x}}
      child [color=black] {node [bwv] {y}}
    }
    child [color=black] {node [bwv] {z}}
}
child [color=black] { node [bwv] {w}};
\end{tikzpicture}
\end{figure}
   
\end{column}%
\hfill%
\begin{column}{.31\textwidth}
\rule{\linewidth}{4pt}
      
    
\begin{figure}[!h]
  \centering
  % \centering
  \caption{Durante}
\begin{tikzpicture}[->,>=stealth',
   % level/.style={sibling distance = 20em, level distance = 3em},
   level 1/.style={sibling distance=6em, level distance = 3em},
  level 2/.style={sibling distance=3em, level distance = 3em},
  level 3/.style={sibling distance=3em, level distance = 3em},
  ] 
  \node [brv] (r){a}
  child [color=black] { node [brv] {b}
    child [color=black] {node [bwv] {x}}
    child [color=black] {node [bwv] {y}}
  }
  child [color=black] { node [bbv] {c}
    child [color=black] {node [bwv] {z}}
    child [color=black] {node [bwv] {w}}
  };
\end{tikzpicture}
\end{figure}
\end{column}%
%
\hfill%
\begin{column}{.31\textwidth}
\rule{\linewidth}{4pt}
      
    
\begin{figure}[!h]
  \centering
  % \centering
  \caption{Depois}
  \begin{tikzpicture}[->,>=stealth',
    % level/.style={sibling distance = 20em, level distance = 3em},
    level 1/.style={sibling distance=6em, level distance = 3em},
   level 2/.style={sibling distance=3em, level distance = 3em},
   level 3/.style={sibling distance=3em, level distance = 3em},
   ] 
   \node [bbv] (r){a}
   child [color=black] { node [brv] {b}
     child [color=black] {node [bwv] {x}}
     child [color=black] {node [bwv] {y}}
   }
   child [color=black] { node [brv] {c}
     child [color=black] {node [bwv] {z}}
     child [color=black] {node [bwv] {w}}
   };
 \end{tikzpicture}

\end{figure}
\end{column}%
\end{columns}
\end{frame}
  