% \chapter{Árvores Rubro-Negras}
\begin{frame}[fragile]
  \frametitle{Árvore Rubro-Negra}
\begin{theorem}
Uma árvore rubro-negra é uma árvore binária de busca balanceada na 
qual cada nó interno tem dois filhos. Cada nó interno tem uma cor, 
tal que:

\begin{itemize}
\item[0.] Cada nó é preto ou vermelho 
\item[1.] O nó raiz é sempre preto.
\item[2.] Todos os nós folha (Null) são pretos.
% \item[3.] Novas inserções são sempre vermelhas.  
% \item[4.] Todos os caminhos percorridos na árvore contém o mesmo número de nós pretos (\textbf{\emph{altura-negra}}).
% \item[5.] Nenhum caminho pode ter dois nós vermelhos consecultivos. 
% \item[6.] Se um nó é vermelho, então ambos os filhos são pretos. Os ponteiros dos nós folha (representados como quadrados pretos) continuam sendo "nulos" e não carregam nenhum valor.
\end{itemize}
\end{theorem}

Precisamos adaptar as operações de inserção e deleção de forma que as 
propriedades rubro negras sejam mantidas.
\end{frame}

\begin{frame}[fragile]
  \frametitle{Árvore Rubro-Negra (cont)}
\begin{theorem}
Uma árvore rubro-negra é uma árvore binária de busca balanceada na 
qual cada nó interno tem dois filhos. Cada nó interno tem uma cor, 
tal que:

\begin{itemize}
% \item[0.] Cada nó é preto ou vermelho 
% \item[1.] O nó raiz é sempre preto.
% \item[2.] Todos os nós folha (Null) são pretos.
\item[3.] Novas inserções são sempre vermelhas.  
\item[4.] Todos os caminhos percorridos na árvore contém o mesmo número de nós pretos (\textbf{\emph{altura-negra}}).
\item[5.] Nenhum caminho pode ter dois nós vermelhos consecultivos. 
\item[6.] Se um nó é vermelho, então ambos os filhos são pretos. Os ponteiros 
          dos nós folha (representados como quadrados pretos) continuam sendo "nulos" e não carregam nenhum valor.
\end{itemize}
\end{theorem}

Precisamos adaptar as operações de inserção e deleção de forma que as 
propriedades rubro negras sejam mantidas.
\end{frame}

\section{Correção de balanceamento de altura-negra}
\label{caso1}

\begin{frame}[fragile]
  \frametitle{Caso 1 - O tio de \textbf{a}, (\textbf{d}), é vermelho}
TVC - tio-vermelho-colorir 
Então como a gente faz pra corrigir a quebra de invariante (coloração rubro-negra?) 

É o que vamos ver agora!
\begin{figure}
\begin{minipage}[t]{0.48\linewidth}
\centering
\caption{Antes}
\begin{tikzpicture}[->,>=stealth',
  level 1/.style={sibling distance=6em, level distance = 3em},
  level 2/.style={sibling distance=2em, level distance = 3em},
  level 3/.style={sibling distance=1em, level distance = 3em}
]
\node [bbv] (r){c}
child [color=black] { node [brv] {b}
        child [color=black] {node [bwv] {z}}
        child [color=black] { node [brv] {a}
          child [color=black] {node [bwv] {x}}
          child [color=black] {node [bwv] {y}}
        }
      }
child [color=black] { node [brv] {d}
        child [color=black] {node [bwv] {p}}
        child [color=black] {node [bwv] {q}}
};
\end{tikzpicture}
\end{minipage}\hfill
\begin{minipage}[t]{0.48\linewidth}
\caption{Depois}
  \begin{tikzpicture}[->,>=stealth',
  level 1/.style={sibling distance=6em, level distance = 3em},
  level 2/.style={sibling distance=2em, level distance = 3em},
  level 3/.style={sibling distance=1em, level distance = 3em}
]
\node [brv] (r){c}
child [color=black] { node [bbv] {b}
        child [color=black] {node [bwv] {z}}
        child [color=black] { node [brv] {a}
          child [color=black] {node [bwv] {x}}
          child [color=black] {node [bwv] {y}}
        }
      }
child [color=black] { node [bbv] {d}
        child [color=black] {node [bwv] {p}}
        child [color=black] {node [bwv] {q}}
};
\end{tikzpicture}
\end{minipage}
\end{figure}
\end{frame}


\begin{frame}[fragile]
\frametitle{Caso 2 - O tio de \textbf{a}, (\textbf{d}), é preto e \textbf{a} é filho direito}
TPR - tio-preto-rotaciona
% \pagebreak
\begin{itemize}
  \item Rotação à esquerda do pai de \textbf{a}, (\textbf{b}), ao redor de \textbf{a}
  \item Continua para o \textbf{Caso 3}
\end{itemize}

\begin{figure}[H]
  \begin{minipage}[t]{0.48\textwidth}
  \caption{Antes}
\begin{tikzpicture}[->,>=stealth',
  % level/.style={sibling distance = 20em, level distance = 3em},
  level 1/.style={sibling distance=6em, level distance = 3em},
  level 2/.style={sibling distance=2em, level distance = 3em},
  level 3/.style={sibling distance=1em, level distance = 3em}
]
\node [bbv] (r){c}
child [color=black] { node [brv] {b}
        child [color=black] {node [bwv] {z}}
        child [color=black] { node [brv] {a}
          child [color=black] {node [bwv] {x}}
          child [color=black] {node [bwv] {y}}
        }
      }
child [color=black] { node [bbv] {d}
        child [color=black] {node [bwv] {p}}
        child [color=black] {node [bwv] {q}}
};
\end{tikzpicture}
\end{minipage}\hfill
\begin{minipage}[t]{0.48\linewidth}
  % \centering
  \caption{Depois}
\begin{tikzpicture}[->,>=stealth',
   % level/.style={sibling distance = 20em, level distance = 3em},
   level 1/.style={sibling distance=10em, level distance = 3em},
  level 2/.style={sibling distance=4em, level distance = 3em},
  level 3/.style={sibling distance=2em, level distance = 3em},
  ] 
\node [bbv] (r){c}
child [color=black] { node [brv] {a}
        child [color=black] { node [brv] {b}
          child [color=black] {node [bwv] {y}}
          child [color=black] {node [bwv] {x}}
        }
        child [color=black] {node [bwv] {z}}
      }
child [color=black] { node [bbv] {d}
        child [color=black] {node [bwv] {p}}
        child [color=black] {node [bwv] {q}}
};
\end{tikzpicture}
\end{minipage}
\end{figure}
\end{frame}

\subsection{}\label{caso3}
\begin{frame}[fragile]
  \frametitle{Caso 3 - O tio de \textbf{a}, (\textbf{d}), é preto e \textbf{a} é filho esquerdo }

\begin{itemize}
  \item Rotação à direita de \textbf{c}, ao redor de \textbf{b}
  \item Troque as cores entre o pai de \textbf{a}, (\textbf{b}) e seu novo irmão (\textbf{c}).
\end{itemize}

\begin{figure}[H]
  \centering
  \begin{minipage}{.3\textwidth}
    \caption{Antes}
  \begin{tikzpicture}[->,>=stealth',
     % level/.style={sibling distance = 20em, level distance = 3em},
     level 1/.style={sibling distance=6em, level distance = 3em},
     level 2/.style={sibling distance=2em, level distance = 3em},
     level 3/.style={sibling distance=1em, level distance = 3em}
    ] 
  \node [bbv] (r){c}
  child [color=black] { node [brv] {a}
          child [color=black] { node [brv] {b}
            child [color=black] {node [bwv] {y}}
            child [color=black] {node [bwv] {x}}
          }
          child [color=black] {node [bwv] {z}}
        }
  child [color=black] { node [bbv] {d}
          child [color=black] {node [bwv] {p}}
          child [color=black] {node [bwv] {q}}
  };
  \end{tikzpicture}

\end{minipage}
\hfill
\begin{minipage}{.3\textwidth}
  \caption{Durante}
\begin{tikzpicture}[->,>=stealth',
   % level/.style={sibling distance = 20em, level distance = 3em},
   level 1/.style={sibling distance=6em, level distance = 3em},
     level 2/.style={sibling distance=2em, level distance = 3em},
     level 3/.style={sibling distance=1em, level distance = 3em}
  ] 
\node [brv] (r){b}
child [color=black] { node [brv] {a}
        child [color=black] {node [bwv] {z}}
        child [color=black] {node [bwv] {x}}
      }
child [color=black] { node [bbv] {c}
        child [color=black] {node [bwv] {y}}
        child [color=black] { node [bbv] {d}
          child [color=black] {node [bwv] {p}}
          child [color=black] {node [bwv] {q}}
        }
};
\end{tikzpicture}

\end{minipage}
\hfill
\begin{minipage}{.3\textwidth}
  \centering
  \caption{Depois}
\begin{tikzpicture}[->,>=stealth',
   % level/.style={sibling distance = 20em, level distance = 3em},
   level 1/.style={sibling distance=6em, level distance = 3em},
   level 2/.style={sibling distance=2em, level distance = 3em},
   level 3/.style={sibling distance=1em, level distance = 3em}
  ] 
\node [bbv] (r){b}
child [color=black] { node [brv] {a}
        child [color=black] {node [bwv] {z}}
        child [color=black] {node [bwv] {x}}
      }
child [color=black] { node [brv] {c}
        child [color=black] {node [bwv] {y}}
        child [color=black] { node [bbv] {d}
          child [color=black] {node [bwv] {p}}
          child [color=black] {node [bwv] {q}}
        }
};
\end{tikzpicture}

\end{minipage}
\end{figure}
\end{frame}

\begin{frame}[fragile]
  \frametitle{Caso 2 - O tio de \textbf{a}, (\textbf{d}), é preto e \textbf{a} é filho direito}
  TPR - tio-preto-rotaciona
  % \pagebreak
  \begin{itemize}
    \item Rotação à esquerda do pai de \textbf{a}, (\textbf{b}), ao redor de \textbf{a}
    \item Continua para o \textbf{Caso 3}
  \end{itemize}
  
  \begin{figure}[H]
    \begin{minipage}[t]{0.48\textwidth}
    \caption{Antes}
  \begin{tikzpicture}[->,>=stealth',
    % level/.style={sibling distance = 20em, level distance = 3em},
    level 1/.style={sibling distance=6em, level distance = 3em},
    level 2/.style={sibling distance=2em, level distance = 3em},
    level 3/.style={sibling distance=1em, level distance = 3em}
  ]
  \node [bbv] (r){c}
  child [color=black] { node [brv] {b}
          child [color=black] {node [bwv] {z}}
          child [color=black] { node [brv] {a}
            child [color=black] {node [bwv] {x}}
            child [color=black] {node [bwv] {y}}
          }
        }
  child [color=black] { node [bbv] {d}
          child [color=black] {node [bwv] {p}}
          child [color=black] {node [bwv] {q}}
  };
  \end{tikzpicture}
  \end{minipage}\hfill
  \begin{minipage}[t]{0.48\linewidth}
    % \centering
    \caption{Depois}
  \begin{tikzpicture}[->,>=stealth',
     % level/.style={sibling distance = 20em, level distance = 3em},
     level 1/.style={sibling distance=10em, level distance = 3em},
    level 2/.style={sibling distance=4em, level distance = 3em},
    level 3/.style={sibling distance=2em, level distance = 3em},
    ] 
  \node [bbv] (r){c}
  child [color=black] { node [brv] {a}
          child [color=black] { node [brv] {b}
            child [color=black] {node [bwv] {y}}
            child [color=black] {node [bwv] {x}}
          }
          child [color=black] {node [bwv] {z}}
        }
  child [color=black] { node [bbv] {d}
          child [color=black] {node [bwv] {p}}
          child [color=black] {node [bwv] {q}}
  };
  \end{tikzpicture}
  \end{minipage}
  \end{figure}
  \end{frame}
  
  \section{Construindo uma árvore rubro-negra do zero}
  \begin{frame}[fragile]
\frametitle{\secname}

Vamos usar os exemplos que fizemos nas atividades passadas (post do dia 10/03), as que vocês submeteram, ok?
Usaremos o Conjunto 1: 3, 46, 41, 11, 42, 14, 10, 5, 24, 20, 31, 44, 48, 39, 21.

Começamos pelo número 3.
\begin{figure}[H]
  
  \centering
  \caption{Conjunto 1 - Inserir 3}
  \tiny
\begin{tikzpicture}[
  % 
  ->,>=stealth',
  % level/.style={sibling distance = 20em, level distance = 3em},
  level 1/.style={sibling distance=14em, level distance = 3em},
  level 2/.style={sibling distance=6em, level distance = 3em},
  level 3/.style={sibling distance=3em, level distance = 3em}]
\node [bv] (r){3}
child [color=black] {node [nil] {}}
child [color=black] {node [nil] {}}
;
\end{tikzpicture}
\end{figure}
\end{frame}

\begin{frame}[fragile]

Daí inserimos o número 46.

\begin{figure}[H]
  
  \centering
  \caption{Conjunto 1 - Inserir 46}
  \tiny
\begin{tikzpicture}[
  % 
  ->,>=stealth',
  % level/.style={sibling distance = 20em, level distance = 3em},
  level 1/.style={sibling distance=14em, level distance = 3em},
  level 2/.style={sibling distance=6em, level distance = 3em},
  level 3/.style={sibling distance=3em, level distance = 3em}]
\node [bv] (r){3}
child [color=black] {node [nil] {}}
child [color=black] {node [rv] {46}}
  % child [color=black] { node [brv] {13}
  %   child [color=black] {node [bwv] {...}}
  %   child [color=black] {node [bwv] {...}}
  % }
  % child [color=black] { node [brv] {25}
  %   child [color=black] {node [bwv] {...}}
  %   child [color=black] { node [bbv] {30}
  %     child [color=black] {node [bwv] {...}}
  %     child [color=black] {node [bwv] {...}}
  %   }
  % }
;
\end{tikzpicture}
\end{figure}
\end{frame}
\begin{frame}[fragile]

E inserimos o número 41. Eita! Dois vermelhos consecultivos.
\begin{figure}[H]
  
  \centering
  \caption{Conjunto 1 - Inserir 41}
  \tiny
\begin{tikzpicture}[
  % 
  ->,>=stealth',
  % level/.style={sibling distance = 20em, level distance = 3em},
  level 1/.style={sibling distance=14em, level distance = 3em},
  level 2/.style={sibling distance=6em, level distance = 3em},
  level 3/.style={sibling distance=3em, level distance = 3em}]
\node [bv] (r){3}
child [color=black] {node [nil] {}}
child [color=black] { node [rv] {46}
  child [color=black] { node [rv] {41}
    child [color=black] {node [nil] {}}
    child [color=black] {node [nil] {}}
  }
  child [color=black] {node [nil] {}}
}
  % child [color=black] { node [brv] {13}
  %   child [color=black] {node [bwv] {...}}
  %   child [color=black] {node [bwv] {...}}
  % }
  % child [color=black] { node [brv] {25}
  %   child [color=black] {node [bwv] {...}}
  %   child [color=black] { node [bbv] {30}
  %     child [color=black] {node [bwv] {...}}
  %     child [color=black] {node [bwv] {...}}
  %   }
  % }
;
\end{tikzpicture}
\end{figure}
\end{frame}
\begin{frame}[fragile]

Observe que aqui temos o caso 2 simétrico (invertido). No caso, \textbf{TPR}.

\begin{figure}[H]
  
  \centering
  \caption{Conjunto 1 - Inserir 41 - Correção (caso 2)}
  \tiny
\begin{tikzpicture}[
  % 
  ->,>=stealth',
  % level/.style={sibling distance = 20em, level distance = 3em},
  level 1/.style={sibling distance=14em, level distance = 3em},
  level 2/.style={sibling distance=6em, level distance = 3em},
  level 3/.style={sibling distance=3em, level distance = 3em}]
\node [bv] (r){3}
child [color=black] {node [nil] {}}
child [color=black] { node [rv] {41}
  child [color=black] {node [nil] {}}
  child [color=black] { node [rv] {46}
    child [color=black] {node [nil] {}}
    child [color=black] {node [nil] {}}
  }
}
  % child [color=black] { node [brv] {13}
  %   child [color=black] {node [bwv] {...}}
  %   child [color=black] {node [bwv] {...}}
  % }
  % child [color=black] { node [brv] {25}
  %   child [color=black] {node [bwv] {...}}
  %   child [color=black] { node [bbv] {30}
  %     child [color=black] {node [bwv] {...}}
  %     child [color=black] {node [bwv] {...}}
  %   }
  % }
;
\end{tikzpicture}
\end{figure}

\end{frame}
\begin{frame}[fragile]

O que faremos então?
\begin{itemize}
  \item[1] Rotação simples à esquerda de 3 ao redor de 41 (RSE 3 AR 41).
  \item[2] ...
\end{itemize}
\begin{figure}[H]
  
  \centering
  \caption{Conjunto 1 - Inserir 41 - Correção (caso 3 - durante)}
  \tiny
\begin{tikzpicture}[
  % 
  ->,>=stealth',
  % level/.style={sibling distance = 20em, level distance = 3em},
  level 1/.style={sibling distance=14em, level distance = 3em},
  level 2/.style={sibling distance=6em, level distance = 3em},
  level 3/.style={sibling distance=3em, level distance = 3em}]
\node [rv] (r){41}
child [color=black] { node [bv] {3}
  child [color=black] {node [nil] {}}
  child [color=black] {node [nil] {}}
}
child [color=black] { node [rv] {46}
  child [color=black] {node [nil] {}}
  child [color=black] {node [nil] {}}
}
  % child [color=black] { node [brv] {13}
  %   child [color=black] {node [bwv] {...}}
  %   child [color=black] {node [bwv] {...}}
  % }
  % child [color=black] { node [brv] {25}
  %   child [color=black] {node [bwv] {...}}
  %   child [color=black] { node [bbv] {30}
  %     child [color=black] {node [bwv] {...}}
  %     child [color=black] {node [bwv] {...}}
  %   }
  % }
;
\end{tikzpicture}
\end{figure}
\end{frame}
\begin{frame}[fragile]


E agora? O que faremos então?
\begin{itemize}
  \item[1] Rotação simples à esquerda de 3 ao redor de 41 (RSE 3 AR 41).
  \item[2] Trocar as cores do pai de 46 (41) com o (tio que agora é irmão) de 46 (3)
\end{itemize}
\begin{figure}[H]
  
  \centering
  \caption{Conjunto 1 - Inserir 41 - Correção (caso 3 - durante)}
  \tiny
\begin{tikzpicture}[
  % 
  ->,>=stealth',
  % level/.style={sibling distance = 20em, level distance = 3em},
  level 1/.style={sibling distance=14em, level distance = 3em},
  level 2/.style={sibling distance=6em, level distance = 3em},
  level 3/.style={sibling distance=3em, level distance = 3em}]
\node [bv] (r){41}
child [color=black] { node [rv] {3}
  child [color=black] {node [nil] {}}
  child [color=black] {node [nil] {}}
}
child [color=black] { node [rv] {46}
  child [color=black] {node [nil] {}}
  child [color=black] {node [nil] {}}
}
  % child [color=black] { node [brv] {13}
  %   child [color=black] {node [bwv] {...}}
  %   child [color=black] {node [bwv] {...}}
  % }
  % child [color=black] { node [brv] {25}
  %   child [color=black] {node [bwv] {...}}
  %   child [color=black] { node [bbv] {30}
  %     child [color=black] {node [bwv] {...}}
  %     child [color=black] {node [bwv] {...}}
  %   }
  % }
;
\end{tikzpicture}
\end{figure}
Continuando... Inserção de 11
\end{frame}
\begin{frame}[fragile]

\begin{figure}[H]
  
  \centering
  \caption{Conjunto 1 - Inserir 11}
  \tiny
\begin{tikzpicture}[
  % 
  ->,>=stealth',
  % level/.style={sibling distance = 20em, level distance = 3em},
  level 1/.style={sibling distance=14em, level distance = 3em},
  level 2/.style={sibling distance=6em, level distance = 3em},
  level 3/.style={sibling distance=3em, level distance = 3em}]
\node [bv] (r){41}
child [color=black] { node [rv] {3}
  child [color=black] {node [nil] {}}
  child [color=black] { node [rv] {11}
    child [color=black] {node [nil] {}}
    child [color=black] {node [nil] {}}
  }
}
child [color=black] { node [rv] {46}
  child [color=black] {node [nil] {}}
  child [color=black] {node [nil] {}}
}
  % child [color=black] { node [brv] {13}
  %   child [color=black] {node [bwv] {...}}
  %   child [color=black] {node [bwv] {...}}
  % }
  % child [color=black] { node [brv] {25}
  %   child [color=black] {node [bwv] {...}}
  %   child [color=black] { node [bbv] {30}
  %     child [color=black] {node [bwv] {...}}
  %     child [color=black] {node [bwv] {...}}
  %   }
  % }
;
\end{tikzpicture}
\end{figure}
\end{frame}
\begin{frame}[fragile]

Aqui temos o caso 1, ok? Tio vermelho, colorir.
\begin{itemize}
  \item Primeiro a gente inverte as cores do pai, tio e avô de 11
  \item ...
\end{itemize}

\begin{figure}[H]
  
  \centering
  \caption{Conjunto 1 - Inserir 11 - Caso 1 - inverte as cores}
  \tiny
\begin{tikzpicture}[
  % 
  ->,>=stealth',
  % level/.style={sibling distance = 20em, level distance = 3em},
  level 1/.style={sibling distance=14em, level distance = 3em},
  level 2/.style={sibling distance=6em, level distance = 3em},
  level 3/.style={sibling distance=3em, level distance = 3em}]
\node [rv] (r){41}
child [color=black] { node [bv] {3}
  child [color=black] {node [nil] {}}
  child [color=black] { node [rv] {11}
    child [color=black] {node [nil] {}}
    child [color=black] {node [nil] {}}
  }
}
child [color=black] { node [bv] {46}
  child [color=black] {node [nil] {}}
  child [color=black] {node [nil] {}}
}
  % child [color=black] { node [brv] {13}
  %   child [color=black] {node [bwv] {...}}
  %   child [color=black] {node [bwv] {...}}
  % }
  % child [color=black] { node [brv] {25}
  %   child [color=black] {node [bwv] {...}}
  %   child [color=black] { node [bbv] {30}
  %     child [color=black] {node [bwv] {...}}
  %     child [color=black] {node [bwv] {...}}
  %   }
  % }
;
\end{tikzpicture}
\end{figure}
\end{frame}
\begin{frame}[fragile]

\begin{itemize}
  \item Primeiro a gente inverte as cores do pai, tio e avô de 11
  \item E como sabemos, o nó raiz nunca pode ser vermelho, nós recolorimos ele para preto, também.
\end{itemize}


\begin{figure}[H]
  
  \centering
  \caption{Conjunto 1 - Inserir 11 - Caso 1 - inverte as cores}
  \tiny
\begin{tikzpicture}[
  % 
  ->,>=stealth',
  % level/.style={sibling distance = 20em, level distance = 3em},
  level 1/.style={sibling distance=14em, level distance = 3em},
  level 2/.style={sibling distance=6em, level distance = 3em},
  level 3/.style={sibling distance=3em, level distance = 3em}]
\node [bv] (r){41}
child [color=black] { node [bv] {3}
  child [color=black] {node [nil] {}}
  child [color=black] { node [rv] {11}
    child [color=black] {node [nil] {}}
    child [color=black] {node [nil] {}}
  }
}
child [color=black] { node [bv] {46}
  child [color=black] {node [nil] {}}
  child [color=black] {node [nil] {}}
}
  % child [color=black] { node [brv] {13}
  %   child [color=black] {node [bwv] {...}}
  %   child [color=black] {node [bwv] {...}}
  % }
  % child [color=black] { node [brv] {25}
  %   child [color=black] {node [bwv] {...}}
  %   child [color=black] { node [bbv] {30}
  %     child [color=black] {node [bwv] {...}}
  %     child [color=black] {node [bwv] {...}}
  %   }
  % }
;
\end{tikzpicture}
\end{figure}
\end{frame}
\begin{frame}[fragile]

Até aqui, a árvore está balanceada, correto? Vamos continuar, inserindo 42.


\begin{figure}[H]
  
  \centering
  \caption{Conjunto 1 - Inserir 42}
  \tiny
\begin{tikzpicture}[
  % 
  ->,>=stealth',
  % level/.style={sibling distance = 20em, level distance = 3em},
  level 1/.style={sibling distance=14em, level distance = 3em},
  level 2/.style={sibling distance=6em, level distance = 3em},
  level 3/.style={sibling distance=3em, level distance = 3em}]
\node [bv] (r){41}
child [color=black] { node [bv] {3}
  child [color=black] {node [nil] {}}
  child [color=black] { node [rv] {11}
    child [color=black] {node [nil] {}}
    child [color=black] {node [nil] {}}
  }
}
child [color=black] { node [bv] {46}
  child [color=black] { node [rv] {42}
    child [color=black] {node [nil] {}}
    child [color=black] {node [nil] {}}
  }
  child [color=black] {node [nil] {}}
}
  % child [color=black] { node [brv] {13}
  %   child [color=black] {node [bwv] {...}}
  %   child [color=black] {node [bwv] {...}}
  % }
  % child [color=black] { node [brv] {25}
  %   child [color=black] {node [bwv] {...}}
  %   child [color=black] { node [bbv] {30}
  %     child [color=black] {node [bwv] {...}}
  %     child [color=black] {node [bwv] {...}}
  %   }
  % }
;
\end{tikzpicture}
\end{figure}
\end{frame}
\begin{frame}[fragile]

Tudo certo, ainda, né? Mesma \textbf{altura-negra} para todos os nós. Partxiu bagunçar a árvore, inserindo o 14?


\begin{figure}[H]
\caption{Conjunto 1 - Inserir 14}
 
\begin{tikzpicture}[
  % 
  ->,>=stealth',
  % level/.style={sibling distance = 20em, level distance = 3em},
  level 1/.style={sibling distance=14em, level distance = 3em},
  level 2/.style={sibling distance=6em, level distance = 3em},
  level 3/.style={sibling distance=3em, level distance = 3em}]
\node [bv] (r){41}
child [color=black] { node [bv] {3}
  child [color=black] {node [nil] {}}
  child [color=black] { node [rv] {11}
    child [color=black] {node [nil] {}}
    child [color=black] { node [rv] {14}
      child [color=black] {node [nil] {}}
      child [color=black] {node [nil] {}}
    }
  }
}
child [color=black] { node [bv] {46}
  child [color=black] { node [rv] {42}
    child [color=black] {node [nil] {}}
    child [color=black] {node [nil] {}}
  }
  child [color=black] {node [nil] {}}
}
  % child [color=black] { node [brv] {13}
  %   child [color=black] {node [bwv] {...}}
  %   child [color=black] {node [bwv] {...}}
  % }
  % child [color=black] { node [brv] {25}
  %   child [color=black] {node [bwv] {...}}
  %   child [color=black] { node [bbv] {30}
  %     child [color=black] {node [bwv] {...}}
  %     child [color=black] {node [bwv] {...}}
  %   }
  % }
;
\end{tikzpicture}
\end{figure}
\end{frame}
\begin{frame}[fragile]
Veja, aqui a gente tem o caso 3 simétrico.

\begin{figure}[H]
  
  \centering
  \caption{Conjunto 1 - Inserir 14 - Caso 3 - Durante}
  \tiny
\begin{tikzpicture}[
  % 
  ->,>=stealth',
  % level/.style={sibling distance = 20em, level distance = 3em},
  level 1/.style={sibling distance=14em, level distance = 3em},
  level 2/.style={sibling distance=6em, level distance = 3em},
  level 3/.style={sibling distance=3em, level distance = 3em}]
\node [bv] (r){41}
child [color=black] { node [rv] {11}
    child [color=black] { node [bv] {3}
      child [color=black] {node [nil] {}}
      child [color=black] {node [nil] {}}
    }
    child [color=black] { node [rv] {14}
      child [color=black] {node [nil] {}}
      child [color=black] {node [nil] {}}
    }
}
child [color=black] { node [bv] {46}
  child [color=black] { node [rv] {42}
    child [color=black] {node [nil] {}}
    child [color=black] {node [nil] {}}
  }
  child [color=black] {node [nil] {}}
}
  % child [color=black] { node [brv] {13}
  %   child [color=black] {node [bwv] {...}}
  %   child [color=black] {node [bwv] {...}}
  % }
  % child [color=black] { node [brv] {25}
  %   child [color=black] {node [bwv] {...}}
  %   child [color=black] { node [bbv] {30}
  %     child [color=black] {node [bwv] {...}}
  %     child [color=black] {node [bwv] {...}}
  %   }
  % }
;
\end{tikzpicture}
\end{figure}
\end{frame}
\begin{frame}[fragile]

Agora é só inverter as cores de 11 e 3. Moleza, né?


\begin{figure}[H]
  \caption{Conjunto 1 - Inserir 14 - Caso 3 - Depois}
    \centering
    \tiny
    \begin{tikzpicture}[
      % 
      ->,>=stealth',
      % level/.style={sibling distance = 20em, level distance = 3em},
      level 1/.style={sibling distance=14em, level distance = 3em},
      level 2/.style={sibling distance=6em, level distance = 3em},
      level 3/.style={sibling distance=3em, level distance = 3em}]
    \node [bv] (r){41}
    child [color=black] { node [bv] {11}
        child [color=black] { node [rv] {3}
          child [color=black] {node [nil] {}}
          child [color=black] {node [nil] {}}
        }
        child [color=black] { node [rv] {14}
          child [color=black] {node [nil] {}}
          child [color=black] {node [nil] {}}
        }
    }
    child [color=black] { node [bv] {46}
      child [color=black] { node [rv] {42}
        child [color=black] {node [nil] {}}
        child [color=black] {node [nil] {}}
      }
      child [color=black] {node [nil] {}}
    }
      % child [color=black] { node [brv] {13}
      %   child [color=black] {node [bwv] {...}}
      %   child [color=black] {node [bwv] {...}}
      % }
      % child [color=black] { node [brv] {25}
      %   child [color=black] {node [bwv] {...}}
      %   child [color=black] { node [bbv] {30}
      %     child [color=black] {node [bwv] {...}}
      %     child [color=black] {node [bwv] {...}}
      %   }
      % }
    ;
    \end{tikzpicture}
  \end{figure}
\end{frame}
\begin{frame}[fragile]

De boa. Agora vamos inserir o 10.

\begin{figure}[H]
  \caption{Conjunto 1 - Inserir 10}
  
    \centering
    \tiny
    \begin{tikzpicture}[
      % 
      ->,>=stealth',
      % level/.style={sibling distance = 20em, level distance = 3em},
      level 1/.style={sibling distance=14em, level distance = 3em},
      level 2/.style={sibling distance=6em, level distance = 3em},
      level 3/.style={sibling distance=3em, level distance = 3em}]
    \node [bv] (r){41}
    child [color=black] { node [bv] {11}
        child [color=black] { node [rv] {3}
          child [color=black] {node [nil] {}}
          child [color=black] { node [rv] {10}
            child [color=black] {node [nil] {}}
            child [color=black] {node [nil] {}}
          }
        }
        child [color=black] { node [rv] {14}
          child [color=black] {node [nil] {}}
          child [color=black] {node [nil] {}}
        }
    }
    child [color=black] { node [bv] {46}
      child [color=black] { node [rv] {42}
        child [color=black] {node [nil] {}}
        child [color=black] {node [nil] {}}
      }
      child [color=black] {node [nil] {}}
    }
      % child [color=black] { node [brv] {13}
      %   child [color=black] {node [bwv] {...}}
      %   child [color=black] {node [bwv] {...}}
      % }
      % child [color=black] { node [brv] {25}
      %   child [color=black] {node [bwv] {...}}
      %   child [color=black] { node [bbv] {30}
      %     child [color=black] {node [bwv] {...}}
      %     child [color=black] {node [bwv] {...}}
      %   }
      % }
    ;
    \end{tikzpicture}
  
  \end{figure}
\end{frame}
\begin{frame}[fragile]

Aqui a gente tem o Caso 1, ok? Tio vermelho, colorir.

\begin{figure}[H]
  \caption{Conjunto 1 - Inserir 10 - TVC}
  
    \centering
    \tiny
    \begin{tikzpicture}[
      % 
      ->,>=stealth',
      % level/.style={sibling distance = 20em, level distance = 3em},
      level 1/.style={sibling distance=14em, level distance = 3em},
      level 2/.style={sibling distance=6em, level distance = 3em},
      level 3/.style={sibling distance=3em, level distance = 3em}]
    \node [bv] (r){41}
    child [color=black] { node [rv] {11}
        child [color=black] { node [bv] {3}
          child [color=black] {node [nil] {}}
          child [color=black] { node [rv] {10}
            child [color=black] {node [nil] {}}
            child [color=black] {node [nil] {}}
          }
        }
        child [color=black] { node [bv] {14}
          child [color=black] {node [nil] {}}
          child [color=black] {node [nil] {}}
        }
    }
    child [color=black] { node [bv] {46}
      child [color=black] { node [rv] {42}
        child [color=black] {node [nil] {}}
        child [color=black] {node [nil] {}}
      }
      child [color=black] {node [nil] {}}
    }
      % child [color=black] { node [brv] {13}
      %   child [color=black] {node [bwv] {...}}
      %   child [color=black] {node [bwv] {...}}
      % }
      % child [color=black] { node [brv] {25}
      %   child [color=black] {node [bwv] {...}}
      %   child [color=black] { node [bbv] {30}
      %     child [color=black] {node [bwv] {...}}
      %     child [color=black] {node [bwv] {...}}
      %   }
      % }
    ;
    \end{tikzpicture}
  
  \end{figure}
\end{frame}
\begin{frame}[fragile]

Vamos inserir o numero 5


\begin{figure}[H]
  \caption{Conjunto 1 - Inserir 5}
  
    \centering
    \tiny
    \begin{tikzpicture}[
      % 
      ->,>=stealth',
      % level/.style={sibling distance = 20em, level distance = 3em},
      level 1/.style={sibling distance=14em, level distance = 3em},
      level 2/.style={sibling distance=6em, level distance = 3em},
      level 3/.style={sibling distance=3em, level distance = 3em}]
    \node [bv] (r){41}
    child [color=black] { node [rv] {11}
        child [color=black] { node [bv] {3}
          child [color=black] {node [nil] {}}
          child [color=black] { node [rv] {10}
          child [color=black] { node [rv] {5}
            child [color=black] {node [nil] {}}
            child [color=black] {node [nil] {}}
          }
            child [color=black] {node [nil] {}}
          }
        }
        child [color=black] { node [bv] {14}
          child [color=black] {node [nil] {}}
          child [color=black] {node [nil] {}}
        }
    }
    child [color=black] { node [bv] {46}
      child [color=black] { node [rv] {42}
        child [color=black] {node [nil] {}}
        child [color=black] {node [nil] {}}
      }
      child [color=black] {node [nil] {}}
    }
      % child [color=black] { node [brv] {13}
      %   child [color=black] {node [bwv] {...}}
      %   child [color=black] {node [bwv] {...}}
      % }
      % child [color=black] { node [brv] {25}
      %   child [color=black] {node [bwv] {...}}
      %   child [color=black] { node [bbv] {30}
      %     child [color=black] {node [bwv] {...}}
      %     child [color=black] {node [bwv] {...}}
      %   }
      % }
    ;
    \end{tikzpicture}
  
  \end{figure}
\end{frame}
\begin{frame}[fragile]

E aí? Tio preto, rotaciona né? Caso 2 seguido do caso 3.


\begin{figure}[H]
  \caption{Conjunto 1 - Inserir 5 - Caso 2}
  
    \centering
    \tiny
    \begin{tikzpicture}[
      % 
      ->,>=stealth',
      % level/.style={sibling distance = 20em, level distance = 3em},
      level 1/.style={sibling distance=18em, level distance = 3em},
      level 2/.style={sibling distance=8em, level distance = 3em},
      level 3/.style={sibling distance=5em, level distance = 3em}]
    \node [bv] (r){41}
    child [color=black] { node [rv] {11}
        child [color=black] { node [bv] {3}
          child [color=black] {node [nil] {}}
          child [color=black] { node [rv] {5}
            child [color=black] {node [nil] {}}
            child [color=black] { node [rv] {10}
              child [color=black] {node [nil] {}}
              child [color=black] {node [nil] {}}
            }
          }
        }
        child [color=black] { node [bv] {14}
          child [color=black] {node [nil] {}}
          child [color=black] {node [nil] {}}
        }
    }
    child [color=black] { node [bv] {46}
      child [color=black] { node [rv] {42}
        child [color=black] {node [nil] {}}
        child [color=black] {node [nil] {}}
      }
      child [color=black] {node [nil] {}}
    }
      % child [color=black] { node [brv] {13}
      %   child [color=black] {node [bwv] {...}}
      %   child [color=black] {node [bwv] {...}}
      % }
      % child [color=black] { node [brv] {25}
      %   child [color=black] {node [bwv] {...}}
      %   child [color=black] { node [bbv] {30}
      %     child [color=black] {node [bwv] {...}}
      %     child [color=black] {node [bwv] {...}}
      %   }
      % }
    ;
    \end{tikzpicture}
  
  \end{figure}
\end{frame}
\begin{frame}[fragile]

  Caso 3, Rotação à esquerda de 3, ao redor de 5. 

\begin{figure}[H]
  \caption{Conjunto 1 - Inserir 5 - Caso 3 - Durante}
  
    \centering
    \tiny
    \begin{tikzpicture}[
      % 
      ->,>=stealth',
      % level/.style={sibling distance = 20em, level distance = 3em},
      level 1/.style={sibling distance=18em, level distance = 3em},
      level 2/.style={sibling distance=8em, level distance = 3em},
      level 3/.style={sibling distance=5em, level distance = 3em},
      level 4/.style={sibling distance=3em, level distance = 3em}]
    \node [bv] (r){41}
    child [color=black] { node [rv] {11}
      child [color=black] { node [rv] {5}
        child [color=black] { node [bv] {3}
          child [color=black] {node [nil] {}}
          child [color=black] {node [nil] {}}
        }
        child [color=black] { node [rv] {10}
            child [color=black] {node [nil] {}}
            child [color=black] {node [nil] {}}
          }
      }
      child [color=black] { node [bv] {14}
        child [color=black] {node [nil] {}}
        child [color=black] {node [nil] {}}
      }
    }
    child [color=black] { node [bv] {46}
      child [color=black] { node [rv] {42}
        child [color=black] {node [nil] {}}
        child [color=black] {node [nil] {}}
      }
      child [color=black] {node [nil] {}}
    }
      % child [color=black] { node [brv] {13}
      %   child [color=black] {node [bwv] {...}}
      %   child [color=black] {node [bwv] {...}}
      % }
      % child [color=black] { node [brv] {25}
      %   child [color=black] {node [bwv] {...}}
      %   child [color=black] { node [bbv] {30}
      %     child [color=black] {node [bwv] {...}}
      %     child [color=black] {node [bwv] {...}}
      %   }
      % }
    ;
    \end{tikzpicture}
  
  \end{figure}
\end{frame}
\begin{frame}[fragile]

  Em seguida, troca as cores do irmão com o pai.


\begin{figure}[H]
  \caption{Conjunto 1 - Inserir 5 - Caso 3 - Depois}
  
    \centering
    \tiny
    \begin{tikzpicture}[
      % 
      ->,>=stealth',
      % level/.style={sibling distance = 20em, level distance = 3em},
      level 1/.style={sibling distance=18em, level distance = 3em},
      level 2/.style={sibling distance=8em, level distance = 3em},
      level 3/.style={sibling distance=4em, level distance = 3em},
      level 4/.style={sibling distance=3em, level distance = 3em}]
    \node [bv] (r){41}
    child [color=black] { node [rv] {11}
      child [color=black] { node [bv] {5}
        child [color=black] { node [rv] {3}
          child [color=black] {node [nil] {}}
          child [color=black] {node [nil] {}}
        }
        child [color=black] { node [rv] {10}
            child [color=black] {node [nil] {}}
            child [color=black] {node [nil] {}}
          }
      }
      child [color=black] { node [bv] {14}
        child [color=black] {node [nil] {}}
        child [color=black] {node [nil] {}}
      }
    }
    child [color=black] { node [bv] {46}
      child [color=black] { node [rv] {42}
        child [color=black] {node [nil] {}}
        child [color=black] {node [nil] {}}
      }
      child [color=black] {node [nil] {}}
    }
      % child [color=black] { node [brv] {13}
      %   child [color=black] {node [bwv] {...}}
      %   child [color=black] {node [bwv] {...}}
      % }
      % child [color=black] { node [brv] {25}
      %   child [color=black] {node [bwv] {...}}
      %   child [color=black] { node [bbv] {30}
      %     child [color=black] {node [bwv] {...}}
      %     child [color=black] {node [bwv] {...}}
      %   }
      % }
    ;
    \end{tikzpicture}
  
  \end{figure}
\end{frame}
\begin{frame}[fragile]

  Agora vamos inserir o que? Ah, o 24. Esse vai ser fácil, pois ao inserir, 
  ele é vermelho, e tem pai preto. A altura-negra continua a mesma. 
  Tudo certo, né?



\begin{figure}[H]
  \caption{Conjunto 1 - Inserir 24}
  
    \centering
    \tiny
    \begin{tikzpicture}[
      % 
      ->,>=stealth',
      % level/.style={sibling distance = 20em, level distance = 3em},
      level 1/.style={sibling distance=18em, level distance = 3em},
      level 2/.style={sibling distance=8em, level distance = 3em},
      level 3/.style={sibling distance=4em, level distance = 3em},
      level 4/.style={sibling distance=3em, level distance = 3em}]
    \node [bv] (r){41}
    child [color=black] { node [rv] {11}
      child [color=black] { node [bv] {5}
        child [color=black] { node [rv] {3}
          child [color=black] {node [nil] {}}
          child [color=black] {node [nil] {}}
        }
        child [color=black] { node [rv] {10}
            child [color=black] {node [nil] {}}
            child [color=black] {node [nil] {}}
          }
      }
      child [color=black] { node [bv] {14}
        child [color=black] {node [nil] {}}
        child [color=black] { node [rv] {24}
          child [color=black] {node [nil] {}}
          child [color=black] {node [nil] {}}
        }
      }
    }
    child [color=black] { node [bv] {46}
      child [color=black] { node [rv] {42}
        child [color=black] {node [nil] {}}
        child [color=black] {node [nil] {}}
      }
      child [color=black] {node [nil] {}}
    }
      % child [color=black] { node [brv] {13}
      %   child [color=black] {node [bwv] {...}}
      %   child [color=black] {node [bwv] {...}}
      % }
      % child [color=black] { node [brv] {25}
      %   child [color=black] {node [bwv] {...}}
      %   child [color=black] { node [bbv] {30}
      %     child [color=black] {node [bwv] {...}}
      %     child [color=black] {node [bwv] {...}}
      %   }
      % }
    ;
    \end{tikzpicture}
  
  \end{figure}
\end{frame}
\begin{frame}[fragile]

  Inserindo o 20.



\begin{figure}[H]
  \caption{Conjunto 1 - Inserir 20}
  
    \centering
    \tiny
    \begin{tikzpicture}[
      % 
      ->,>=stealth',
      % level/.style={sibling distance = 20em, level distance = 3em},
      level 1/.style={sibling distance=18em, level distance = 3em},
      level 2/.style={sibling distance=8em, level distance = 3em},
      level 3/.style={sibling distance=4em, level distance = 3em},
      level 4/.style={sibling distance=3em, level distance = 3em}]
    \node [bv] (r){41}
    child [color=black] { node [rv] {11}
      child [color=black] { node [bv] {5}
        child [color=black] { node [rv] {3}
          child [color=black] {node [nil] {}}
          child [color=black] {node [nil] {}}
        }
        child [color=black] { node [rv] {10}
            child [color=black] {node [nil] {}}
            child [color=black] {node [nil] {}}
          }
      }
      child [color=black] { node [bv] {14}
        child [color=black] {node [nil] {}}
        child [color=black] { node [rv] {24}
          child [color=black] { node [rv] {20}
            child [color=black] {node [nil] {}}
            child [color=black] {node [nil] {}}
          }
          child [color=black] {node [nil] {}}
        }
      }
    }
    child [color=black] { node [bv] {46}
      child [color=black] { node [rv] {42}
        child [color=black] {node [nil] {}}
        child [color=black] {node [nil] {}}
      }
      child [color=black] {node [nil] {}}
    }
      % child [color=black] { node [brv] {13}
      %   child [color=black] {node [bwv] {...}}
      %   child [color=black] {node [bwv] {...}}
      % }
      % child [color=black] { node [brv] {25}
      %   child [color=black] {node [bwv] {...}}
      %   child [color=black] { node [bbv] {30}
      %     child [color=black] {node [bwv] {...}}
      %     child [color=black] {node [bwv] {...}}
      %   }
      % }
    ;
    \end{tikzpicture}
  
  \end{figure}
\end{frame}
\begin{frame}[fragile]

Mais uma vez, caso 2 + caso 3. Tem esses 3 passos!
\begin{itemize}
  \item[P1] Rotação simples à direita de 24 ao redor de 20.
  \item[P2] Rotação simples à esquerda de 14 ao redor de 20.
  \item[P3] Inverter as cores de 20 e 14.
\end{itemize}

Vamos lá, P1

\begin{figure}[H]
  \caption{Conjunto 1 - Inserir 20 - P1}
  
    \centering
    \tiny
    \begin{tikzpicture}[
      % 
      ->,>=stealth',
      % level/.style={sibling distance = 20em, level distance = 3em},
      level 1/.style={sibling distance=18em, level distance = 3em},
      level 2/.style={sibling distance=8em, level distance = 3em},
      level 3/.style={sibling distance=4em, level distance = 3em},
      level 4/.style={sibling distance=3em, level distance = 3em}]
    \node [bv] (r){41}
    child [color=black] { node [rv] {11}
      child [color=black] { node [bv] {5}
        child [color=black] { node [rv] {3}
          child [color=black] {node [nil] {}}
          child [color=black] {node [nil] {}}
        }
        child [color=black] { node [rv] {10}
            child [color=black] {node [nil] {}}
            child [color=black] {node [nil] {}}
          }
      }
      child [color=black] { node [bv] {14}
        child [color=black] {node [nil] {}}
        child [color=black] { node [rv] {20}
          child [color=black] {node [nil] {}}
          child [color=black] { node [rv] {24}
            child [color=black] {node [nil] {}}
            child [color=black] {node [nil] {}}
          }
        }
      }
    }
    child [color=black] { node [bv] {46}
      child [color=black] { node [rv] {42}
        child [color=black] {node [nil] {}}
        child [color=black] {node [nil] {}}
      }
      child [color=black] {node [nil] {}}
    }
      % child [color=black] { node [brv] {13}
      %   child [color=black] {node [bwv] {...}}
      %   child [color=black] {node [bwv] {...}}
      % }
      % child [color=black] { node [brv] {25}
      %   child [color=black] {node [bwv] {...}}
      %   child [color=black] { node [bbv] {30}
      %     child [color=black] {node [bwv] {...}}
      %     child [color=black] {node [bwv] {...}}
      %   }
      % }
    ;
    \end{tikzpicture}
  
\end{figure}
\end{frame}
\begin{frame}[fragile]

Vamos para o passo 2 (P2), rotação à esquerda de 14 ao redor de 20.

\begin{figure}[H]
  \caption{Conjunto 1 - Inserir 20 - P2}
    \tiny
    \centering
    \begin{tikzpicture}[
      % 
      ->,>=stealth',
      % level/.style={sibling distance = 20em, level distance = 3em},
      level 1/.style={sibling distance=18em, level distance = 3em},
      level 2/.style={sibling distance=8em, level distance = 3em},
      level 3/.style={sibling distance=4em, level distance = 3em},
      level 4/.style={sibling distance=3em, level distance = 3em}]
    \node [bv] (r){41}
    child [color=black] { node [rv] {11}
      child [color=black] { node [bv] {5}
        child [color=black] { node [rv] {3}
          child [color=black] {node [nil] {}}
          child [color=black] {node [nil] {}}
        }
        child [color=black] { node [rv] {10}
            child [color=black] {node [nil] {}}
            child [color=black] {node [nil] {}}
          }
      }
      child [color=black] { node [rv] {20}
        child [color=black] { node [bv] {14}
          child [color=black] {node [nil] {}}
          child [color=black] {node [nil] {}}
        }
        child [color=black] { node [rv] {24}
          child [color=black] {node [nil] {}}
          child [color=black] {node [nil] {}}
        }
      }
    }
    child [color=black] { node [bv] {46}
      child [color=black] { node [rv] {42}
        child [color=black] {node [nil] {}}
        child [color=black] {node [nil] {}}
      }
      child [color=black] {node [nil] {}}
    }
      % child [color=black] { node [brv] {13}
      %   child [color=black] {node [bwv] {...}}
      %   child [color=black] {node [bwv] {...}}
      % }
      % child [color=black] { node [brv] {25}
      %   child [color=black] {node [bwv] {...}}
      %   child [color=black] { node [bbv] {30}
      %     child [color=black] {node [bwv] {...}}
      %     child [color=black] {node [bwv] {...}}
      %   }
      % }
    ;
    \end{tikzpicture}
  
  \end{figure}

\end{frame}
\begin{frame}[fragile]

Terminamos com o passo 2 (P3), invertendo as cores de 14 e 20. 
% Agora tá como tudo deveria ser. O Thanos agradece.

\begin{figure}[H]
  \caption{Conjunto 1 - Inserir 20 - p3}
    \tiny
    \begin{tikzpicture}[
      % 
      ->,>=stealth',
      % level/.style={sibling distance = 20em, level distance = 3em},
      level 1/.style={sibling distance=18em, level distance = 3em},
      level 2/.style={sibling distance=8em, level distance = 3em},
      level 3/.style={sibling distance=4em, level distance = 3em},
      level 4/.style={sibling distance=3em, level distance = 3em}]
    \node [bv] (r){41}
    child [color=black] { node [rv] {11}
      child [color=black] { node [bv] {5}
        child [color=black] { node [rv] {3}
          child [color=black] {node [nil] {}}
          child [color=black] {node [nil] {}}
        }
        child [color=black] { node [rv] {10}
            child [color=black] {node [nil] {}}
            child [color=black] {node [nil] {}}
          }
      }
      child [color=black] { node [bv] {20}
        child [color=black] { node [rv] {14}
          child [color=black] {node [nil] {}}
          child [color=black] {node [nil] {}}
        }
        child [color=black] { node [rv] {24}
          child [color=black] {node [nil] {}}
          child [color=black] {node [nil] {}}
        }
      }
    }
    child [color=black] { node [bv] {46}
      child [color=black] { node [rv] {42}
        child [color=black] {node [nil] {}}
        child [color=black] {node [nil] {}}
      }
      child [color=black] {node [nil] {}}
    }
      % child [color=black] { node [brv] {13}
      %   child [color=black] {node [bwv] {...}}
      %   child [color=black] {node [bwv] {...}}
      % }
      % child [color=black] { node [brv] {25}
      %   child [color=black] {node [bwv] {...}}
      %   child [color=black] { node [bbv] {30}
      %     child [color=black] {node [bwv] {...}}
      %     child [color=black] {node [bwv] {...}}
      %   }
      % }
    ;
    \end{tikzpicture}
  \end{figure}
\end{frame}
\begin{frame}[fragile]

Agora vamos inserir o 31

\begin{figure}[H]
  \caption{Conjunto 1 - Inserir 31}
  \tiny
    \begin{tikzpicture}[
      % 
      ->,>=stealth',
      % level/.style={sibling distance = 20em, level distance = 3em},
      level 1/.style={sibling distance=18em, level distance = 3em},
      level 2/.style={sibling distance=8em, level distance = 3em},
      level 3/.style={sibling distance=4em, level distance = 3em},
      level 4/.style={sibling distance=3em, level distance = 3em}]
    \node [bv] (r){41}
    child [color=black] { node [rv] {11}
      child [color=black] { node [bv] {5}
        child [color=black] { node [rv] {3}
          child [color=black] {node [nil] {}}
          child [color=black] {node [nil] {}}
        }
        child [color=black] { node [rv] {10}
            child [color=black] {node [nil] {}}
            child [color=black] {node [nil] {}}
          }
      }
      child [color=black] { node [bv] {20}
        child [color=black] { node [rv] {14}
          child [color=black] {node [nil] {}}
          child [color=black] {node [nil] {}}
        }
        child [color=black] { node [rv] {24}
          child [color=black] {node [nil] {}}
          child [color=black] { node [rv] {31}
            child [color=black] {node [nil] {}}
            child [color=black] {node [nil] {}}
          }
        }
      }
    }
    child [color=black] { node [bv] {46}
      child [color=black] { node [rv] {42}
        child [color=black] {node [nil] {}}
        child [color=black] {node [nil] {}}
      }
      child [color=black] {node [nil] {}}
    }
      % child [color=black] { node [brv] {13}
      %   child [color=black] {node [bwv] {...}}
      %   child [color=black] {node [bwv] {...}}
      % }
      % child [color=black] { node [brv] {25}
      %   child [color=black] {node [bwv] {...}}
      %   child [color=black] { node [bbv] {30}
      %     child [color=black] {node [bwv] {...}}
      %     child [color=black] {node [bwv] {...}}
      %   }
      % }
    ;
    \end{tikzpicture}
  
  \end{figure}
\end{frame}
\begin{frame}[fragile]

  E aí? Tio vermelho, colorir. Fácil, né?

\begin{figure}[H]
  \caption{Conjunto 1 - Inserir 31}
    \centering
    \tiny
    \begin{tikzpicture}[
      % 
      ->,>=stealth',
      % level/.style={sibling distance = 20em, level distance = 3em},
      level 1/.style={sibling distance=18em, level distance = 3em},
      level 2/.style={sibling distance=8em, level distance = 3em},
      level 3/.style={sibling distance=4em, level distance = 3em},
      level 4/.style={sibling distance=3em, level distance = 3em}]
    \node [bv] (r){41}
    child [color=black] { node [rv] {11}
      child [color=black] { node [bv] {5}
        child [color=black] { node [rv] {3}
          child [color=black] {node [nil] {}}
          child [color=black] {node [nil] {}}
        }
        child [color=black] { node [rv] {10}
            child [color=black] {node [nil] {}}
            child [color=black] {node [nil] {}}
          }
      }
      child [color=black] { node [rv] {20}
        child [color=black] { node [bv] {14}
          child [color=black] {node [nil] {}}
          child [color=black] {node [nil] {}}
        }
        child [color=black] { node [bv] {24}
          child [color=black] {node [nil] {}}
          child [color=black] { node [rv] {31}
            child [color=black] {node [nil] {}}
            child [color=black] {node [nil] {}}
          }
        }
      }
    }
    child [color=black] { node [bv] {46}
      child [color=black] { node [rv] {42}
        child [color=black] {node [nil] {}}
        child [color=black] {node [nil] {}}
      }
      child [color=black] {node [nil] {}}
    }
      % child [color=black] { node [brv] {13}
      %   child [color=black] {node [bwv] {...}}
      %   child [color=black] {node [bwv] {...}}
      % }
      % child [color=black] { node [brv] {25}
      %   child [color=black] {node [bwv] {...}}
      %   child [color=black] { node [bbv] {30}
      %     child [color=black] {node [bwv] {...}}
      %     child [color=black] {node [bwv] {...}}
      %   }
      % }
    ;
    \end{tikzpicture}
    
  \end{figure}
\end{frame}
\begin{frame}[fragile]

  Só que não. Agora temos novamente o caso 2, que implica em caso 3. 
  Primeiro eu vou isolar apenas os nós relevantes para facilitar a visualização.
 
 
\begin{figure}[H]
  \caption{Conjunto 1 - Inserir 31 - Passos para caso 2 e caso 3}
    \centering
    \tiny
    \begin{tikzpicture}[
      % 
      ->,>=stealth',
      % level/.style={sibling distance = 20em, level distance = 3em},
      level 1/.style={sibling distance=18em, level distance = 3em},
      level 2/.style={sibling distance=8em, level distance = 3em},
      level 3/.style={sibling distance=4em, level distance = 3em},
      level 4/.style={sibling distance=3em, level distance = 3em}]
    \node [bv] (r){41}
    child [color=black] { node [rv] {11}
      child [color=black] {node [bwv] {...}}
      child [color=black] { node [rv] {20}
        child [color=black] {node [bwv] {...}}
        child [color=black] {node [bwv] {...}} 
      }
    }
    child [color=black] { node [bv] {46}
      child [color=black] {node [bwv] {...}}
      child [color=black] {node [bwv] {...}}
    }
      % child [color=black] { node [brv] {13}
      %   child [color=black] {node [bwv] {...}}
      %   child [color=black] {node [bwv] {...}}
      % }
      % child [color=black] { node [brv] {25}
      %   child [color=black] {node [bwv] {...}}
      %   child [color=black] { node [bbv] {30}
      %     child [color=black] {node [bwv] {...}}
      %     child [color=black] {node [bwv] {...}}
      %   }
      % }
    ;
    \end{tikzpicture}
    
  \end{figure}

\end{frame}
\begin{frame}[fragile]

 
  Então faremos os três passos a seguir:
\begin{itemize}
  \item[P1] Rotação de 11 à esquerda de 20
  \item[P2] Rotação de 41 à direita de 20
  \item[P3] Inverter cores de 20 e 41.
\end{itemize}


 
\begin{figure}[H]
  \caption{Conjunto 1 - Inserir 31 - P1}
  % \begin{minipage}{.5\textwidth}
  % \begin{subfigure}{\textwidth}
    \centering
    \tiny
    \begin{tikzpicture}[
      % 
      ->,>=stealth',
      % level/.style={sibling distance = 20em, level distance = 3em},
      level 1/.style={sibling distance=18em, level distance = 3em},
      level 2/.style={sibling distance=8em, level distance = 3em},
      level 3/.style={sibling distance=4em, level distance = 3em},
      level 4/.style={sibling distance=3em, level distance = 3em}]
    \node [bv] (r){41}
    child [color=black] { node [rv] {20}
    child [color=black] { node [rv] {11}
        child [color=black] {node [bwv] {...}}
        child [color=black] {node [bwv] {...}} 
      }
    child [color=black] {node [bwv] {...}}
    }
    child [color=black] { node [bv] {46}
      child [color=black] {node [bwv] {...}}
      child [color=black] {node [bwv] {...}}
    }
      % child [color=black] { node [brv] {13}
      %   child [color=black] {node [bwv] {...}}
      %   child [color=black] {node [bwv] {...}}
      % }
      % child [color=black] { node [brv] {25}
      %   child [color=black] {node [bwv] {...}}
      %   child [color=black] { node [bbv] {30}
      %     child [color=black] {node [bwv] {...}}
      %     child [color=black] {node [bwv] {...}}
      %   }
      % }
    ;
    \end{tikzpicture}
  % \end{subfigure}
    % \end{minipage}
    % \hfill
    % \begin{minipage}{.5\textwidth}
    %   % \begin{subfigure}
    %     \includegraphics[width=\textwidth]{qface.jpg}
    %   % \end{subfigure}
    % \end{minipage}
  \end{figure}

\end{frame}
\begin{frame}[fragile]

 
\begin{figure}[H]
  \caption{Conjunto 1 - Inserir 31 - P2}
  % \begin{minipage}{.5\textwidth}
  % \begin{subfigure}{\textwidth}
    \centering
    \tiny
    \begin{tikzpicture}[
      % 
      ->,>=stealth',
      % level/.style={sibling distance = 20em, level distance = 3em},
      level 1/.style={sibling distance=18em, level distance = 3em},
      level 2/.style={sibling distance=8em, level distance = 3em},
      level 3/.style={sibling distance=4em, level distance = 3em},
      level 4/.style={sibling distance=3em, level distance = 3em}]
    \node [rv] (r){20}
    child [color=black] { node [rv] {11}
        child [color=black] {node [bwv] {...}}
        child [color=black] {node [bwv] {...}} 
      }
    child [color=black] { node [bv] {41}
      child [color=black] {node [bwv] {...}}
      child [color=black] { node [bv] {46}
        child [color=black] {node [bwv] {...}}
        child [color=black] {node [bwv] {...}}
      }
    }
      % child [color=black] { node [brv] {13}
      %   child [color=black] {node [bwv] {...}}
      %   child [color=black] {node [bwv] {...}}
      % }
      % child [color=black] { node [brv] {25}
      %   child [color=black] {node [bwv] {...}}
      %   child [color=black] { node [bbv] {30}
      %     child [color=black] {node [bwv] {...}}
      %     child [color=black] {node [bwv] {...}}
      %   }
      % }
    ;
    \end{tikzpicture}
  % \end{subfigure}
    % \end{minipage}
    % \hfill
    % \begin{minipage}{.5\textwidth}
    %   % \begin{subfigure}
    %     \includegraphics[width=\textwidth]{qface.jpg}
    %   % \end{subfigure}
    % \end{minipage}
  \end{figure}
\end{frame}
\begin{frame}[fragile]

Agora é recolorir. Lembrando que nó raiz é sempre preto.

\begin{figure}[H]
  \caption{Conjunto 1 - Inserir 31 - P3}
  % \begin{minipage}{.5\textwidth}
  % \begin{subfigure}{\textwidth}
    \centering
    \tiny
    \begin{tikzpicture}[
      % 
      ->,>=stealth',
      % level/.style={sibling distance = 20em, level distance = 3em},
      level 1/.style={sibling distance=18em, level distance = 3em},
      level 2/.style={sibling distance=8em, level distance = 3em},
      level 3/.style={sibling distance=4em, level distance = 3em},
      level 4/.style={sibling distance=3em, level distance = 3em}]
    \node [bv] (r){20}
    child [color=black] { node [rv] {11}
        child [color=black] {node [bwv] {...}}
        child [color=black] {node [bwv] {...}} 
      }
    child [color=black] { node [rv] {41}
      child [color=black] {node [bwv] {...}}
      child [color=black] { node [bv] {46}
        child [color=black] {node [bwv] {...}}
        child [color=black] {node [bwv] {...}}
      }
    }
      % child [color=black] { node [brv] {13}
      %   child [color=black] {node [bwv] {...}}
      %   child [color=black] {node [bwv] {...}}
      % }
      % child [color=black] { node [brv] {25}
      %   child [color=black] {node [bwv] {...}}
      %   child [color=black] { node [bbv] {30}
      %     child [color=black] {node [bwv] {...}}
      %     child [color=black] {node [bwv] {...}}
      %   }
      % }
    ;
    \end{tikzpicture}
  % \end{subfigure}
    % \end{minipage}
    % \hfill
    % \begin{minipage}{.5\textwidth}
    %   % \begin{subfigure}
    %     \includegraphics[width=\textwidth]{qface.jpg}
    %   % \end{subfigure}
    % \end{minipage}
  \end{figure}

\end{frame}
\begin{frame}[fragile]

  Vamos voltar toda a árvore?



\begin{figure}[H]
  \caption{Conjunto 1 - Inserir 31 - Finalizado}
    \centering
    \tiny
\begin{tikzpicture}[
  % 
  ->,>=stealth',
  % level/.style={sibling distance = 20em, level distance = 3em},
  level 1/.style={sibling distance=18em, level distance = 3em},
  level 2/.style={sibling distance=8em, level distance = 3em},
  level 3/.style={sibling distance=4em, level distance = 3em},
  level 4/.style={sibling distance=3em, level distance = 3em}]
\node [bv] (r){20}
child [color=black] { node [rv] {11}
  child [color=black] { node [bv] {5}
    child [color=black] { node [rv] {3}
      child [color=black] {node [nil] {}}
      child [color=black] {node [nil] {}}
    }
    child [color=black] { node [rv] {10}
      child [color=black] {node [nil] {}}
      child [color=black] {node [nil] {}}
    }
  }
  child [color=black] { node [bv] {14}
    child [color=black] {node [nil] {}}
    child [color=black] {node [nil] {}}
  }
}
child [color=black] { node [rv] {41}
  child [color=black] { node [bv] {24}
    child [color=black] {node [nil] {}}
    child [color=black] { node [rv] {31}
      child [color=black] {node [nil] {}}
      child [color=black] {node [nil] {}}
    }
  }
  child [color=black] { node [bv] {46}
    child [color=black] { node [rv] {42}
      child [color=black] {node [nil] {}}
      child [color=black] {node [nil] {}}
    }
    child [color=black] {node [nil] {}}
  }
}
; 
\end{tikzpicture}

\end{figure}
\end{frame}
\begin{frame}[fragile]

Ainda temos que inserir o 44, 48, 39 e 21. Vamos lá, começamos pelo 44.



\begin{figure}[H]
  \caption{Conjunto 1 - Inserir 44}
    \centering
    \tiny
\begin{tikzpicture}[
  % 
  ->,>=stealth',
  % level/.style={sibling distance = 20em, level distance = 3em},
  level 1/.style={sibling distance=18em, level distance = 3em},
  level 2/.style={sibling distance=8em, level distance = 3em},
  level 3/.style={sibling distance=4em, level distance = 3em},
  level 4/.style={sibling distance=3em, level distance = 3em}]
\node [bv] (r){20}
child [color=black] { node [rv] {11}
  child [color=black] { node [bv] {5}
    child [color=black] { node [rv] {3}
      child [color=black] {node [nil] {}}
      child [color=black] {node [nil] {}}
    }
    child [color=black] { node [rv] {10}
      child [color=black] {node [nil] {}}
      child [color=black] {node [nil] {}}
    }
  }
  child [color=black] { node [bv] {14}
    child [color=black] {node [nil] {}}
    child [color=black] {node [nil] {}}
  }
}
child [color=black] { node [rv] {41}
  child [color=black] { node [bv] {24}
    child [color=black] {node [nil] {}}
    child [color=black] { node [rv] {31}
      child [color=black] {node [nil] {}}
      child [color=black] {node [nil] {}}
    }
  }
  child [color=black] { node [bv] {46}
    child [color=black] { node [rv] {42}
      child [color=black] {node [nil] {}}
      child [color=black] { node [rv] {44}
        child [color=black] {node [nil] {}}
        child [color=black] {node [nil] {}}
      }
    }
    child [color=black] {node [nil] {}}
  }
}
; 
\end{tikzpicture}

\end{figure}
\end{frame}
\begin{frame}[fragile]

Caso 2 + caso 3, certo? Então faremos os três passos a seguir:
\begin{itemize}
  \item[P1] Rotação de 42 à esquerda de 44
  \item[P2] Rotação de 46 à direita de 44
  \item[P3] Inverter cores de 44 e 46.
\end{itemize}


\begin{figure}[H]
  \caption{Conjunto 1 - Inserir 44 - Correção de balanceamento P1}
    \centering
    \tiny
\begin{tikzpicture}[
  % 
  ->,>=stealth',
  % level/.style={sibling distance = 20em, level distance = 3em},
  level 1/.style={sibling distance=18em, level distance = 3em},
  level 2/.style={sibling distance=8em, level distance = 3em},
  level 3/.style={sibling distance=4em, level distance = 3em},
  level 4/.style={sibling distance=3em, level distance = 3em}]
\node [bv] (r){20}
child [color=black] { node [rv] {11}
  child [color=black] { node [bv] {5}
    child [color=black] { node [rv] {3}
      child [color=black] {node [nil] {}}
      child [color=black] {node [nil] {}}
    }
    child [color=black] { node [rv] {10}
      child [color=black] {node [nil] {}}
      child [color=black] {node [nil] {}}
    }
  }
  child [color=black] { node [bv] {14}
    child [color=black] {node [nil] {}}
    child [color=black] {node [nil] {}}
  }
}
child [color=black] { node [rv] {41}
  child [color=black] { node [bv] {24}
    child [color=black] {node [nil] {}}
    child [color=black] { node [rv] {31}
      child [color=black] {node [nil] {}}
      child [color=black] {node [nil] {}}
    }
  }
  child [color=black] { node [bv] {46}
    child [color=black] { node [rv] {44}
      child [color=black] { node [rv] {42}
        child [color=black] {node [nil] {}}
        child [color=black] {node [nil] {}}
      }
      child [color=black] {node [nil] {}}
    }
    child [color=black] {node [nil] {}}
  }
}
; 
\end{tikzpicture}

\end{figure}
\end{frame}
\begin{frame}[fragile]

Rotação de 46 à direita de 44.

\begin{figure}[H]
  \caption{Conjunto 1 - Inserir 44 - Correção de balanceamento P2}
    \centering
    \tiny
\begin{tikzpicture}[
  % 
  ->,>=stealth',
  % level/.style={sibling distance = 20em, level distance = 3em},
  level 1/.style={sibling distance=18em, level distance = 3em},
  level 2/.style={sibling distance=8em, level distance = 3em},
  level 3/.style={sibling distance=4em, level distance = 3em},
  level 4/.style={sibling distance=3em, level distance = 3em}]
\node [bv] (r){20}
child [color=black] { node [rv] {11}
  child [color=black] { node [bv] {5}
    child [color=black] { node [rv] {3}
      child [color=black] {node [nil] {}}
      child [color=black] {node [nil] {}}
    }
    child [color=black] { node [rv] {10}
      child [color=black] {node [nil] {}}
      child [color=black] {node [nil] {}}
    }
  }
  child [color=black] { node [bv] {14}
    child [color=black] {node [nil] {}}
    child [color=black] {node [nil] {}}
  }
}
child [color=black] { node [rv] {41}
  child [color=black] { node [bv] {24}
    child [color=black] {node [nil] {}}
    child [color=black] { node [rv] {31}
      child [color=black] {node [nil] {}}
      child [color=black] {node [nil] {}}
    }
  }
  child [color=black] { node [rv] {44}
    child [color=black] { node [rv] {42}
      child [color=black] {node [nil] {}}
      child [color=black] {node [nil] {}}
    }
    child [color=black] { node [bv] {46}
      child [color=black] {node [nil] {}}
      child [color=black] {node [nil] {}}
    }
  }
}
; 
\end{tikzpicture}

\end{figure}
\end{frame}
\begin{frame}[fragile]


Inverter as cores de 46 e 44.

\begin{figure}[H]
  \caption{Conjunto 1 - Inserir 44 - Correção de balanceamento P3}
    \centering
    \tiny
\begin{tikzpicture}[
  % 
  ->,>=stealth',
  % level/.style={sibling distance = 20em, level distance = 3em},
  level 1/.style={sibling distance=18em, level distance = 3em},
  level 2/.style={sibling distance=8em, level distance = 3em},
  level 3/.style={sibling distance=4em, level distance = 3em},
  level 4/.style={sibling distance=3em, level distance = 3em}]
\node [bv] (r){20}
child [color=black] { node [rv] {11}
  child [color=black] { node [bv] {5}
    child [color=black] { node [rv] {3}
      child [color=black] {node [nil] {}}
      child [color=black] {node [nil] {}}
    }
    child [color=black] { node [rv] {10}
      child [color=black] {node [nil] {}}
      child [color=black] {node [nil] {}}
    }
  }
  child [color=black] { node [bv] {14}
    child [color=black] {node [nil] {}}
    child [color=black] {node [nil] {}}
  }
}
child [color=black] { node [rv] {41}
  child [color=black] { node [bv] {24}
    child [color=black] {node [nil] {}}
    child [color=black] { node [rv] {31}
      child [color=black] {node [nil] {}}
      child [color=black] {node [nil] {}}
    }
  }
  child [color=black] { node [bv] {44}
    child [color=black] { node [rv] {42}
      child [color=black] {node [nil] {}}
      child [color=black] {node [nil] {}}
    }
    child [color=black] { node [rv] {46}
      child [color=black] {node [nil] {}}
      child [color=black] {node [nil] {}}
    }
  }
}
; 
\end{tikzpicture}

\end{figure}
\end{frame}
\begin{frame}[fragile]

Tudo balanceado né? Inseriremos o 48 agora.


\begin{figure}[H]
  \caption{Conjunto 1 - Inserir 48}
    \centering
    \tiny
\begin{tikzpicture}[
  % 
  ->,>=stealth',
  % level/.style={sibling distance = 20em, level distance = 3em},
  level 1/.style={sibling distance=18em, level distance = 3em},
  level 2/.style={sibling distance=8em, level distance = 3em},
  level 3/.style={sibling distance=4em, level distance = 3em},
  level 4/.style={sibling distance=3em, level distance = 3em}]
\node [bv] (r){20}
child [color=black] { node [rv] {11}
  child [color=black] { node [bv] {5}
    child [color=black] { node [rv] {3}
      child [color=black] {node [nil] {}}
      child [color=black] {node [nil] {}}
    }
    child [color=black] { node [rv] {10}
      child [color=black] {node [nil] {}}
      child [color=black] {node [nil] {}}
    }
  }
  child [color=black] { node [bv] {14}
    child [color=black] {node [nil] {}}
    child [color=black] {node [nil] {}}
  }
}
child [color=black] { node [rv] {41}
  child [color=black] { node [bv] {24}
    child [color=black] {node [nil] {}}
    child [color=black] { node [rv] {31}
      child [color=black] {node [nil] {}}
      child [color=black] {node [nil] {}}
    }
  }
  child [color=black] { node [bv] {44}
    child [color=black] { node [rv] {42}
      child [color=black] {node [nil] {}}
      child [color=black] {node [nil] {}}
    }
    child [color=black] { node [rv] {46}
      child [color=black] {node [nil] {}}
      child [color=black] { node [rv] {48}
        child [color=black] {node [nil] {}}
        child [color=black] {node [nil] {}}
      }
    }
  }
}
; 
\end{tikzpicture}

\end{figure}
\end{frame}
\begin{frame}[fragile]

Aqui beleza, caso 1. Tio vermelho, colorir.



\begin{figure}[H]
  \caption{Conjunto 1 - Inserir 48}
    \centering
    \tiny
\begin{tikzpicture}[
  % 
  ->,>=stealth',
  % level/.style={sibling distance = 20em, level distance = 3em},
  level 1/.style={sibling distance=18em, level distance = 3em},
  level 2/.style={sibling distance=8em, level distance = 3em},
  level 3/.style={sibling distance=4em, level distance = 3em},
  level 4/.style={sibling distance=3em, level distance = 3em}]
\node [bv] (r){20}
child [color=black] { node [rv] {11}
  child [color=black] { node [bv] {5}
    child [color=black] { node [rv] {3}
      child [color=black] {node [nil] {}}
      child [color=black] {node [nil] {}}
    }
    child [color=black] { node [rv] {10}
      child [color=black] {node [nil] {}}
      child [color=black] {node [nil] {}}
    }
  }
  child [color=black] { node [bv] {14}
    child [color=black] {node [nil] {}}
    child [color=black] {node [nil] {}}
  }
}
child [color=black] { node [rv] {41}
  child [color=black] { node [bv] {24}
    child [color=black] {node [nil] {}}
    child [color=black] { node [rv] {31}
      child [color=black] {node [nil] {}}
      child [color=black] {node [nil] {}}
    }
  }
  child [color=black] { node [rv] {44}
    child [color=black] { node [bv] {42}
      child [color=black] {node [nil] {}}
      child [color=black] {node [nil] {}}
    }
    child [color=black] { node [bv] {46}
      child [color=black] {node [nil] {}}
      child [color=black] { node [rv] {48}
        child [color=black] {node [nil] {}}
        child [color=black] {node [nil] {}}
      }
    }
  }
}
; 
\end{tikzpicture}

\end{figure}
\end{frame}
\begin{frame}[fragile]

Tá achando que vai ter que balancear tudo de novo? Não!!! Olha que o tio agora é 11, e ele é vermelho. Então é só inverter as cores novamente do 11, 20 e 41. Só que como o 20 é raiz, ele vai ser sempre preto.



\begin{figure}[H]
  \caption{Conjunto 1 - Inserir 48}
    \centering
    \tiny
\begin{tikzpicture}[
  % 
  ->,>=stealth',
  % level/.style={sibling distance = 20em, level distance = 3em},
  level 1/.style={sibling distance=18em, level distance = 3em},
  level 2/.style={sibling distance=8em, level distance = 3em},
  level 3/.style={sibling distance=4em, level distance = 3em},
  level 4/.style={sibling distance=3em, level distance = 3em}]
\node [bv] (r){20}
child [color=black] { node [bv] {11}
  child [color=black] { node [bv] {5}
    child [color=black] { node [rv] {3}
      child [color=black] {node [nil] {}}
      child [color=black] {node [nil] {}}
    }
    child [color=black] { node [rv] {10}
      child [color=black] {node [nil] {}}
      child [color=black] {node [nil] {}}
    }
  }
  child [color=black] { node [bv] {14}
    child [color=black] {node [nil] {}}
    child [color=black] {node [nil] {}}
  }
}
child [color=black] { node [bv] {41}
  child [color=black] { node [bv] {24}
    child [color=black] {node [nil] {}}
    child [color=black] { node [rv] {31}
      child [color=black] {node [nil] {}}
      child [color=black] {node [nil] {}}
    }
  }
  child [color=black] { node [rv] {44}
    child [color=black] { node [bv] {42}
      child [color=black] {node [nil] {}}
      child [color=black] {node [nil] {}}
    }
    child [color=black] { node [bv] {46}
      child [color=black] {node [nil] {}}
      child [color=black] { node [rv] {48}
        child [color=black] {node [nil] {}}
        child [color=black] {node [nil] {}}
      }
    }
  }
}
; 
\end{tikzpicture}

\end{figure}
\end{frame}
\begin{frame}[fragile]

Vamos inserir o 39.


\begin{figure}[H]
  \caption{Conjunto 1 - Inserir 39}
    \centering
    \tiny
\begin{tikzpicture}[
  % 
  ->,>=stealth',
  % level/.style={sibling distance = 20em, level distance = 3em},
  level 1/.style={sibling distance=18em, level distance = 3em},
  level 2/.style={sibling distance=8em, level distance = 3em},
  level 3/.style={sibling distance=4em, level distance = 3em},
  level 4/.style={sibling distance=3em, level distance = 3em}]
\node [bv] (r){20}
child [color=black] { node [bv] {11}
  child [color=black] { node [bv] {5}
    child [color=black] { node [rv] {3}
      child [color=black] {node [nil] {}}
      child [color=black] {node [nil] {}}
    }
    child [color=black] { node [rv] {10}
      child [color=black] {node [nil] {}}
      child [color=black] {node [nil] {}}
    }
  }
  child [color=black] { node [bv] {14}
    child [color=black] {node [nil] {}}
    child [color=black] {node [nil] {}}
  }
}
child [color=black] { node [bv] {41}
  child [color=black] { node [bv] {24}
    child [color=black] {node [nil] {}}
    child [color=black] { node [rv] {31}
      child [color=black] {node [nil] {}}
      child [color=black] { node [rv] {39}
        child [color=black] {node [nil] {}}
        child [color=black] {node [nil] {}}
      }
    }
  }
  child [color=black] { node [rv] {44}
    child [color=black] { node [bv] {42}
      child [color=black] {node [nil] {}}
      child [color=black] {node [nil] {}}
    }
    child [color=black] { node [bv] {46}
      child [color=black] {node [nil] {}}
      child [color=black] { node [rv] {48}
        child [color=black] {node [nil] {}}
        child [color=black] {node [nil] {}}
      }
    }
  }
}
; 
\end{tikzpicture}

\end{figure}
\end{frame}
\begin{frame}[fragile]

Veja que vai ter um tio preto, e se encaixa direto no caso 3. Então primeiramente vamos fazer (P1) uma rotação simples à esquerda de 24 ao redor de 31. Depois (P2) inverter as cores de 31 e 24.


\begin{figure}[H]
  \caption{Conjunto 1 - Inserir 39 - P1}
    \centering
    \tiny
\begin{tikzpicture}[
  % 
  ->,>=stealth',
  % level/.style={sibling distance = 20em, level distance = 3em},
  level 1/.style={sibling distance=18em, level distance = 3em},
  level 2/.style={sibling distance=8em, level distance = 3em},
  level 3/.style={sibling distance=4em, level distance = 3em},
  level 4/.style={sibling distance=3em, level distance = 3em}]
\node [bv] (r){20}
child [color=black] { node [bv] {11}
  child [color=black] { node [bv] {5}
    child [color=black] { node [rv] {3}
      child [color=black] {node [nil] {}}
      child [color=black] {node [nil] {}}
    }
    child [color=black] { node [rv] {10}
      child [color=black] {node [nil] {}}
      child [color=black] {node [nil] {}}
    }
  }
  child [color=black] { node [bv] {14}
    child [color=black] {node [nil] {}}
    child [color=black] {node [nil] {}}
  }
}
child [color=black] { node [bv] {41}
  child [color=black] { node [rv] {31}
      child [color=black] { node [bv] {24}
          child [color=black] {node [nil] {}}
          child [color=black] {node [nil] {}}
      }
      child [color=black] { node [rv] {39}
        child [color=black] {node [nil] {}}
        child [color=black] {node [nil] {}}
      }
  }
  child [color=black] { node [rv] {44}
    child [color=black] { node [bv] {42}
      child [color=black] {node [nil] {}}
      child [color=black] {node [nil] {}}
    }
    child [color=black] { node [bv] {46}
      child [color=black] {node [nil] {}}
      child [color=black] { node [rv] {48}
        child [color=black] {node [nil] {}}
        child [color=black] {node [nil] {}}
      }
    }
  }
}
; 
\end{tikzpicture}

\end{figure}
\end{frame}
\begin{frame}[fragile]

Agora passo 2


\begin{figure}[H]
  \caption{Conjunto 1 - Inserir 39 - P2}
    \centering
    \tiny
\begin{tikzpicture}[
  % 
  ->,>=stealth',
  % level/.style={sibling distance = 20em, level distance = 3em},
  level 1/.style={sibling distance=18em, level distance = 3em},
  level 2/.style={sibling distance=8em, level distance = 3em},
  level 3/.style={sibling distance=4em, level distance = 3em},
  level 4/.style={sibling distance=3em, level distance = 3em}]
\node [bv] (r){20}
child [color=black] { node [bv] {11}
  child [color=black] { node [bv] {5}
    child [color=black] { node [rv] {3}
      child [color=black] {node [nil] {}}
      child [color=black] {node [nil] {}}
    }
    child [color=black] { node [rv] {10}
      child [color=black] {node [nil] {}}
      child [color=black] {node [nil] {}}
    }
  }
  child [color=black] { node [bv] {14}
    child [color=black] {node [nil] {}}
    child [color=black] {node [nil] {}}
  }
}
child [color=black] { node [bv] {41}
  child [color=black] { node [bv] {31}
      child [color=black] { node [rv] {24}
          child [color=black] {node [nil] {}}
          child [color=black] {node [nil] {}}
      }
      child [color=black] { node [rv] {39}
        child [color=black] {node [nil] {}}
        child [color=black] {node [nil] {}}
      }
  }
  child [color=black] { node [rv] {44}
    child [color=black] { node [bv] {42}
      child [color=black] {node [nil] {}}
      child [color=black] {node [nil] {}}
    }
    child [color=black] { node [bv] {46}
      child [color=black] {node [nil] {}}
      child [color=black] { node [rv] {48}
        child [color=black] {node [nil] {}}
        child [color=black] {node [nil] {}}
      }
    }
  }
}
; 
\end{tikzpicture}

\end{figure}
\end{frame}
\begin{frame}[fragile]


Finalmente, vamos inserir o 21.

\begin{figure}[H]
  \caption{Conjunto 1 - Inserir 21}
    \centering
    \tiny
\begin{tikzpicture}[
  % 
  ->,>=stealth',
  % level/.style={sibling distance = 20em, level distance = 3em},
  level 1/.style={sibling distance=18em, level distance = 3em},
  level 2/.style={sibling distance=8em, level distance = 3em},
  level 3/.style={sibling distance=4em, level distance = 3em},
  level 4/.style={sibling distance=3em, level distance = 3em}]
\node [bv] (r){20}
child [color=black] { node [bv] {11}
  child [color=black] { node [bv] {5}
    child [color=black] { node [rv] {3}
      child [color=black] {node [nil] {}}
      child [color=black] {node [nil] {}}
    }
    child [color=black] { node [rv] {10}
      child [color=black] {node [nil] {}}
      child [color=black] {node [nil] {}}
    }
  }
  child [color=black] { node [bv] {14}
    child [color=black] {node [nil] {}}
    child [color=black] {node [nil] {}}
  }
}
child [color=black] { node [bv] {41}
  child [color=black] { node [bv] {31}
      child [color=black] { node [rv] {24}
        child [color=black] { node [rv] {21}
            child [color=black] {node [nil] {}}
            child [color=black] {node [nil] {}}
        }
          child [color=black] {node [nil] {}}
      }
      child [color=black] { node [rv] {39}
        child [color=black] {node [nil] {}}
        child [color=black] {node [nil] {}}
      }
  }
  child [color=black] { node [rv] {44}
    child [color=black] { node [bv] {42}
      child [color=black] {node [nil] {}}
      child [color=black] {node [nil] {}}
    }
    child [color=black] { node [bv] {46}
      child [color=black] {node [nil] {}}
      child [color=black] { node [rv] {48}
        child [color=black] {node [nil] {}}
        child [color=black] {node [nil] {}}
      }
    }
  }
}
; 
\end{tikzpicture}

\end{figure}
\end{frame}
\begin{frame}[fragile]


Tio vermelho, colorir!


\begin{figure}[H]
  \caption{Conjunto 1 - Inserir 21 - Caso 1}
    \centering
    \tiny
\begin{tikzpicture}[
  % 
  ->,>=stealth',
  % level/.style={sibling distance = 20em, level distance = 3em},
  level 1/.style={sibling distance=18em, level distance = 3em},
  level 2/.style={sibling distance=8em, level distance = 3em},
  level 3/.style={sibling distance=4em, level distance = 3em},
  level 4/.style={sibling distance=3em, level distance = 3em}]
\node [bv] (r){20}
child [color=black] { node [bv] {11}
  child [color=black] { node [bv] {5}
    child [color=black] { node [rv] {3}
      child [color=black] {node [nil] {}}
      child [color=black] {node [nil] {}}
    }
    child [color=black] { node [rv] {10}
      child [color=black] {node [nil] {}}
      child [color=black] {node [nil] {}}
    }
  }
  child [color=black] { node [bv] {14}
    child [color=black] {node [nil] {}}
    child [color=black] {node [nil] {}}
  }
}
child [color=black] { node [bv] {41}
  child [color=black] { node [rv] {31}
      child [color=black] { node [bv] {24}
        child [color=black] { node [rv] {21}
            child [color=black] {node [nil] {}}
            child [color=black] {node [nil] {}}
        }
          child [color=black] {node [nil] {}}
      }
      child [color=black] { node [bv] {39}
        child [color=black] {node [nil] {}}
        child [color=black] {node [nil] {}}
      }
  }
  child [color=black] { node [rv] {44}
    child [color=black] { node [bv] {42}
      child [color=black] {node [nil] {}}
      child [color=black] {node [nil] {}}
    }
    child [color=black] { node [bv] {46}
      child [color=black] {node [nil] {}}
      child [color=black] { node [rv] {48}
        child [color=black] {node [nil] {}}
        child [color=black] {node [nil] {}}
      }
    }
  }
}
; 
\end{tikzpicture}

\end{figure}
\end{frame}
\section*{Exercícios}
\begin{frame}[fragile]


Ufa! Agora, pegue seus exercícios da AVL e compara se essa árvore tá parecida com a que você fez.


Agora pegue aquelas mesmas árvores, e faça como árvores rubro-negra. Submissão online (Google Classroom) dia 14/04 às 23:59.

\begin{itemize}
  \item Conjunto 2: 4, 39, 30, 27, 41, 46, 47, 10, 31, 8, 17, 12, 50, 2, 14
  \item Conjunto 3: 24, 50, 44, 15, 43, 35, 40, 14, 48, 45, 36, 21, 37, 32, 31
  \item Conjunto 4: 30, 9, 19, 42, 7, 10, 44, 40, 31, 38, 35, 34, 26, 33, 1
  \item Conjunto 5: 6, 29, 5, 9, 16, 28, 49, 36, 22, 39, 27, 37, 1, 4, 48
\end{itemize}
\end{frame}