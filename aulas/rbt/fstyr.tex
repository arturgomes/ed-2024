\documentclass{beamer}
\usetheme[pageofpages=of,% String used between the current page and the
                         % total page count.
          alternativetitlepage=true,% Use the fancy title page.
         % titlepagelogo=trinity-stacked-full,% Logo for the first page.
          watermark=trinity-stacked,% Watermark used in every page.
          watermarkheight=100px,% Height of the watermark.
          watermarkheightmult=4,% The watermark image is 4 times bigger
                                % than watermarkheight.
          ]{Torino}
% Nouvelle is a green and red alternative to the chameleon color theme.
\usecolortheme{nouvelle}%
\usepackage{graphicx} % Allows including images
\usepackage{booktabs} % Allows the use of \toprule, \midrule and \bottomrule in tables
\usepackage{amssymb}
\usepackage{amsmath}
\usepackage{gensymb}
\usepackage{epstopdf}
\usepackage{listings}
\usepackage{verbatim}
\usepackage[active,tightpage]{preview}
\PreviewEnvironment{tikzpicture}

\usepackage{tikz}
\usepackage{amssymb}
\usetikzlibrary{arrows,positioning}
\usepackage[english]{babel}
\usepackage{amsmath}

% \usepackage{hyperref}
% \usepackage{HD}
\usepackage[color]{circus}

\newcommand*{\fcourier}{\fontfamily{pcr}\selectfont}
\DeclareTextFontCommand{\tco}{\fcourier}
% Haemodialysis macro definitions
%
\def\HMCS#1{[HMCS{#1}]}
\def\R#1{\textbf{R-{#1}}}
\def\SS#1{\textbf{S-{#1}}}
%listing CSP code
\newenvironment{mylisting}
{\begin{list}{}{\setlength{\leftmargin}{1em}}\item\scriptsize\bfseries}
{\end{list}}
\newenvironment{mytinylisting}
{\begin{list}{}{\setlength{\leftmargin}{1em}}\item\tiny\bfseries}
{\end{list}}
\newcommand{\CSPM}{$CSP_M$}
\def\Circus{{\sf\slshape{Circus}}}
\def\CRefine{{\sf\slshape{CRefine}}}
\def\JCircus{{\sf\slshape{JCircus}}}
\def\CircusT{{\sf\slshape{Circus~Time}}}

%et al. in italic
\newcommand{\etal}{\textit{et al}. }
\newcommand{\ie}{\textit{i}.\textit{e}., }
\newcommand{\eg}{\textit{e}.\textit{g}. }
% drafting notes
% swap comment around to make them disappear/appear
%\def\DRAFT#1{~\newline\textbf{DRAFT:}\textsc{{#1}}\textbf{:END}\newline~}
\def\DRAFT#1{}
%

\newcounter{examplecounter}
\newenvironment{lawn}[1][]{%
    \refstepco unter{bexamplecounter}%
  \textbf{Law \arabic{examplecounter} (#1)}%
  \quad
}{%
}
% \numberwithin{examplecounter}{section}
\newenvironment{prov}
  {\newline\noindent\ provided }
  {}
\newenvironment{provs}
  {\newline\noindent\ provided\begin{itemize}}
  {\end{itemize}}
% \mode<presentation> {
%  \usetheme{Frankfurt}
% }


%----------------------------------------------------------------------------------------
% TITLE PAGE
%----------------------------------------------------------------------------------------


\begin{document}
\title{A Verified Translation from \Circus\ to \CSPM}

\author{Artur O. Gomes}
\institute{ Trinity College Dublin \\School of Computer Science and Statistics \\Lero, the Irish Software Research Centre \\\url{gomesa@tcd.ie} }

\date{\today} % Date, can be changed to a custom date

\begin{frame}
\titlepage
\end{frame}


\begin{frame}
\frametitle{Motivation}
\begin{itemize}
  \item Manual work on formal validation of systems is time consuming 
  and it can be as error-prone as is the software to be validated.
\end{itemize}

\end{frame}

\begin{frame}
\frametitle{Motivation}
\begin{itemize}
  \item Why a Verified Translation Tool?
  \begin{itemize}
    \item No tool support for model checking \Circus\ directly
    \item Manual translation into CSP is hard
    \item Model checking \Circus\ is very often performed by the \Circus\ community.
    \item Verification Grand Challenge (VGC)~\footnote{Woodcock, Jim, and Richard Banach. "The Verification Grand Challenge." J. UCS 13.5 (2007): 661-668.}: "\emph{Build an integrated suite of tools to support all aspects of verified software construction: requirements capture, specification, validation, test case generation, refinement, analysis, verification, and run time checking}"
  \end{itemize}
\end{itemize}

\end{frame}
\begin{frame}
\frametitle{Why is it so important?}
\begin{itemize}
  \item The Haemodialysis case study~\footnote{Gomes, A.O., Butterfield, A. "Modelling the Haemodialysis Machine with Circus."
In: ABZ’16. LNCS, vol. 9675, pp. 409–-424. Springer (2016)}
  \begin{itemize}
    \item \Circus\ specification of a Haemodialysis machine
    \item Model checking using FDR\footnote{Gibson-Robinson, Thomas, et al. "FDR3-—A Modern Refinement Checker for CSP." \emph{International Conference on Tools and Algorithms for the Construction and Analysis of Systems}. Springer Berlin Heidelberg, 2014.}
    \item Translation from \Circus\ to CSP
      \begin{itemize}
        \item Not always straigthforward
        \item Time consuming task
      \end{itemize}
  \item From our conclusions:
      \begin{itemize}
        \item The manual translation is not easy for large systems
        \item An automatic tool would be useful for this task.
      \end{itemize}
  \end{itemize}
\end{itemize}

\end{frame}


\begin{frame}
\frametitle{Research Question}
\begin{itemize}
  \item How do we verify a Haskell implementation of the translation tool?
  \item Why Haskell? Does it makes the task easier?
  \item What precisely means for the translator to be correct?
  \item What is the relationship of the translation tool in development here and the arguments in Oliveira \emph{et. al.}\footnote{Oliveira, M.V.M., Sampaio, A.C.A., Antonino, P.R.G., Ramos, R.T., Cavalcanti,
A.L.C., Woodcock, J.C.P.: \emph{Compositional Analysis and Design of CML Models}. Tech. Rep. D24.1, COMPASS Deliverable (2013)} about the correctness of their translation scheme?
\end{itemize}

\end{frame}

\begin{frame}
\frametitle{How do we translate \Circus\ into CSP?}

\begin{figure}[htbp]
  \centering
    \includegraphics[width=0.8\textheight]{figs/circus-translation}
  \label{phd-table}
 \end{figure}
\end{frame}

\begin{frame}
\frametitle{Translating \Circus\ into CSP - An overview}
\begin{itemize}
  \item Transform a \Circus\ state into a CSP process offering to update and retrieve the values of its components
  \item \Circus\ processes and actions are translated into CSP
\end{itemize}
\end{frame}

% \begin{frame}
% \frametitle{Can we transliterate 100\% of \Circus?}
% \begin{itemize}
%   \item No.
%   \item Only a subset of \Circus\ can be translated
%   \item A few constructs are not equivalent
%   \begin{itemize}
%     \item Example: in the action $c?x \rightarrow A(x)$, the value of $x$ can not be updated in $A(x)$.
%   \end{itemize}
%   \item Guarded commands and some Z constructs, such as schemas, are not directly translated into CSP
% \end{itemize}
% \end{frame}

\begin{frame}
\frametitle{Can we transliterate 100\% of \Circus?}
\begin{itemize}
  \item A few constructs are not equivalent
  \item Example: in the \Circus\ action $c?x \rightarrow A(x)$.
  \begin{itemize}
    \item In Circus, it is the same of: $\Box~c.k \rightarrow x:= k ; A$ 
    \item In CSP, it means: $\Box~c.k \then A[k/x]$
  \end{itemize}
  \item Another example is a prefix followed by a sequential composition.
  \begin{itemize}
    \item \Circus: $e \rightarrow~P~;~Q~=~e~\rightarrow~(P;Q)$
    \item CSP: $(e~\rightarrow~P);Q~\neq~e~\rightarrow~(P;Q)$
  \end{itemize}
  \item Guarded commands and some Z constructs, such as schemas, are translated into CSP
\end{itemize}
\end{frame}

\begin{frame}
\frametitle{Implementing the Tool}
\begin{itemize}
  \item Based on work from Oliveira \emph{et. al.}
  \begin{itemize}
    \item[I] We transform staterich \Circus\ processes into stateless processes, with help of the refinement laws of \Circus.
    \item[II] Then we translate stateless \Circus\ processes into CSP
  \end{itemize}
  \item Built in Haskell in an extended version of JAZA that parses \Circus\ specification written in \LaTeX
  \end{itemize}
\end{frame}

\begin{frame}[fragile]
\frametitle{Implementing the Tool}
\begin{figure}[htbp]
  \centering
    \includegraphics[width=0.6\textwidth]{figs/circus-to-csp-diagram}
  \caption{Mapping \Circus\ into \CSPM}
 \end{figure}
 Where
 \begin{itemize}
  \item $Circus_{SR}$: with staterich processes
  \item $Circus_{SL}$: staterich processes are translated into stateless processes
  \item $Circus_{CSP}$: subset of \Circus\ ready for mapping into \CSPM\
\end{itemize}
\end{frame}

\begin{frame}
\frametitle{Tasks for the PhD - What we've done}
\begin{itemize}
  \item Implementing the parser - Extending JAZA
  \item Implementing part of the refinement laws
  \item Implementing the $\Omega_A$ transformation functions towards a stateless process
  \item Implementing the $\Upsilon$ mapping functions from \Circus\ to CSP
  \end{itemize}
\end{frame}
\begin{frame}[fragile]
\frametitle{Refinement laws}
Variable block introduction law example
   \begin{circus}
       A = \circvar\ x:T \circspot A %
   \end{circus}%
   \vspace{-20px}
   \begin{prov}
       $x \notin FV(A)$
   \end{prov}
\linebreak \linebreak is implemented in Haskell as
\begin{verbatim}
crl_variableBlockIntroduction a x t
  = case p1 of
      True -> CActionCommand (CVarDecl [(Choose x t)] a)
      False -> a
    where
      p1 = not (elem x (getFV a))
\end{verbatim}
\end{frame}
\begin{frame}
\frametitle{Implementing the $\Omega_A$ transformation functions}
Some examples...
\small
\begin{circus}
\Omega_A (\Skip) \circdef \Skip
\also \Omega_A (c \then A) \circdef c \then \Omega_A (A)
\also \Omega_A (A_1 \extchoice A_2) \circdef
\\\t1 get.v_0?vv_0 \then \ldots \then get.v_n?vv_n \then
\\\t1 get.l_0?vl_0 \then \ldots \then get.l_m?vl_m \then
\\\t2 (\Omega'_A (A_1) \extchoice \Omega'_A (A_2))
\also \Omega_A (\Semi x : \langle v_1,...,v_n \rangle \circspot A(x)) \circdef \Omega_A (A(v_1)\circseq \ldots \circseq A(v_n))
\also \Omega_A (\Extchoice x : \langle v_1,...,v_n \rangle \circspot A(x)) \circdef \Omega_A (A(v_1)\extchoice \ldots \extchoice A(v_n))
\end{circus}
\end{frame}

\begin{frame}[fragile]
\frametitle{Implementing the $\Omega_A$ transformation functions}
We illustrate how the $\Omega$ functions
\small
\begin{circus}
\Omega_A (\Skip) \circdef \Skip
\also \Omega_A (c \then A) \circdef c \then \Omega_A (A)
\end{circus}
\normalsize
are implemented in Haskell as
\small
\begin{verbatim}
omega_CActions :: CAction -> CAction
omega_CActions CSPSkip = CSPSkip
omega_CActions (CSPCommAction (ChanComm c []) a)
  = (CSPCommAction (ChanComm c []) (omega_CActions a))
\end{verbatim}
\end{frame}

\begin{frame}
\frametitle{Implementing the $\Upsilon$ mapping functions}
Some examples...
\small
\begin{circus}
\Upsilon_A(\Skip) \defs~\tco{SKIP}
\also \Upsilon_A(c!v \then\ A)\circdef~\tco{c!v -> }\Upsilon_A(A)
\also \Upsilon_A(A \extchoice B) \circdef~\Upsilon_A(A) ~\tco{[]} \Upsilon_A(B)
\also \Upsilon_A(\Semi x : S \circspot A)\circdef~\tco{; x :}\Upsilon_{seq}(S)~\tco{@}~\Upsilon_A(A)
\also \Upsilon_A(\Extchoice x : S \circspot A)\circdef~\tco{[] x :}~\Upsilon_{\mathbb{P}}(S)~\tco{@}~\Upsilon_A(A)
\end{circus}
\end{frame}

\begin{frame}[fragile]
\frametitle{Implementing the $\Upsilon$ mapping functions}
We then illustrate how the $\Upsilon$ functions are implemented
\small
\begin{circus}
\Upsilon_A(c\then\ A)\circdef~\tco{c -> }\Upsilon_A(A)
\end{circus}
\begin{verbatim}
mapping_CAction :: [ZPara] -> CAction -> ZName
mapping_CAction spec (CSPCommAction c a)
  = (get_channel_name c)
    ++ " -> " ++ mapping_CAction spec (a)
\end{verbatim}
\vspace{-20px}
\begin{circus}
\Upsilon_A(A \extchoice B) \circdef~\Upsilon_A(A) ~\tco{[]} \Upsilon_A(B)
\end{circus}
\begin{verbatim}
mapping_CAction spec (CSPExtChoice a b)
  = mapping_CAction spec (a)
    ++ " [] " ++ mapping_CAction spec (b)
\end{verbatim}
\end{frame}

\begin{frame}
\frametitle{Tasks for the PhD - In execution}
\begin{itemize}
  \item Implementing some auxiliary functions - part of the $\Omega_P$ transformation towards the stateless process
  \item Integrating both $\Omega$ and $\Upsilon$ functions into JAZA
  \item First round of tests - Using the Haemodialysis case study~\footnote{Gomes, A.O., Butterfield, A. "Modelling the Haemodialysis Machine with Circus."In: ABZ’16. LNCS, vol. 9675, pp. 409–-424. Springer (2016)} and the ARINC 653~\footnote{Oliveira Gomes, Artur. "Formal Specification of the ARINC 653 Architecture Using Circus." (2012).} \Circus\ specifications
  \end{itemize}
\end{frame}



\begin{frame}
\frametitle{Tasks for the PhD - $\Omega_P$ transformation}
The staterich process P,
\begin{figure}[htbp]
  % \centering
    \includegraphics[width=0.5\textwidth]{figs/staterich}
 \end{figure}
 is then transformed into 
 \begin{figure}[htbp]
  % \centering
    \includegraphics[width=0.5\textwidth]{figs/statelesss}
 \end{figure}
\end{frame}

\begin{frame}
\frametitle{Tasks for the PhD - Next steps}
\begin{itemize}
 \item Verifying the tool with Haskabelle\footnote{Rittweiler, Tobias, and Florian Haftmann. "Haskabelle--converting Haskell source files to Isabelle/HOL theories." (2013).} - Isabelle/HOL
  \begin{itemize}
    \item Defining the specification of the translation to be captured in Isabelle/HOL
    \item Establishing the theorem of correctness of the implementation
    \item Defining the refinement relation between laws and translations from Oliveira \emph{et. al.},
and the corresponding Haskell code that implements those translations
  \end{itemize} 
  \item Second round of tests after the verification process of the tool
  \end{itemize}
\end{frame}

\begin{frame}
\frametitle{Verifying the tool with Haskabelle}
\begin{itemize}
  \item Run Haskabelle to get Isabelle/HOL version of Haskell AST type.
  \item Give abstract definition of $\Omega$ using this type.
  \begin{itemize}
    \item Encode refinement laws in Isabelle/HOL
  \end{itemize}
  \item Incrementally work on translation
  \item Another contribution: Formalising the translation specification in Isabelle/HOL
\end{itemize}
\begin{tikzpicture}[->,>=stealth',shorten >=1pt,auto,node distance=2cm,
                thick,main node/.style={rectangle,draw,font=\sffamily\small    \bfseries}]  

\node[main node] (1) {$P_{SR}$};
\node[main node] (2) [right = of 1] {$(P'_{SL},MEM)$};
\node[main node] (4) [below = of 1] {$P_{SR_{AST}}$};
\node[main node] (5) [right = of 4, below = of 2] {$(P_{SL_{AST}},MEM_{AST})$};
\node[main node] (3) [below right = 1cm of 2] {$(P'_{CSP},MEM_{CSP})$};

% \node[main node] (6) [right = of 5, below = of 3] {$(P'_{CSP},MEM_{CSP})$};

\path[every node/.style={sloped,anchor=south,auto=false,font=\sffamily\small}]
(1) edge node {$\Omega$} (2)
(1) edge node {Haskell Repr} (4)
(2) edge [bend left] node [below] {$\Upsilon$} (3)
(4) edge node {} (1)
(5) edge node {Haskell Repr} (2)
(5) edge [bend right] node [below] {$\Upsilon_{Haskell}$} (3)
(4) edge node {$\Omega_{Haskell}$} (5);
\end{tikzpicture}
\end{frame}

% \begin{frame}
% \frametitle{Tasks for the PhD - Schedule table}
% \begin{figure}[htbp]
%   \centering
%     \includegraphics[width=0.9\textwidth]{figs/schedule-phd-table}
%   \caption{Schedule from 2016 until 2018}
%  \end{figure}
% \end{frame}

% \begin{frame}
% \frametitle{Tasks for the PhD - Schedule}
% \begin{figure}[htbp]
%   \centering
%     \includegraphics[width=\textwidth]{figs/schedule-phd-diagram}
%   \caption{Schedule from 2016 until 2018}
%  \end{figure}
% \end{frame}

\begin{frame}
\frametitle{Thank you}
  % \begin{itemize}
  %   \item Started the PhD in September 2014, whilst lecturing in Brazil, but arrived in Dublin in January 2015
  %   \item On leave until December 2017
  %   \item Can apply for an leave extension of 6 months
  %   \item Viva should happen during the leave
  % \end{itemize}
  \begin{figure}
  \includegraphics[width=0.7\textwidth]{sponsors}
  \end{figure}
\end{frame}

\end{document}
