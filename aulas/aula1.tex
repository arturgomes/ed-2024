\title{Estruturas de Dados: Apresentação da Disciplina}
\date{\today}
\frame{\titlepage}

\begin{frame}
  \frametitle{Ementa da Disciplina}
  Descrição do conteúdo a ser abordado na disciplina. Este conteúdo é importado automaticamente do Plano Pedagógico do Curso (PPC) vigente. Os tópicos principais incluem:
  \begin{itemize}
    \item Tabelas de Dispersão
    \item Árvores Binárias de Busca
    \item Árvores Balanceadas
    \item Busca Digital
    \item Processamento de Cadeias: Busca de Padrão e Compactação de Dados
  \end{itemize}
  Esta ementa é projetada para fornecer uma compreensão fundamental das estruturas de dados essenciais para a computação, com um foco especial em suas aplicações práticas e teóricas.
\end{frame}

\begin{frame}
  \frametitle{Objetivos Gerais}
  A disciplina visa:
  \begin{itemize}
    \item Compreender os fundamentos e a importância das estruturas de dados, enfatizando sua relevância para a eficiência dos algoritmos e o impacto no desempenho de software.
    \item Desenvolver habilidades de análise de complexidade, capacitando os alunos a analisar a complexidade de tempo e espaço de algoritmos associados a diferentes estruturas de dados.
  \end{itemize}
\end{frame}


\begin{frame}
  \frametitle{Objetivos Específicos - Tabelas de Dispersão}
  Os alunos deverão:
  \begin{itemize}
    \item Compreender o conceito, uso e implementação de tabelas de dispersão.
    \item Analisar e aplicar técnicas de tratamento de colisões.
  \end{itemize}
\end{frame}


\begin{frame}
  \frametitle{Objetivos Específicos - Árvores Binárias de Busca}
  Os alunos deverão:
  \begin{itemize}
    \item Entender a estrutura, propriedades e operações de árvores binárias de busca.
    \item Implementar inserção, busca, remoção e percurso em árvores binárias de busca.
  \end{itemize}
\end{frame}
\begin{frame}
  \frametitle{Objetivos Específicos - Árvores Balanceadas}
  Os alunos deverão:
  \begin{itemize}
    \item Aprender sobre árvores AVL e Red-Black, suas propriedades e operações.
    \item Desenvolver habilidades para implementar e manter árvores balanceadas.
  \end{itemize}
\end{frame}

\begin{frame}
  \frametitle{Objetivos Específicos - Busca Digital}
  Os alunos deverão:
  \begin{itemize}
    \item Entender o princípio da busca digital e sua aplicação em estruturas como Tries.
    \item Implementar árvores de busca digital para otimizar a busca por cadeias de caracteres.
  \end{itemize}
\end{frame}

\begin{frame}
  \frametitle{Objetivos Específicos - Processamento de Cadeias}
  Os alunos deverão:
  \begin{itemize}
    \item Compreender e aplicar algoritmos para busca de padrão em textos.
    \item Estudar e implementar técnicas de compactação de dados.
  \end{itemize}
\end{frame}

\begin{frame}
  \frametitle{Semana 1: Introdução (04/03 - 10/03)}
  \textbf{Objetivos:} Apresentar o curso, objetivos, metodologia e avaliação. Introdução aos conceitos básicos de estruturas de dados.
  
  \textbf{Leitura Recomendada:} Capítulos introdutórios dos livros de Cormen e Ascencio.
\end{frame}

\begin{frame}
  \frametitle{Semana 2: Tabelas de Dispersão (11/03 - 17/03)}
  \textbf{Tópicos:} Conceitos básicos, funções hash, tratamento de colisões.
  
  \textbf{Prática:} Exercícios de implementação de funções hash simples.
\end{frame}

\begin{frame}
  \frametitle{Semana 3: Listas, Filas, Pilhas (18/03 - 24/03)}
  \textbf{Tópicos:} Estrutura de uma lista, e os métodos de busca, inserção e remoção.
  
  \textbf{Prática:} Implementação de listas, filas e pilhas, e reuso de funções.
\end{frame}

\begin{frame}
  \frametitle{Semana 4: Árvores Binárias de Busca - Parte 1 (25/03 - 31/03)}
  \textbf{Tópicos:} Conceitos, inserção, busca e percurso.
  
  \textbf{Prática:} Implementação de árvores binárias de busca.
\end{frame}

\begin{frame}
  \frametitle{Semana 5: Árvores Binárias de Busca - Parte 2 (01/04 - 07/04)}
  \textbf{Tópicos:} Remoção, propriedades e aplicações práticas.
  
  \textbf{Prática:} Exercícios de complexidade e otimização.
\end{frame}

\begin{frame}
  \frametitle{Semana 6: Árvores Balanceadas - AVL (08/04 - 14/04)}
  \textbf{Tópicos:} Introdução, rotações, inserção e remoção.
  
  \textbf{Prática:} Implementação de árvores AVL.
\end{frame}

\begin{frame}
  \frametitle{Semana 7: Árvores Balanceadas - Red-Black (15/04 - 21/04)}
  \textbf{Tópicos:} Conceitos, propriedades, operações.
  
  \textbf{Prática:} Implementação de árvores Red-Black.
\end{frame}

\begin{frame}
  \frametitle{Semana 8: Busca Digital (22/04 - 28/04)}
  \textbf{Tópicos:} Trie, operações básicas, aplicabilidade.
  
  \textbf{Prática:} Implementação de Tries para busca de padrões.
\end{frame}

\begin{frame}
  \frametitle{Semana 9: Revisão e Avaliação Intermediária (29/04 - 05/05)}
  \textbf{Atividades:} Primeira prova teórica/prática.
\end{frame}

\begin{frame}
  \frametitle{Semana 10: Processamento de Cadeias - Busca de Padrão (06/05 - 12/05)}
  \textbf{Tópicos:} Algoritmos de busca de padrões, KMP, Boyer-Moore.
  
  \textbf{Prática:} Implementação e análise de algoritmos de busca.
\end{frame}

\begin{frame}
  \frametitle{Semana 11: Processamento de Cadeias - Compactação de Dados (13/05 - 19/05)}
  \textbf{Tópicos:} Técnicas de compactação, Huffman, LZW.
  
  \textbf{Prática:} Implementação de algoritmos de compactação.
\end{frame}

\begin{frame}
  \frametitle{Semana 12-13: Projetos Práticos (20/05 - 02/06)}
  \textbf{Atividades:} Desenvolvimento de projetos práticos em pequenos grupos, aplicando as estruturas de dados estudadas.
\end{frame}

\begin{frame}
  \frametitle{Semana 14: Aplicações Avançadas e Estudo de Caso (03/06 - 09/06)}
  \textbf{Tópicos:} Discussão de estudos de caso, uso de estruturas de dados em sistemas reais.
\end{frame}

\begin{frame}
  \frametitle{Semana 15: Revisão Geral (10/06 - 16/06)}
  \textbf{Atividades:} Revisão de todos os tópicos, sessões de dúvidas.
\end{frame}

\begin{frame}
  \frametitle{Semana 16-17: Apresentação dos Projetos Práticos (17/06 - 30/06)}
  \textbf{Atividades:} Apresentação e avaliação dos projetos práticos. Feedback individual e em grupo.
\end{frame}

\begin{frame}
  \frametitle{Semana 18: Avaliação Final e Encerramento (01/07 - 05/07)}
  \textbf{Atividades:} Avaliação final teórica. Encerramento do curso, feedback e discussão sobre aprendizados.
\end{frame}
\begin{frame}
  \frametitle{Metodologia da Disciplina}
  A disciplina será conduzida através de:
  \begin{itemize}
    \item Aulas Expositivas: Utilização do quadro e giz, e projetor para explicar conceitos, desenhar estruturas e demonstrar algoritmos.
    \item Práticas de Programação: Exercícios em sala de aula e laboratório de computação para aplicação prática dos conceitos.
  \end{itemize}
\end{frame}
\begin{frame}
  \frametitle{Estudo de Casos e Projetos Práticos}
  A disciplina incluirá:
  \begin{itemize}
    \item Análise de Casos Reais: Discussão sobre a aplicação de estruturas de dados em problemas reais.
    \item Desenvolvimento de Projetos: Trabalho em projetos práticos em grupos para solucionar desafios complexos.
  \end{itemize}
\end{frame}
\begin{frame}
  \frametitle{Avaliação e Feedback}
  Para promover o aprendizado efetivo, será aplicado:
  \begin{itemize}
    \item Avaliações Contínuas: Quizzes, exercícios de programação e participação.
    \item Feedback Construtivo: Orientações individuais e em grupo sobre as atividades.
  \end{itemize}
\end{frame}
\begin{frame}
  \frametitle{Recursos Complementares}
  Para aprofundamento do conhecimento:
  \begin{itemize}
    \item Materiais Online e Leituras Adicionais: Artigos, tutoriais, fóruns, além das leituras dos livros-texto.
  \end{itemize}
\end{frame}
\begin{frame}
  \frametitle{Bibliografia Básica}
  Livros essenciais para o estudo da disciplina:
  \begin{itemize}
    \item ASCENCIO, Ana Fernanda Gomes. \emph{Estruturas de dados: algoritmos, análise da complexidade e implementações em Java e C/C++}. São Paulo, SP: Pearson, 2010-2012. ISBN 978-85-7605-881-6.
    \item CORMEN, Thomas H. \emph{Algoritmos: teoria e prática}. Rio de Janeiro, RJ: Elsevier, 2012. ISBN 978-85-352-3699-6.
    \item EDMONDS, J. \emph{How to Think About Algorithms}. 1.ed. Cambridge: Cambridge University Press, 2008. ISBN: 978-0521614108.
  \end{itemize}
\end{frame}
\begin{frame}
  \frametitle{Bibliografia Complementar}
  Leituras adicionais para aprofundamento:
  \begin{itemize}
    \item DEITEL, Harvey M.; DEITEL, P. J. \emph{C++: como programar}. 5. ed. Porto Alegre, RS: Bookman, 2006. ISBN 85-7307-740-9.
    \item GUSFIELD, D. \emph{Algorithms on strings trees and sequences}. Cambridge: Cambridge University Press, 1997. ASIN: B009NG2XWA.
    \item KLEINBERG, Jon; TARDOS, Éva. \emph{Algorithm design}. Boston, MA: Pearson, c2014. ISBN 0321295358.
  \end{itemize}
\end{frame}
