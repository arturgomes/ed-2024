\section{Remoção em Arvores AVL}
\subsection{Introdução}
\begin{frame}[noframenumbering,plain,fragile]{\secname : \subsecname}

  \begin{itemize}
  \item Assim como na inserção da AVL, temos que adaptar nosso algoritmo da BST para satisfazer as propriedades de balanceamento.
  \item Após a inserção, calculamos o nível entre sub-árvore esquerda e sub-árvore direita, 
  \item Caso essa diferença seja maior que 1, então realizamos rotações para corrigir o balanceamento
\end{itemize}


  
\end{frame}
%%%%%%%%%%%%%%%%
%%% frame %%%%%%
%%%%%%%%%%%%%%%%

\begin{frame}[noframenumbering,plain,fragile]{\secname : \subsecname}

  \begin{itemize}
  \item No caso da remoção na AVL, o procedimento também é o mesmo da BST
  \item E no final, calculamos o nível entre as subárvores
  \item E finalizamos com balanceamento, caso seja necessario
  \item VAMOS DIRETO PARA OS EXEMPLOS?
  \end{itemize}
\end{frame}
